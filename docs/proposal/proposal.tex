

\documentclass{article}
\usepackage[utf8]{inputenc}
\usepackage{authblk}
\usepackage{setspace}
\usepackage{natbib}
\usepackage{hyperref}
%\usepackage{cite}
\usepackage[margin=1in]{geometry}
\usepackage{array}
\usepackage{graphicx}
\usepackage{caption}
\graphicspath{ {./figures/} }
\usepackage{subcaption}
\usepackage{amsmath}
\usepackage{lineno}
\usepackage{soul}
\usepackage{xcolor}
\sethlcolor{yellow}
\linenumbers

%%%%%%%%%%%%%%%%%
\title{Thesis outline}
\date{\today}
\author{Christophe Rouleau-Desrochers}
\begin{document}
%%%%%%%%%%%%%%%%%%%%%%%%%%%%%%

\maketitle


%<><><><><><><><><><><><><><><><><><><><>
% Introduction
%<><><><><><><><><><><><><><><><><><><><>
\section{Introduction}
Research from the past two decades have shown increasing evidence that human activity keeps affecting many worldwide environmental processes. This is shown by the increasing impact of invasive species, their corresponding loss of biodiversity which is furthermore affected by its main driver, habitat loss and framentation. That alone raises major concern and actions have been deployed to mitigate these impacts. Human activity, notably their greenhouse gas emissions may have long-lasting consequences, for which predictions by the IPCC have been overwhelmingly alarming since some of their reports have been shown to have been to pesimistic. Climate change currently holds the status of a scientific consensus i.e. scientifics arounds the world, experts in their domain all agree that climate change happens and the speed and the magnitude at which it happens is caused by human activity. However, how climate change impacts thousands of environmental and social processes worldwide is to be discussed with precaution as attribution of its impacts lacks evidence for the most part. \\ 
One of the first attribution of climate change impacts on biological systems has been shown in the early 2000s and since then, a lot of ink has been leaked on the impacts of climate change on the phenology of multiple organisms. Bird nesting, tree budburst, fruit ripening, etc., have all been shown to be affected in some ways by climate change. We are still unraveling how the complex nature of climate will affect each organisms and the truth is that at this stage, we have very little idea. The best tool to make projections and predictions that could be useful in climate action and midications is still increasing data quality and availability and developping better model performance. However, we have very few knowledge on how species will react to warmer temperatures and changes of precipitation regimes. For this, better technology may increase observational data availability and quality for a lot of species, but experiments remain a crucial tool where covarying factors can be removed in order to focus on the parameters we want to investigate the most. \\
Spring and fall phenological events are shifting. These have been shown to happen around the world, but the consequences on spring phenology, while there has been a lot of speculations and hypothesis made, we have no idea of the patern by which these shifts in phenology will affect species around the world. More precisely, spring phenological events have been advancing at a rate of **/decade and autumn phenophases show either a delay or no trend. The former is mainly driven by temperature, for which a lot of research has been conducted. The drivers of the later are far less known for many reasons. Autumn phenophases have attracted less attention, the data is usually much more noisy since meteorological conditions in the fall can drastically influence the quality of the phenological data. E.g. if monitoring canopy leaf drop over a 2 monh period. While trees go through leaf absission, and the separation might not be complete, rain, strong, but even mild winds can advance leaf drop and thus affect data quality. However, the belief is that autumn phenophase, such as budset, leaf colouring and leaf drop are driven by shortening photoperiod and colder temperatures. 
% ============================
% 1. CC  X Phenology
% ============================
\subsection{Climate change impacts on tree phenology} 
Climate change impacts on biological systems and how phenological trends are already shifting with warming temperatures. 

%%%
\begin{table}[p]
\centering
\caption{Fuelinex species grouped by tree type, life history, and wood anatomy.}
\begin{tabular}{|>{\raggedright\arraybackslash}p{7cm}|p{5cm}|p{3cm}|p{1cm}|}
\hline
\multicolumn{4}{|c|}{\textbf{Deciduous Trees}} \\
\hline
\textbf{Common Name (Latin)} & \textbf{Life History Strategy} & \textbf{Wood Anatomy} & \textbf{n (approx)} \\
\hline
Bur oak (\textit{Quercus macrocarpa}) & Slow-growth, long life & Ring-porous & 87\\
Bitter cherry (\textit{Prunus virginiana}) & Fast-growth, short life & Diffuse-porous & 78\\
Box elder (\textit{Acer negundo}) & Fast-growth, short life  & Diffuse-porous & 90\\
Balsam poplar (\textit{Populus balsamifera}) & Fast-growth, short life  & Diffuse-porous &84 \\
Paper birch (\textit{Betula papyrifera}) & Fast-growth, short life  & Diffuse-porous &90\\
\hline
\multicolumn{4}{|c|}{\textbf{Evergreen Trees}} \\
\hline
White pine (\textit{Pinus strobus}) & Slow-growth, long life & & 89\\
Giant Sequoia (\textit{Sequoiadendron giganteum}) & Slow-growth, long life & & 54\\
\hline
\end{tabular}
\end{table}
%%%
% ============================
% Wildchrokie
% ============================
\subsection{Wildchrokie}
\begin {enumerate}
	\item Common garden from 2015 to 2023
	\item Four species within the Betulacea family (Table 2)
%%%
\begin{table}[p]
\centering
\caption{Wilchrokie species grouped by tree type, life history, and wood anatomy.}
\begin{tabular}{|>{\raggedright\arraybackslash}p{7cm}|p{5cm}|p{3cm}|p{1cm}|}
\hline
\multicolumn{4}{|c|}{\textbf{Deciduous Trees}} \\
\hline
\textbf{Common Name (Latin)} & \textbf{Life History Strategy} & \textbf{Wood Anatomy} & \textbf{n} \\
\hline
Paper birch (\textit{Betula papyrifera}) & Fast-growth, short life  & Diffuse-porous & 8\\
Yellow birch (\textit{Betula alleghaniensis}) & Moderate-growth, moderate life & Diffuse-porous & 21\\
Grey birch (\textit{Betula populifolia}) & Fast-growth, short life & Diffuse-porous & 29\\
Grey alder (\textit{Alnus incana}) & Fast-growth, short life & Diffuse-porous & 31\\
\hline
\end{tabular}
\end{table}
%%%
	\item Data: phenology, height, tree rings
	\item Analysis: Hierarchical model to understand how tree ring width relates to GDD
\end {enumerate}

% ============================
% Treespotters
% ============================
\subsection{Treespotters}
\begin {enumerate}
	\item Citizen science project from 2015 to today (Table 3)
\begin{table}[h]
\centering
\caption{Treespotters species grouped by tree type, life history, and wood anatomy.}
\begin{tabular}{|>{\raggedright\arraybackslash}p{7cm}|p{5cm}|p{3cm}|p{1cm}|}
\hline
\multicolumn{4}{|c|}{\textbf{Deciduous Trees}} \\
\hline
\textbf{Common Name (Latin)} & \textbf{Life History Strategy} & \textbf{Wood Anatomy} & \textbf{n} \\
\hline
American basswood (\textit{Tilia americana}) & Fast-growth, moderate life & Diffuse-porous & 5\\
Eastern cottonwood (\textit{Populus deltoides}) & Fast-growth, short life & Diffuse-porous & 4\\
Northern red oak (\textit{Quercus rubra}) & Moderate-growth, long life & Ring-porous & 4\\
White oak (\textit{Quercus alba}) & Slow-growth, long life & Ring-porous & 5\\
Pignut hickory (\textit{Carya glabra}) & Slow-growth, long life & Ring-porous & 4\\
Shagbark hickory (\textit{Carya ovata}) & Slow-growth, long life & Ring-porous & 4\\
River birch (\textit{Betula nigra}) & Fast-growth, short life & Diffuse-porous & 5\\
Yellow birch (\textit{Betula alleghaniensis}) & Moderate-growth, moderate life & Diffuse-porous & 4\\
Sugar maple (\textit{Acer saccharum}) & Slow-growth, long life & Diffuse-porous & 5\\
Red maple (\textit{Acer rubrum}) & Slow-growth, long life & Diffuse-porous & 4\\
Yellow buckeye (\textit{Aesculus flava}) & Moderate-growth, moderate life & Diffuse-porous & 5\\
\hline
\end{tabular}
\end{table}

	\item Tree coring
	\item Data: phenology, tree rings
	\item Analysis: Hierarchical model to understand how tree ring width relates to GDD	
\end {enumerate}


\section {References}
\bibliography{Exported_Items.bib}
\bibliographystyle{nature} % set citation style 


% old text
%\section*{Chapter 2.1. Wildchrokie: Phenological observations coupling with tree-ring width measurements for a 6-year common garden experiment} While the assumption that a longer growing season leads to increased growth is an intuitive and common one, recent evidence shows that this may not be the case. 
%How phenological season length relates to growth. This is a major question in fundamental biology, but also critical to forecasts of climate change itself, since most carbon models assume that plants experiencing longer seasons will sequester more carbon, but recent studies have called this assumption into question.


% ============================
% 2. Tree-ring stuff 
% ============================
%\subsection{Tree rings measurements as a proxy for growth}
%Using tree ring data to investigate the relationship between phenology and growth

%\begin{enumerate}
% 1)
%	\item Triggers and mechanisms behind growth onset, duration and rate.
%Growth onset and duration vary because of inter-annual differences in weather, with cambium reactivation in spring being highly dependent on temperature. 
% 2)
%	\item How radial growth is influenced by extreme weather events and their timing. 
%Growh rate increased by temperature, depends on \textbf{when} it is warmer. 
% Include in main text: Long seasons at low temperature will produce fewer cell rows than at warmer temperature.
% 3) 
%	\item Which is more important? How fast does a tree grow, or how long does it grow for?
%how growth rate may have a more direct influence on tree growth than the growing season length. 
% 4)
%	\item Methods to measure tree growth and why using tree ring images may better capture tree growth response than traditional diameter and height measurements.
% Include in main text: Diameter and height measurements are widely used to assess yearly biomass increment. However, these measurements are punctual and are often the cumulative result of many climatic events and constraints that occur during a tree's lifespan
%\end{enumerate}



\end{document}
