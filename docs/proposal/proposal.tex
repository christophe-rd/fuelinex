

\documentclass{article}
\usepackage[utf8]{inputenc}
\usepackage{authblk}
\usepackage{setspace}
\usepackage{natbib}
\usepackage{hyperref}
%\usepackage{cite}
\usepackage[margin=1in]{geometry}
\usepackage{array}
\usepackage{graphicx}
\usepackage{caption}
\graphicspath{ {./figures/} }
\usepackage{subcaption}
\usepackage{amsmath}
\usepackage{lineno}
\usepackage{soul}
\usepackage{xcolor}
\sethlcolor{yellow}
\linenumbers

%%%%%%%%%%%%%%%%%
\title{Thesis outline}
\date{\today}
\author{Christophe Rouleau-Desrochers}
\begin{document}
%%%%%%%%%%%%%%%%%%%%%%%%%%%%%%

\maketitle


%<><><><><><><><><><><><><><><><><><><><>
% Introduction
%<><><><><><><><><><><><><><><><><><><><>
\section{Introduction}
Research from the past two decades have shown increasing evidence that human activity keeps affecting many worldwide environmental processes. This is shown by the increasing impact of invasive species, their corresponding loss of biodiversity which is furthermore affected by its main driver, habitat loss and framentation. That alone raises major concern and actions have been deployed to mitigate these impacts. Human activity, notably their greenhouse gas emissions may have long-lasting consequences, for which predictions by the IPCC have been overwhelmingly alarming since some of their reports have been shown to have been to pesimistic. Climate change currently holds the status of a scientific consensus i.e. scientifics arounds the world, experts in their domain all agree that climate change happens and the speed and the magnitude at which it happens is caused by human activity. However, how climate change impacts thousands of environmental and social processes worldwide is to be discussed with precaution as attribution of its impacts lacks evidence for the most part. 

\subsection {Trends of spring and autumn phenological events and their drivers}
The most frequently observed biological impact of climate change over the past decades are major changes on spring and autumn phenology --- the timing of recurring life history events \cite{parmesan_globally_2003,cleland_shifting_2007,lieth_phenology_1974,woolway_phenological_2021,menzel_european_2006}. Understanding the consequences of these shifts on ecosystems requires understanding how much the growing season has changed \cite{duputie_phenological_2015}. Spring phenological events (e.g. budburst and leafout) have been advancing from 0.5 \cite{wolfe_climate_2005} to 4.2 days/decade \cite{chmielewski_response_2001,fu_recent_2014} and are mainly driven by temperature \cite{chuine_why_2010,cleland_shifting_2007,penuelas_responses_2001}. In contrast, autumn phenology (e.g. budset and leaf colouring) is delayed, though to a much lesser extent than spring \cite{gallinat_autumn_2015,jeong_macroscale_2014}. The drivers regulating autumn phenology are far less understood than those of spring for many reasons. First, autumn phenology has attracted much less attention compared to spring \cite{piao_plant_2019}. Second, the data is often much noisier, since meteorological conditions in the fall can drastically influence the phenology phenomena (e.g. trees going through leaf senescence are subjected to a gradual leaf abscission, and the leaves might be at different abscission stage, but a strong wind spell may trigger leaf drop for all leaves, thus affecting data quality. However, the belief is that autumn phenophases are driven by shortening photoperiod and colder temperatures \cite{cooke_dynamic_2012,flynn_temperature_2018,korner_phenology_2010} and colder temperatures \cite{cooke_dynamic_2012,delpierre_temperate_2016}. 

\subsection{Evidence of declining sensitivity to warming}

\subsection{Mechanisms that could limit growth despite having a longer growing season}
I hypothesize two possible drivers that could explain why a longer growing season might not lead to increased growth: external (environmental) \cite{kolar_response_2016} or internal (via physiological constraints)\cite{zohner_effect_2023} limits to growth. The complex nature of climate change makes predicting the external drivers to growth hard to quantify at the individual level, as these drivers affect communities as a whole. Drought, spring frost and heat waves are commonly mentioned as the main extreme events that could limit tree growth under climate change \cite{tyree_xylem_2002, choat_triggers_2018, li_widespread_2023,trenberth_global_2014,intergovernmental_panel_on_climate_change_detection_2014,chiang_evidence_2021,polgar_leafout_2011,reinmann_compensatory_2023} (See tables). To better understand these mechanisms, experiments are paramount to robustly tease apart the external vs internal drivers (e.g. warmer springs from severe drought later in the season---a common co-occurring reality in natural environments) \cite{morin_changes_2010,primack_observations_2015}. This is essential to refine forest carbon sequestration projections \cite{green_limits_2022,cabon_cross-biome_2022}. However, experiments are most often performed on juvenile trees, which are critical for their role in forest regeneration projections, but their responses can hardly be translated to mature trees, which hold the overwhelming carbon biomass proportion of forests \cite{augspurger_differences_2003,silvestro_longer_2023,vitasse_ontogenic_2013}. \cite{berra_assessing_2019,piao_plant_2019,teng_bringing_2025}. Thus, I propose to use a combination of one experiment to test internal (Chapter 1) limits to growth along with two observational studies (Chapter 2). This will allow me to address the paradox of the absence of increased growth despite apparently improved growing season conditions. 

\subsection{How these shifts translate into effects on trees/forests are not clear}
Shifts in spring and autumn phenology support a long-lasting and intuitive assumption that earlier spring and delayed autumn events lead to longer seasons---and thus increased growth \cite{keenan_net_2014}. However, research from the past three years has cast doubt on this hypothesis \cite{dow_warm_2022,green_limits_2022,silvestro_longer_2023}. Recently, Dow \textit{et al}. (2022) showed that despite an earlier growth onset, neither growth rate nor overall annual increment was increased by longer seasons. This could substantially affect carbon-cycle model projections and thus feedbacks to future climate \cite{richardson_climate_2013,swidrak_comparing_2013}. 
Understanding these findings requires answering why trees do not grow more despite longer growing seasons. 

Spring and fall phenological events are shifting. These have been shown to happen around the world, but the consequences on spring phenology, while there has been a lot of speculations and hypothesis made, we have no idea of the patern by which these shifts in phenology will affect species around the world. More precisely, spring phenological events have been advancing at a rate of **/decade and autumn phenophases show either a delay or no trend. 
% ============================
% 1. CC  X Phenology
% ============================
\subsection{Climate change impacts on tree phenology} 
Climate change impacts on biological systems and how phenological trends are already shifting with warming temperatures. 

%%%
\begin{table}[p]
\centering
\caption{Fuelinex species grouped by tree type, life history, and wood anatomy.}
\begin{tabular}{|>{\raggedright\arraybackslash}p{7cm}|p{5cm}|p{3cm}|p{1cm}|}
\hline
\multicolumn{4}{|c|}{\textbf{Deciduous Trees}} \\
\hline
\textbf{Common Name (Latin)} & \textbf{Life History Strategy} & \textbf{Wood Anatomy} & \textbf{n (approx)} \\
\hline
Bur oak (\textit{Quercus macrocarpa}) & Slow-growth, long life & Ring-porous & 87\\
Bitter cherry (\textit{Prunus virginiana}) & Fast-growth, short life & Diffuse-porous & 78\\
Box elder (\textit{Acer negundo}) & Fast-growth, short life  & Diffuse-porous & 90\\
Balsam poplar (\textit{Populus balsamifera}) & Fast-growth, short life  & Diffuse-porous &84 \\
Paper birch (\textit{Betula papyrifera}) & Fast-growth, short life  & Diffuse-porous &90\\
\hline
\multicolumn{4}{|c|}{\textbf{Evergreen Trees}} \\
\hline
White pine (\textit{Pinus strobus}) & Slow-growth, long life & & 89\\
Giant Sequoia (\textit{Sequoiadendron giganteum}) & Slow-growth, long life & & 54\\
\hline
\end{tabular}
\end{table}
%%%
% ============================
% Wildchrokie
% ============================
\subsection{Wildchrokie}
\begin {enumerate}
	\item Common garden from 2015 to 2023
	\item Four species within the Betulacea family (Table 2)
%%%
\begin{table}[p]
\centering
\caption{Wilchrokie species grouped by tree type, life history, and wood anatomy.}
\begin{tabular}{|>{\raggedright\arraybackslash}p{7cm}|p{5cm}|p{3cm}|p{1cm}|}
\hline
\multicolumn{4}{|c|}{\textbf{Deciduous Trees}} \\
\hline
\textbf{Common Name (Latin)} & \textbf{Life History Strategy} & \textbf{Wood Anatomy} & \textbf{n} \\
\hline
Paper birch (\textit{Betula papyrifera}) & Fast-growth, short life  & Diffuse-porous & 8\\
Yellow birch (\textit{Betula alleghaniensis}) & Moderate-growth, moderate life & Diffuse-porous & 21\\
Grey birch (\textit{Betula populifolia}) & Fast-growth, short life & Diffuse-porous & 29\\
Grey alder (\textit{Alnus incana}) & Fast-growth, short life & Diffuse-porous & 31\\
\hline
\end{tabular}
\end{table}
%%%
	\item Data: phenology, height, tree rings
	\item Analysis: Hierarchical model to understand how tree ring width relates to GDD
\end {enumerate}

% ============================
% Treespotters
% ============================
\subsection{Treespotters}
\begin {enumerate}
	\item Citizen science project from 2015 to today (Table 3)
\begin{table}[h]
\centering
\caption{Treespotters species grouped by tree type, life history, and wood anatomy.}
\begin{tabular}{|>{\raggedright\arraybackslash}p{7cm}|p{5cm}|p{3cm}|p{1cm}|}
\hline
\multicolumn{4}{|c|}{\textbf{Deciduous Trees}} \\
\hline
\textbf{Common Name (Latin)} & \textbf{Life History Strategy} & \textbf{Wood Anatomy} & \textbf{n} \\
\hline
American basswood (\textit{Tilia americana}) & Fast-growth, moderate life & Diffuse-porous & 5\\
Eastern cottonwood (\textit{Populus deltoides}) & Fast-growth, short life & Diffuse-porous & 4\\
Northern red oak (\textit{Quercus rubra}) & Moderate-growth, long life & Ring-porous & 4\\
White oak (\textit{Quercus alba}) & Slow-growth, long life & Ring-porous & 5\\
Pignut hickory (\textit{Carya glabra}) & Slow-growth, long life & Ring-porous & 4\\
Shagbark hickory (\textit{Carya ovata}) & Slow-growth, long life & Ring-porous & 4\\
River birch (\textit{Betula nigra}) & Fast-growth, short life & Diffuse-porous & 5\\
Yellow birch (\textit{Betula alleghaniensis}) & Moderate-growth, moderate life & Diffuse-porous & 4\\
Sugar maple (\textit{Acer saccharum}) & Slow-growth, long life & Diffuse-porous & 5\\
Red maple (\textit{Acer rubrum}) & Slow-growth, long life & Diffuse-porous & 4\\
Yellow buckeye (\textit{Aesculus flava}) & Moderate-growth, moderate life & Diffuse-porous & 5\\
\hline
\end{tabular}
\end{table}

	\item Tree coring
	\item Data: phenology, tree rings
	\item Analysis: Hierarchical model to understand how tree ring width relates to GDD	
\end {enumerate}


\section {References}
\bibliography{Exported_Items.bib}
\bibliographystyle{nature} % set citation style 


% old text
%\section*{Chapter 2.1. Wildchrokie: Phenological observations coupling with tree-ring width measurements for a 6-year common garden experiment} While the assumption that a longer growing season leads to increased growth is an intuitive and common one, recent evidence shows that this may not be the case. 
%How phenological season length relates to growth. This is a major question in fundamental biology, but also critical to forecasts of climate change itself, since most carbon models assume that plants experiencing longer seasons will sequester more carbon, but recent studies have called this assumption into question.


% ============================
% 2. Tree-ring stuff 
% ============================
%\subsection{Tree rings measurements as a proxy for growth}
%Using tree ring data to investigate the relationship between phenology and growth

%\begin{enumerate}
% 1)
%	\item Triggers and mechanisms behind growth onset, duration and rate.
%Growth onset and duration vary because of inter-annual differences in weather, with cambium reactivation in spring being highly dependent on temperature. 
% 2)
%	\item How radial growth is influenced by extreme weather events and their timing. 
%Growh rate increased by temperature, depends on \textbf{when} it is warmer. 
% Include in main text: Long seasons at low temperature will produce fewer cell rows than at warmer temperature.
% 3) 
%	\item Which is more important? How fast does a tree grow, or how long does it grow for?
%how growth rate may have a more direct influence on tree growth than the growing season length. 
% 4)
%	\item Methods to measure tree growth and why using tree ring images may better capture tree growth response than traditional diameter and height measurements.
% Include in main text: Diameter and height measurements are widely used to assess yearly biomass increment. However, these measurements are punctual and are often the cumulative result of many climatic events and constraints that occur during a tree's lifespan
%\end{enumerate}



\end{document}
