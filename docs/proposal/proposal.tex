

\documentclass{article}
\usepackage[utf8]{inputenc}
\usepackage{authblk}
\usepackage{setspace}
\usepackage{natbib}
\usepackage{hyperref}
%\usepackage{cite}
\usepackage[margin=1in]{geometry}
\usepackage{array}
\usepackage{graphicx}
\usepackage{caption}
\graphicspath{ {./figures/} }
\usepackage{subcaption}
\usepackage{amsmath}
\usepackage{lineno}
\usepackage{soul}
\usepackage{xcolor}
\sethlcolor{yellow}
\linenumbers

%%%%%%%%%%%%%%%%%
\title{Thesis Proposal}
\date{\today}
\author{Christophe Rouleau-Desrochers}
\begin{document}
%%%%%%%%%%%%%%%%%%%%%%%%%%%%%%

\maketitle


%<><><><><><><><><><><><><><><><><><><><>
% INTRODUCTION %
%<><><><><><><><><><><><><><><><><><><><>
\section{Introduction}
% <><><><><><><><><><><><><><><><><><><><><><><><><><><><><><>
% SECTION 1.1. %
% <><><><><><><><><><><><><><><><><><><><><><><><><><><><><><>
\subsection{Climate change impacts on tree phenology}
Research from the past two decades have shown increasing evidence that human activity keeps affecting many worldwide environmental processes. This is shown by the increasing impact of invasive species, their corresponding loss of biodiversity which is furthermore affected by its main driver, habitat loss and framentation. That alone raises major concern and actions have been deployed to mitigate these impacts. Human activity, notably their greenhouse gas emissions may have long-lasting consequences, for which predictions by the IPCC have been overwhelmingly alarming since some of their reports have been shown to have been to pesimistic. Climate change currently holds the status of a scientific consensus i.e. scientifics arounds the world, experts in their domain all agree that climate change happens and the speed and the magnitude at which it happens is caused by human activity. However, how climate change impacts thousands of environmental and social processes worldwide is to be discussed with precaution as attribution of its impacts lacks evidence for the most part. 
% <><><><><><><><><><><><><><><><><><><><><><><><><><><><><><>

% <><><><><><><><><><><><><><><><><><><><><><><><><><><><><><>
\textbf{Trends of spring and autumn phenological events and their drivers}
The most frequently observed biological impact of climate change over the past decades are major changes on spring and autumn phenology --- the timing of recurring life history events \cite{parmesan_globally_2003,cleland_shifting_2007,lieth_phenology_1974,woolway_phenological_2021,menzel_european_2006}. Understanding the consequences of these shifts on ecosystems requires understanding how much the growing season has changed \cite{duputie_phenological_2015}. Spring phenological events (e.g. budburst and leafout) have been advancing from 0.5 \cite{wolfe_climate_2005} to 4.2 days/decade \cite{chmielewski_response_2001,fu_recent_2014} and are mainly driven by temperature \cite{chuine_why_2010,cleland_shifting_2007,penuelas_responses_2001}. In contrast, autumn phenology (e.g. budset and leaf colouring) is delayed, though to a much lesser extent than spring \cite{gallinat_autumn_2015,jeong_macroscale_2014}. The drivers regulating autumn phenology are far less understood than those of spring for many reasons. First, autumn phenology has attracted much less attention compared to spring \cite{piao_plant_2019}. Second, the data is often much noisier, since meteorological conditions in the fall can drastically influence the phenology phenomena (e.g. trees going through leaf senescence are subjected to a gradual leaf abscission, and the leaves might be at different abscission stage, but a strong wind spell may trigger leaf drop for all leaves, thus affecting data quality. However, the belief is that autumn phenophases are driven by shortening photoperiod and colder temperatures \cite{cooke_dynamic_2012,flynn_temperature_2018,korner_phenology_2010} and colder temperatures \cite{cooke_dynamic_2012,delpierre_temperate_2016}. Different hypothesis are proposed to explain delayed autumn phenophases. First, warmer autumn temperature may extend the activity of photosynthetic enzymes which could be maintained at a a higher level. Thus, the degradation rate of chlorophyll would decrease and timing of senescence would be delayed (Tao2021Divergent). Another mechanism would be through summer conditions... However, a counterinteractive hypothesis suggests an 
antagonistic effect of warming and brightening---caused by reductions in atmospheric pollution and cloud  cover (Sanchez2015Reassessment)--- on leaf senescence (Wu2021Atmospheric). Brightening accelerates the leaf senescence process and reduces the temperature sensitivity of leaf senescence, counteracting the expected warming-induced delays in leaf senescence. The photo-protection and sink limitation hypothesis provide plausible explanations for the negative effect of radiation on leaf senescence and the declining effect of temperature sensitivity of leaf senescence in response to brightening \\
% <><><><><><><><><><><><><><><><><><><><><><><><><><><><><><>
\textbf{Evidence of declining sensitivity to warming}
% this section needs some heavy work!
Bud dormancy needs to be broken by exposure to cold temperatures during the dormancy period (chilling).

The advancement of spring phenology that has been observed over the past decades might decelerate and this relates to winter dormancy. In the fall, trees in boreal and temperate forests slowly enter dormancy which is initiated with budset. During this phases, cold hardiness increases which prepares the trees for the upcoming cold temperatures and prevents tissue damage. Then, the trees go through dormancy in the winter, during which a certain duration of chilling temperatures between 0 and 5 °C ---with some interaction with photoperiod for some species---- is necessary for the trees to be ready to accumulate forcing (Vitasse2014TheInteraction). In the late winter and early spring, they go through two forms of deacclimation before the buds are ready to burst (Vitasse2014TheInteraction). Trees require a certain amount of heat (forcing) after they come out of the dormancy period to initiate leaf unfolding in spring. Heat requirement is met sooner in warm springs. However, heat requirement is negatively correlated with chilling (Yongshuo2015Declining), but it is this interaction between chilling and forcing requirements that determines the timing of leaf unfolding. Consequently, a decrease in chilling accumulation could explain the observed weaker spring temperature sensitivities as a decline in the relative importance of warm spring temperatures for spring phenological events in the temperate zone, as other environmental triggers (e.g. winter temperatures that determine "chilling") play a larger role (Wolkovich2021ASimple). 


For an observer it remains hidden whether a plant is in endo- or ecodormancy. To predict phenological events, one needs to account for both, internal and external determinants of dormancy (e.g., Basler \& Körner, 2012, 2014; Lundell et al., 2020). Because the drivers of endodormancy and ecodormancy, i.e., the chilling dose, photoperiod, and thermal forcing, interact non-linearly and species-specific, efforts for a general theory for modelling phenology failed so far (Basler, 2016).

In addition, determining the dormancy of a specific individual is complicated because one needs to account for the internal and external determinants of dormancy which vary by the chilling dose, photoperiod and forcing, interacting non-linearly. A general phylogenetic rule for modelling phenology failed because of these constaints. (***not sure of that) (Basler2016, Korner2023Four). \\

% <><><><><><><><><><><><><><><><><><><><><><><><><><><><><><>
\textbf{Mechanisms that could limit growth despite having a longer growing season}
Plants seasonal activity has internal and external controls, both determined by environmental conditions (Korner2023Four). Internal controls operate via autonomous clocks, activating genes, regulatory loops, and releasing hormones, often integrating chilling or photoperiod signals. The external controls (often addressed as ‘forcing’) act directly on the rate of development, meristem activity, tissue differentiation and metabolism. Understanding these controls is critical to justify how a plant can adjust they activity schedule to change growth increment under better or worse conditions. 
% reword the the previous sentences as they're copy pasted from korner2023. Just doing this quick and dirty **for now**
In lights of this, I hypothesize two possible drivers that could explain why a longer growing season might not lead to increased growth: external (environmental) \cite{kolar_response_2016} or internal (via physiological constraints)\cite{zohner_effect_2023} limits to growth. 

\textit{External constraints}
The complex nature of climate change makes predicting the external limits to growth hard to quantify at the individual level, as these drivers affect communities as a whole. However, drought, spring frost and heat waves are commonly mentioned as the main extreme events that could limit tree growth under climate change \cite{tyree_xylem_2002, choat_triggers_2018, li_widespread_2023,trenberth_global_2014,intergovernmental_panel_on_climate_change_detection_2014,chiang_evidence_2021,polgar_leafout_2011,reinmann_compensatory_2023} (See tables). Their respective mechanisms, global trend of occurence, consequences and difference among species are describe in Tables ***. \\

\textit{Internal constraints}
As for the internal constraints to growth, recent hypothesis propose that broadleaf deciduous tree species may be sink-saturated, such that longer growing season with more carbon fixation do not necessarily increase growth (Dow2022). This one pathway is directly linked the the internal controls of plant growth, which are under strong genetic control. Many studied showed, that in addition to heigh and diameter growth varying across species, these are also affected locally, where populations from higher altitude or latitude grow less under the same conditions than individuals from lower altitude or latitude. This is further suported by phenological studies showing that growth end arrives earlier from populations of higher latitude, demonstrating local adaptation to potentially avoid fall frost, before nutrient uptake has finished. These trees rely on photoperiod cues for setting buds (stopping height growth). 
% Maybe go towards carbon fertilization: This goes against a long-standing, but recently debunked paradigm of growth stimulated by carbon fertilization (Jiang2020TheFate (notinzotero). 

Growth seasonality has huge consequences on overall annual growth and so are the environemental conditions during these periods. For instance, warming spring temperatures seem to positively affect growth, but recent evidence suggest a shift in this net positive effect near the summer solstice where a \\
% <><><><><><><><><><><><><><><><><><><><><><><><><><><><><><>
\textbf{How these shifts translate into effects on trees/forests are not clear}
Shifts in spring and autumn phenology support a long-lasting and intuitive assumption that earlier spring and delayed autumn events lead to longer seasons---and thus increased growth \cite{keenan_net_2014} (Stridbeck2022Partly). However, research from the past three years has cast doubt on this hypothesis \cite{dow_warm_2022,green_limits_2022,silvestro_longer_2023}. Recently, Dow \textit{et al}. (2022) showed that despite an earlier growth onset, neither growth rate nor overall annual increment was increased by longer seasons. This could substantially affect carbon-cycle model projections and thus feedbacks to future climate \cite{richardson_climate_2013,swidrak_comparing_2013}. 
Understanding these findings requires answering why trees do not grow more despite longer growing seasons. ** Carbon allocation in wood is pooly understood and the common linear relationship of wood growth as a function of C assimilation is an important limitation of vegetation models because of the poor understanding of empirical and mechanistic basis (Cabon2022). The debate revolving around whether wood growth is controled via photosynthesis (source limitation) or environmental limitations to cambial cell development (sink limitation) seem to bend toward a sink limitation as a result of recent work. Cambial activity appears to be more sensitive than photosynthesis to a range of enviromental conditions including water, temperature and nutrients (Cabon2022). The decoupling between these two processes suggest that internal constraints to growth might be more prevalent than originally thought.

% <><><><><><><><><><><><><><><><><><><><><><><><><><><><><><>
\textbf{Growing season shifts and consequences on forest ecosystems and services}
Spring and fall phenological events are shifting with debatable consequences on tree growth. Since cambial activity is highly sensitive to water, temperature and nutrients suggesting a sink limitation to growth, this could have far-reaching consequences given the hard-to-predict future climate change where any of these variables have the potential to have huge amplitude changes. This expected assymetry of environmental changes under climate change makes understanding the internal and external drivers to growth critical. Especially, the capacity to tease appart different biomes---as for example boreal vs tropical forests are expected to react differently---is critical and empirical data coming from experiments, but also from observations are paramount if we want to be able to predict the changes of forest carbon offset from human GHG emissions.  

% <><><><><><><><><><><><><><><><><><><><><><><><><><><><><><>
% SECTION 1.2 %
% <><><><><><><><><><><><><><><><><><><><><><><><><><><><><><>
\subsection{Nature of the problem, and how to address it}

\textbf{1.2.1. Past phenological trends don't predict future phenological changes}
"The past is not necessarily a guide to the future, but it does partly help explain the present."
Observed phenological trends in the last decades cannot be used directly to extrapolate future phenological changes because: (1) the mechanisms guiding them aren't clear and (2) phenological responses of trees to warming are very likely to not be linear (Fu2013Sensitivity). Indeed, accurate predictions require an in-depth accurate mechanistic understanding of leaf unfolding process and its sensitivity to environmental drivers, especially to temperature and photoperiod (Fu2013Sensitivity). Therefore, the very foundation of the assumption that longer seasons increase growth may change under future climate change. The well observed and understood advance in spring phenology may be offset by warmer winters and delayed autumns may stop being delayed and instead advance in results to earlier growth cessation of increase summer stress. 

\textbf{1.2.2. The assumption that longer seasons lead to increased growth is called into question}
Recent work demonsration an absence of increased growth despite better environmental conditions introduces a paradox that cast doubts onto an apparent very simple and intuitive positive relationship of longer growing seasons on growth. In lights of this, we need to understand better the drivers regulating growth across biomes, but also within species composing these biomes. There is strong differences in phenology across species and so are their responses to warming. This highlights a weekness of current carbon sequestration models where for a lot of them, all species are pooled in together, which introduces a lot of noise in the data and therefore compromising the accuracy of the model projections. To better understand how different species respond to warming, there are different strategies that can be used. Experiments are paramount to robustly tease apart the external vs internal drivers (e.g. warmer springs from severe drought later in the season---a common co-occurring reality in natural environments) \cite{morin_changes_2010,primack_observations_2015}. This is essential to refine forest carbon sequestration projections \cite{green_limits_2022,cabon_cross-biome_2022}. However, experiments are most often performed on juvenile trees, which are critical for their role in forest regeneration projections, but their responses can hardly be translated to mature trees, which hold the overwhelming carbon biomass proportion of forests \cite{augspurger_differences_2003,silvestro_longer_2023,vitasse_ontogenic_2013}. Leaf phenology through ground-based observations can provide valuable insights into the growth onset and end of trees not suitable for experimental trials since cambial and leaf phenology are closely linked to another. It is to say that knowing when leaves elongate and color, can guide when trees start and stop to grow---fundamental metrics to determine the growing season length. Ground observations has the advantage of providing accurate data of phenological events for specific sites and species. Recently, the widespread use of smart-phones has opened a whole new world of possible phenological through citizen scientitists records of data over much larger areas and for a wider range of species (Piao2019, Dickinson2012, Hufkens2019). While there are drawbacks of these observations (e.g. non-standard protocols, highly uneven spatial-temporal distribution of these observations), these methods have a huge potential to diversify the phenology data. % these few lines shouldn't go here. 

\textbf{1.2.3. Impacts on carbon source-sink projections} 
% below might not be relevant here
Atmospheric dryness characterised as the difference between the actual water vapour pressure and the saturation water vapour pressur is often measured through the proxy of vapor deficit pressure (VPD) has been increasing (Yuan2019Increased*;Yuan2025Impacts) and is projected to keep increasing under continuous anthropogenic forcing (Fang2022Globally*).

\textbf{1.2.4. Goals of my thesis} 

% <><><><><><><><><><><><><><><><><><><><><><><><><><><><><><>
% SECTION 1.3. %
% <><><><><><><><><><><><><><><><><><><><><><><><><><><><><><>
\subsection{Complexity of measuring growth and defining growing season length}
\textif{What is a growing season?} \\
To understand how trees respond to growing season conditions, it is important to define the growing season and growth.
First of all, a problem that arises when one tries to quantify how shifting growing seasons affect growth comes from the definition of the growing season itself. Recently, Korner 2023 proposed four definitions adressing this issue: 1. true growing season, based on measurable growth; 2. phenological season, based on visible phenological markers; 3. the productive season, based on primary production and 4. meteorological season, based on environmental conditions.

Here, I will focus on how definition 2., incorporating 4. affects definition 1. as the data collected for this thesis can't adress 3. I will use definition 2. to infer a "window of opportunity" during which meteorological conditions (4.) will be used to calculate growing degree days (GDD). I am using the meteorological season within a contrained window, instead of simply using it irrespective of phenology because of the illusion that an absolute increase in GDD over the last decades---irrespective of the timing of phenophases---also increases growth. Springs are warmer, falls are also sometimes warmer, and summers are warmer which together, increase the number of GDD which may appear to be a reliable proxy for better environmental conditions. However, a fundamental distinction is that there is a "theoritical" and a "real" period at which trees can grow, which is usually defined through the period between budburst and leaf senesence, highlighting the importance of accurate phenology data (REF).

Models using degree-days are increasing even though they have been used for decades in agriculture. These rely on developmental patterns that are based on temperature dependance to estimate a particular ecological process, in my case, tree-ring width. These models describe a particular response variable as a composite of time and temperature as opposed to time alone. This is a partimonious method that requires three variables: daily minimum and maximum temperatures and the base temperature at which the process of interest cannot occur (cambial activation in this case) (McMaster1997Growing-degree;Moore2014Developmental).  Though this simplicity comes with a drawback of over-simplifying potentially complex developmental processes in response to varying environemental conditions within a season (Bonhomme2000Bases).

\textif{What is growth?} \\ 
What is growth? Wood formation (xylogenesis) is the major biological process by which carbon is allocated and long-term stored in woody plants. Radial growth is determined by the production of xylem and phloem cells that begins with cambial activation and cell production by cambial initials, following by cell differentiation through the following events: 1. Cell enlargement 2. Secondary-wall formation and lignification and 3.programmed cell death (Silvestro2025From;Etzold2021Number). The rate and duration of these phases lead to radial growth increment. These lead to irreversible biomass increment that are shown by tree-ring, in which secondary xylem cells account disproportionally to the number of cells produced because they divide more than phloem cells (Rathgeber2016Biological; Plomion2001Wood). \\


\textbf{1.3.1. Traditional diameter measurements miss the resolution of annual growth increment} \\
Foresters have measured tree diameter and height for decades to infer allometries that could give them a good estimate of how much wood they could collect in a forest. The widely used method in forestry is to use diameter tape at breast height at punctual time intervals, but the drawbacks are that they can't provide short-term indicators of growth---especially if taken at multiple years intervals--- such that extreme events affecting growth may be missed. This growth data provides insufficient resolution to infer a relationship between growth and environmental conditions. Well-studied dendrochronology methods have been used for decades and served many purposes, such as calibrating radiocarbon methods and understanding Earth's past climate. \\

% get this from Helama2023. Moreover, the construction of the terrestrial radiocarbon calibration curve is based on tree-ring dated wood (Reimer et al., 2020), similar to the recent annually-resolved refinements of the curve enabling us to examine regional 14C differences
Now, these methods can be used to understand investigate more precise growth patterns and their relationship with different environmental factors

\textbf{1.3.2. Growth increment needs to incorporate wood density in order to evaluate how much structural carbohydrates were stored within a single year.}
While tree-ring width are reliable proxies to how much trees grow in each year, at relatively low cost and time, the inclusion of wood density in the analysis may provide data hidden within the tree rings. Indeed Cufar 2008 (Cufar2008Tree-ring) showed the application of densitometry showed that within and between year density variations in beech provide more information at higher temporal resolution than tree-rings only. 

In addition to densitometry, increasing number of studies are going beyond traditional ring width by performing analysis using wood anatomy data.0 For instance, low cost techniques (e.g. "blue intensity" proxy for latewood desity (Babst2016Blue*, Campbell2007Blue*) and high resolution imaging (von Arx G 2014Roxas; SkippyWSL)) give rise to a whole new world of possibility regarding the microscopic components of wood anatomical features such as micro-anatomical analyses and to apply these into a wide range of applications (Pearl2020New). The character of annual rings, cell structure, timing of growth and markers for trauma can assist in answering a variety ecological and physiological questions previously unanswered with ring width or density alone, such as how growth is affected by growing season length under anthropogenic forcing.

\textbf{1.3.3. Asynchrony between primary and secondary growth (internal growth control?}
I argue that internal growth control in trees may shape their growth responses to growing season length and these can be split into two categories: primary and secondary growth synchronicity and growth determinancy. 

Primary (shoot) and secondary (xylem and phloem formation) growth both contribute to how much carbon is stored in plant tissue, but how much they differ in their responses to environmental drivers is poorly understood. After a dormancy period, trees will start growing, both vertically (primary) and horizontally (secondary), but there is extreme variation among species as to when each growth starts, for how long it lasts and when it stops. For instance, ring-porous speceis intiate primary growth early in the season, sometimes, even before budburst (e.g. oaks) (Stridbeck2022Partly), whereas xylogenesis in diffuse-porour species are usually more synchronized with budburst. These two examples highlight how more complicated it might be to infer general conclusions as to how growing season shifts may also shift growth, where some species may extend their primary growth, but restrict their secondary growth and vice versa.

In addition to differences in primary and secondary growth synchronicity, the role of internal growth control---genetically controlled developmental program where overlooked growth determinancy may mishape our understanding of growth responses to growing season length. In perrenial plants, two dichotomous growth strategies are commonly mentioned: determinate and indeterminate growth, though it may appear as species may exist along a gradient of these. Indeterminate growth is usually associated with short-lived and fast growth species where these life-history strategies may give them a competing advantage as tissue growth can be produced quickly in response to changing environmental conditions, but this comes with higher risk of late spring and early fall frost as well as late droughts. At the opposite side of the spectrum, determinate species are usually long-lived and slow-growth and are mainly constrained by conditions during bud formation, this may increase bud survival at the detriment of opportunistic growth in face of better-than-expected conditions. 

Thus, these characteristics, that greatly vary across species lead make hard the growth pattern predictions of species to shifting growing season length.% shitty sentence that I should either remove or change a lot

\textbf{1.3.4. Getting growth temporal resolution is labor-intensive and expensive (e.g. dendrometer costs)}
To know when trees start and stop to grow within a single growing season (data not extractable through tree-rings), there are two methods, one being labor intensive and the other being expensive. 

1. Trephor allows recurent sampling of mature trees where a 2mm microcore is extracted from the tree a different times during the growing season. By sampling multiple times a single tree, growth temporality can be infered by counting the ring cell increment between each sample. While this non-detrustive tool can be extremely valuable, getting large sample size can hardly be feasible. 

2. Dendrometers allow to monitor stem radius variation, measuring irreversible secondary growth, but aslo stem water fluctuations and thermal expansion, often leading to biased estimates of growth increment temporality (Camarero2009Plastic).

% talk here about intra-annual density fluctuation: https://brill.com/view/journals/iawa/37/2/article-p232_8.xml


% <><><><><><><><><><><><><><><><><><><><><><><><><><><><><><>
% SECTION 1.4. %
% <><><><><><><><><><><><><><><><><><><><><><><><><><><><><><>
\subsection{Objectives} 

% <><><><><><><><><><><><><><><><><><><><><><><><><><><><><><>
% SECTION 1.5. %
% <><><><><><><><><><><><><><><><><><><><><><><><><><><><><><>
\subsection {Research questions} 

\textbf{1.4.1. Fuelinex}

\textbf{1.4.2. CookieSpotters}

% PLACE SOMEWHERE ELSE
%Thus, I propose to use a combination of one experiment to test internal (Chapter 1) limits to growth along with two observational studies (Chapter 2). This will allow me to address the paradox of the absence of increased growth despite apparently improved growing season conditions. 

\section{Methods}

% ============================
% 1. CC  X Phenology
% ============================
% --- --- --- --- --- --- --- --- --- --- --- --- --- --- --- ---
% Tables spring frost, drought and heat waves $
% --- --- --- --- --- --- --- --- --- --- --- --- --- --- --- ---
		\textbf{3.1. Spring frosts} \\ %emwJuly1 -- Wow, Christophe! This is super impressive and the notes and refs generally align with my view of the literature (though I think you could find other refs beyond Chiang that say it's other parts of the world, but overall this is very good). I would consider keeping this for your introduction chapter for sure, just be careful to not try to write all of this out for the reader  early in the intro as it will be too specific too fast, but this is great!
\resizebox{\textwidth}{!}{
\begin{tabular}{|>{\raggedright\arraybackslash}p{4cm}|p{12cm}|}
\hline
\textbf{} & \multicolumn{1}{c|}{\textbf{}} \\
\hline

\textbf{Mechanisms} & Early warm spells $\rightarrow$ early leaf out $\rightarrow$ hard frost ($<$-2Celsius) $\rightarrow$ tissue death = loss of photosynthetic capacity \cite{polgar_leafout_2011}; Response: second cohort of leaves are more efficient and mitigate carbon sequestration loss \cite{reinmann_compensatory_2023} \\
\hline
\textbf{Global trend of occurrence} & Most vulnerable regions are the ones with no past risk of occurrence (); $\uparrow$ in Europe and East Asia, but $\downarrow$ North America; Global trend is controversial \cite{reinmann_compensatory_2023} \\
\hline
\textbf{Consequences (Individual and Ecosystem level consequences)} & Loss of vegetative tissue = $\downarrow$ photosynthesis = $\downarrow$ and remobilization of NSC to repair damaged tissues = $\downarrow$ secondary growth (Meyer24); Loss of reproductive tissue (higher flower mortality) (REF); Costs for orchards and stuff \cite{reinmann_compensatory_2023} \\
\hline
\textbf{Differences across species/provenance} & \\
\hline
\end{tabular}
} 

\clearpage
	\textbf{3.2. Drought} \\
\resizebox{\textwidth}{!}{
\begin{tabular}{|>{\raggedright\arraybackslash}p{4cm}|p{12cm}|}
\hline
\textbf{} & \multicolumn{1}{c|}{\textbf{}} \\
\hline
\textbf{Mechanisms} & — Hot temperature + low precipitation (aka global-change-type drought \cite{tyree_xylem_2002})= $\uparrow$ evapotranspiration$\rightarrow$ less water in soil $\rightarrow$ cavitation $\rightarrow$ embolism $\rightarrow$ hydraulic failure \cite{tyree_xylem_2002} = tissue death \citep{choat_triggers_2018}; 

— Earlier spring phenology = longer GS $\rightarrow$ increases vegetative growth $\rightarrow$ increases evapotranspiration $\rightarrow$ increases drawdown of soil moisture = progressive water stress \cite{li_widespread_2023}

— Long-term vs short-term stomatal responses and consequences on tissue death \citep{choat_triggers_2018}; 

— Recovery and its determinants \citep{choat_triggers_2018,li_widespread_2023}\\
\hline
\textbf{Global trend of occurrence} & — $\uparrow$ precipitation anomalies since 1990 \cite{trenberth_global_2014};  

— Models often exclude PDO/ENSO which limit the capacity to attribute increasing droughts to CC  \cite{trenberth_global_2014}; 

— Weak evidence of detection and attribution of changes in meteorological drought since the mid-20th century \cite{intergovernmental_panel_on_climate_change_detection_2014}; 

— Using a spacial, model-based perspective, anthropogenic forcing increased the frequency, duration and intensity of SPI-based droughts for Americas, Mediterreanean, W/S Africa and E Asia \cite{chiang_evidence_2021} \\
\hline
\textbf{Consequences (Individual and Ecosystem level consequences)} & — Recurring droughts may limit trees' ability to recover from other types of stress.

—Tree mortality (e.g. Texas and California extreme droughts are estimated to have killed 300 and 102 million trees \cite{li_widespread_2023})\\ %emwJuly1 e.g. is always lowercase
\hline
\textbf{Differences across species/provenance} &  \\
\hline
\end{tabular}
} \\
\par

\textbf{3.3. Heat waves : \textit{needs to be filled}}\\
\resizebox{\textwidth}{!}{
\begin{tabular}{|>{\raggedright\arraybackslash}p{4cm}|p{12cm}|}
\hline
\textbf{} & \multicolumn{1}{c|}{\textbf{}} \\
\hline
\textbf{Definition:} &  Heat wave is a period of excessively hot weather (5 or more consecutive days of prolonged heat in which the daily maximum temperature is higher than the average maximum temperature by 5 °C ), which may be accompanied by high humidity (Marx2021Heat) \\
\hline
\textbf{Mechanisms} &  $\uparrow$ atmospheric CO2 = $\uparrow$ temperature $\rightarrow$ $\uparrow$ heat waves... More specifically: A mechanism for the increase occurence of heat waves is a weakeking of the polar jet stream (important weather factor for middle latitude regions of North America, Europe and Asia) caused by global warming which increases the occurence of stationnary weather, resulting in heavy rain falls or heat waves(Marx2021Heat).  Extreme heat $\rightarrow$ growth either through (1) Directly via cell processes disruption or (2) indirectly via effects of rising leaf-to-air vaport deficit (VPD) (Gagne2020Limited).

Increased temperature leads to reduced photosynthesis which can be attributed to: 
1. Damage to photosynthetic machinery
2. Inactivation of RUBISCO
3. Reduction to RuBP regeneration
4. Membrane stability (Markus2025Heat)
5. Increased mitochondrial respiration and photorespiration
\\
\hline
\textbf{Global trend of occurrence} & Heat waves have increased (Meehl2004More;Gagne2020Limited; Teskey2015Responses) and are expected to increase under future climate change (Yao2013Comparison;Teskey2015Responses; Dosio2018Extreme;IPCC2014). Summertime extreme temperatures associated with prolonged heat waves, lasting for several weeks, now impact approximately 10\% of land surfaces, up from only 1\% in the 1960s. (Teskey2015Responses).
The more intense and more frequently occurring heat waves cannot be explained solely by natural climate variations and without human-made climate change (Marx2021Heat).
 \\

\hline
\textbf{Consequences (Individual and Ecosystem level consequences)} &  - Reduced photosynthesis (Gagne2020Limited) - Increased mortality - Photosynthetic tissue loss \\ 
\hline
\textbf{Differences across species/provenance} & Some species have thermal photosynthetic/respiratory acclimatation while others don't. Growth and survival will change depending on species to thermally acclimate to both photosynthesis and respiration 
- This is explained by growth strategies of gymnosperms vs angiosperms (which are usually better)
 \\
\hline
\end{tabular}
}


\subsection{Climate change impacts on tree phenology} 
Climate change impacts on biological systems and how phenological trends are already shifting with warming temperatures. 

%%%
\begin{table}[p]
\centering
\caption{Fuelinex species grouped by tree type, life history, and wood anatomy.}
\begin{tabular}{|>{\raggedright\arraybackslash}p{7cm}|p{5cm}|p{3cm}|p{1cm}|}
\hline
\multicolumn{4}{|c|}{\textbf{Deciduous Trees}} \\
\hline
\textbf{Common Name (Latin)} & \textbf{Life History Strategy} & \textbf{Wood Anatomy} & \textbf{n (approx)} \\
\hline
Bur oak (\textit{Quercus macrocarpa}) & Slow-growth, long life & Ring-porous & 87\\
Bitter cherry (\textit{Prunus virginiana}) & Fast-growth, short life & Diffuse-porous & 78\\
Box elder (\textit{Acer negundo}) & Fast-growth, short life  & Diffuse-porous & 90\\
Balsam poplar (\textit{Populus balsamifera}) & Fast-growth, short life  & Diffuse-porous &84 \\
Paper birch (\textit{Betula papyrifera}) & Fast-growth, short life  & Diffuse-porous &90\\
\hline
\multicolumn{4}{|c|}{\textbf{Evergreen Trees}} \\
\hline
White pine (\textit{Pinus strobus}) & Slow-growth, long life & & 89\\
Giant Sequoia (\textit{Sequoiadendron giganteum}) & Slow-growth, long life & & 54\\
\hline
\end{tabular}
\end{table}
%%%
% ============================
% Wildchrokie
% ============================
\subsection{Wildchrokie}
\begin {enumerate}
	\item Common garden from 2015 to 2023
	\item Four species within the Betulacea family (Table 2)
%%%
\begin{table}[p]
\centering
\caption{Wilchrokie species grouped by tree type, life history, and wood anatomy.}
\begin{tabular}{|>{\raggedright\arraybackslash}p{7cm}|p{5cm}|p{3cm}|p{1cm}|}
\hline
\multicolumn{4}{|c|}{\textbf{Deciduous Trees}} \\
\hline
\textbf{Common Name (Latin)} & \textbf{Life History Strategy} & \textbf{Wood Anatomy} & \textbf{n} \\
\hline
Paper birch (\textit{Betula papyrifera}) & Fast-growth, short life  & Diffuse-porous & 8\\
Yellow birch (\textit{Betula alleghaniensis}) & Moderate-growth, moderate life & Diffuse-porous & 21\\
Grey birch (\textit{Betula populifolia}) & Fast-growth, short life & Diffuse-porous & 29\\
Grey alder (\textit{Alnus incana}) & Fast-growth, short life & Diffuse-porous & 31\\
\hline
\end{tabular}
\end{table}
%%%
	\item Data: phenology, height, tree rings
	\item Analysis: Hierarchical model to understand how tree ring width relates to GDD
\end {enumerate}

% ============================
% Treespotters
% ============================
\subsection{Treespotters}
\begin {enumerate}
	\item Citizen science project from 2015 to today (Table 3)
\begin{table}[h]
\centering
\caption{Treespotters species grouped by tree type, life history, and wood anatomy.}
\begin{tabular}{|>{\raggedright\arraybackslash}p{7cm}|p{5cm}|p{3cm}|p{1cm}|}
\hline
\multicolumn{4}{|c|}{\textbf{Deciduous Trees}} \\
\hline
\textbf{Common Name (Latin)} & \textbf{Life History Strategy} & \textbf{Wood Anatomy} & \textbf{n} \\
\hline
American basswood (\textit{Tilia americana}) & Fast-growth, moderate life & Diffuse-porous & 5\\
Eastern cottonwood (\textit{Populus deltoides}) & Fast-growth, short life & Diffuse-porous & 4\\
Northern red oak (\textit{Quercus rubra}) & Moderate-growth, long life & Ring-porous & 4\\
White oak (\textit{Quercus alba}) & Slow-growth, long life & Ring-porous & 5\\
Pignut hickory (\textit{Carya glabra}) & Slow-growth, long life & Ring-porous & 4\\
Shagbark hickory (\textit{Carya ovata}) & Slow-growth, long life & Ring-porous & 4\\
River birch (\textit{Betula nigra}) & Fast-growth, short life & Diffuse-porous & 5\\
Yellow birch (\textit{Betula alleghaniensis}) & Moderate-growth, moderate life & Diffuse-porous & 4\\
Sugar maple (\textit{Acer saccharum}) & Slow-growth, long life & Diffuse-porous & 5\\
Red maple (\textit{Acer rubrum}) & Slow-growth, long life & Diffuse-porous & 4\\
Yellow buckeye (\textit{Aesculus flava}) & Moderate-growth, moderate life & Diffuse-porous & 5\\
\hline
\end{tabular}
\end{table}

	\item Tree coring
	\item Data: phenology, tree rings
	\item Analysis: Hierarchical model to understand how tree ring width relates to GDD	
\end {enumerate}


\section {References}
\bibliography{Exported_Items.bib}
\bibliographystyle{nature} % set citation style 


% old text
%\section*{Chapter 2.1. Wildchrokie: Phenological observations coupling with tree-ring width measurements for a 6-year common garden experiment} While the assumption that a longer growing season leads to increased growth is an intuitive and common one, recent evidence shows that this may not be the case. 
%How phenological season length relates to growth. This is a major question in fundamental biology, but also critical to forecasts of climate change itself, since most carbon models assume that plants experiencing longer seasons will sequester more carbon, but recent studies have called this assumption into question.


% ============================
% 2. Tree-ring stuff 
% ============================
%\subsection{Tree rings measurements as a proxy for growth}
%Using tree ring data to investigate the relationship between phenology and growth

%\begin{enumerate}
% 1)
%	\item Triggers and mechanisms behind growth onset, duration and rate.
%Growth onset and duration vary because of inter-annual differences in weather, with cambium reactivation in spring being highly dependent on temperature. 
% 2)
%	\item How radial growth is influenced by extreme weather events and their timing. 
%Growh rate increased by temperature, depends on \textbf{when} it is warmer. 
% Include in main text: Long seasons at low temperature will produce fewer cell rows than at warmer temperature.
% 3) 
%	\item Which is more important? How fast does a tree grow, or how long does it grow for?
%how growth rate may have a more direct influence on tree growth than the growing season length. 
% 4)
%	\item Methods to measure tree growth and why using tree ring images may better capture tree growth response than traditional diameter and height measurements.
% Include in main text: Diameter and height measurements are widely used to assess yearly biomass increment. However, these measurements are punctual and are often the cumulative result of many climatic events and constraints that occur during a tree's lifespan
%\end{enumerate}



\end{document}
