

\documentclass{article}
\usepackage[utf8]{inputenc}
\usepackage{authblk}
\usepackage{setspace}
\usepackage[margin=1.25in]{geometry}
\usepackage{graphicx}
\graphicspath{ {./figures/} }
\usepackage{subcaption}
\usepackage{amsmath}
\usepackage{lineno}
\linenumbers



%%%%%% Bibliography %%%%%%
% Replace "sample" in the \addbibresource line below with the name of your .bib file.


%%%%%%%%%%%%%%%%%
\title{Thesis outline}
\date{\today}
\author{Christophe Rouleau-Desrochers}
\begin{document}
%%%%%%%%%%%%%%%%%%%%%%%%%%%%%%
%%%%%%%%%%%%%%%%%%%%%%%%%%%%%%
%%%%%%%%%%%%%%%%%%%%%%%%%%%%%%

\maketitle
%%%%%%%%%%%%%%%%%%%%%%%%%%%%%%
\section*{Introduction}
%%%%%%%%%%%%%%%%%%%%%%%%%%%%%%


%<><><><><><><><><><><><><><><><><><><><>
% Starting the whole thing
%<><><><><><><><><><><><><><><><><><><><>
\begin{enumerate}

% 1. CC  X Biological systems
\item CC impacts on biological systems and how phenological trends are already shifting with warming temperatures. 
\begin{enumerate}
% a)
\item Warmer temperature led to earlier spring events for amphibians, birds, butterflies and wild plants (Walther, 2002)
% b)
\item Autumn phenological events are delayed, but the trend is not as clear as spring's. We have a good mechanistic understanding of the drivers that lead plants to leaf out early, but we don't for Autumn. \textit{Maybe talk about why the trend isn't clear (e.g. monitoring leaf fall and colouring is hard. Can be highly influenced by a single episode of wind, precipitation or frost (Gunderson, 2012)}  
% c)
\item Counterinteraction of winter warming that delays spring phenology because of non-met chilling requirements \textit{don't want to get lost in the weeds here though}. Talk about deacclimatation forms during spring
% d)
\item Overall, earlier spring and delayed autumn lead to a longer phenological growing season (Korner, 2023 for pheno GS definition).
% e)
\item Potential impacts of spring frost. Explain how reliance on photoperiod can be a better strategy to avoid spring frosts. Species that are less photoperiod dependant may be more vulnerable to spring frost.
% f)
\item Increased drought events in the summer and how earlier spring might increase water deficit later on in the GS. (Vitesse 2021)
\item Pros and cons of early SOS: \\ 
\textbf{Pros}: Potential competitive ability of carbon uptake at the individual and stand level (increased productivity) (Estiarte, 2015); More days to reach fruit maturity. \\ 
\textbf{Cons}: Trophic mismatch (though limited support) (Loughnan 2024); Increased summer drought induced stress; Increased pest and disease pressure; Soil nutrient depletion (to read: Reich 2006)
\item Pros and cons of delayed EOS: \\
\textbf{Pros}: Photosynthesis can occur for longer, increasing carbon sequestration (Keenan, 2014) ; May increase nutrient resorption efficiency (Richardson 2010); May delay frost exposure (Gunderson, 2012)\\ 
\textbf{Cons}: Delayed leaf senescence could kill leaf (cold spell) before nutrient resorption (Estiarte, 2015 ; Augspurger, 2013) ; Phenological mismatches; Disruption of dormancy cycles (Korner, 2010); Extension of pest life cycles (Ayres, 2000).
\end{enumerate}

% 2. Tree-ring stuff 
\item Tree rings measurements allows for a finer scale understanding of the cambium and leaf phenology relationship.
\begin{enumerate}
% a)
\item  Diameter and height measurements are widely used to assess yearly biomass increment. However, these measurements are punctual and are often the cumulative result of many climatic events and constraints that occurent during a tree's lifespan. Thus the use of high resolution, tree ring images allows for a more detailed assessment of the climate influence on tree growth.
% b)
\item Cambial phenology. Growth onset and duration vary because of inter-annual differences in weather, with cambium reactivation in spring being highly dependent on temperature. 
% c)
\item Radial growth increased by temperature depends on \textbf{when} it is warmer. Long seasons at low temperature will produce fewer cell rows than at warmer temperature. 
\item The growth rate has a more direct influence on tree growth than the growing season length. 
\item test
\end{enumerate}

%<><><><><><><><><><><><><><><><><><><><>
% Closing the whole thing
%<><><><><><><><><><><><><><><><><><><><>
\end{enumerate}

%%%%%%%%%%%%%%%%%%%%%%%%%%%%%%
\section*{Chapter 1. Fuelinex: fueling next year's growth with longer growing season*** }
\subsection* {Question}
How do extended growing seasons affect tree growth, both in the current and subsequent years? 
\subsection* {Objectives}
\subsection* {Methods}
%%%%%%%%%%%%%%%%%%%%%%%%%%%%%%

%%%%%%%%%%%%%%%%%%%%%%%%%%%%%%
\section*{Chapter 2.1. Wildchrokie: Phenological observations coupling with tree-ring width measurements for a 6-year common garden experiment} \\
While the assumption that a longer growing season leads to increased growth is an intuitive and common one, recent evidence shows that this may not be the case. 

\subsection* {Question}
How phenological season length relates to growth. This is a major question in fundamental biology, but also critical to forecasts of climate change itself, since most carbon models assume that plants experiencing longer seasons will sequester more carbon, but recent studies have called this assumption into question.
\subsection* {Objectives}
Using unique phenological data from the Arnold Arboretum, we aim to couple ground phenological observations (leafout, flowering, fruiting, leaf coloring) with tree ring data.
\subsection* {Methods}\

\section *{Chapter 2.2. CoringTreeSpotters: Ring width and citizen science phenological observations relationships)

%%%%%%%%%%%%%%%%%%%%%%%%%%%%%%

%%%%%%%%%%%%%%%%%%%%%%%%%%%%%%
\section*{Conclusion}
%%%%%%%%%%%%%%%%%%%%%%%%%%%%%%


\end{document}
