

\documentclass{article}
\usepackage[utf8]{inputenc}
\usepackage{authblk}
\usepackage{setspace}
\usepackage[margin=1.25in]{geometry}
\usepackage{graphicx}
\graphicspath{ {./figures/} }
\usepackage{subcaption}
\usepackage{amsmath}
\usepackage{lineno}
\linenumbers



%%%%%% Bibliography %%%%%%
% Replace "sample" in the \addbibresource line below with the name of your .bib file.


%%%%%%%%%%%%%%%%%
\title{Thesis outline}
\date{\today}
\author{Christophe Rouleau-Desrochers}
\begin{document}
%%%%%%%%%%%%%%%%%%%%%%%%%%%%%%

\maketitle
\section *{Ideas yet to be introduced in the outline:}
The idea of how photoperiod and warming requirements interact with each other. See note:Mechanisms on how photoperiod perception interacts with warming requirements from Zohner's paper in 2016.\\
Why warmer temperature may increase growht


%<><><><><><><><><><><><><><><><><><><><>
% Introduction
%<><><><><><><><><><><><><><><><><><><><>
\section*{Introduction}

% ============================
% 1. CC  X Phenology
% ============================
\subsection *{Impacts of climate change on tree phenology} : CC impacts on biological systems and how phenological trends are already shifting with warming temperatures. 
\begin{enumerate}
% a)
\item Warmer temperature led to earlier spring events for amphibians, birds, butterflies and wild plants (Walther, 2002)
% b)
\item Autumn phenological events are delayed, but the trend is not as clear as spring's. We have a good mechanistic understanding of the drivers that lead plants to leaf out early, but we don't for Autumn. \textit{Maybe talk about why the trend isn't clear (e.g. monitoring leaf fall and colouring is hard. Can be highly influenced by a single episode of wind, precipitation or frost (Gunderson, 2012)}  
% c)
\item The acceleration of earlier spring events is slowing down because of the counterinteraction of warmer winters that delays spring phenology because of non-met chilling requirements, which increase forcing requirements --> later budburst \textit{Potentially talk about deacclimatation forms during spring}. 
% d)
\item Overall, earlier spring and delayed autumn lead to a longer phenological growing season (Korner, 2023 for pheno GS definition).
% e)
\item Potential impacts of spring frost. \textit{Explain how the strategy to rely on photoperiodic cues can decrease spring frost risks}
% f)
\item Drought events are increasing in frequency and severity. This influences tree growth, and earlier spring might increase water deficit later on in the GS. (Vitasse 2021)
% g)
\item Pros and cons of early SOS: \\ 
\textbf{Pros}: 
	\begin {enumerate}
		\item Potential competitive ability of carbon uptake at the individual and stand level (increased productivity) (Estiarte, 2015); 
		\item More days to reach fruit maturity. 
	\end {enumerate}
\textbf{Cons}: 
	\begin {enumerate}
		\item Trophic mismatch (though limited support) (Loughnan 2024); 
		\item Increased summer drought induced stress;
		\item Increased pest and disease pressure;
		\item Soil nutrient depletion (to read: Reich 2006)
	\end {enumerate}
\item Pros and cons of delayed EOS: \\
\textbf{Pros}: 
	\begin {enumerate}
		\item Photosynthesis can occur for longer, increasing carbon sequestration (Keenan, 2014) ; 
		\item May increase nutrient resorption efficiency (Richardson 2010); 
		\item May delay frost exposure (Gunderson, 2012)
	\end {enumerate}
\textbf{Cons}: 
	\begin {enumerate}
		\item Delayed leaf senescence could kill leaf (cold spell) before nutrient resorption (Estiarte, 2015 ; Augspurger, 2013); 
		\item Phenological mismatches; 
		\item Disruption of dormancy cycles (Korner, 2010); 
		\item Extension of pest life cycles (Ayres, 2000).
	\end {enumerate}
\end{enumerate}

% ============================
% 2. Tree-ring stuff 
% ============================
\subsection *{Tree rings measurements as a proxy for growth}:  allows for a finer scale understanding of the cambium and leaf phenology relationship
\begin{enumerate}
% a)
	\item  Diameter and height measurements are widely used to assess yearly biomass increment. However, these measurements are punctual and are often the cumulative result of many climatic events and constraints that occurent during a tree's lifespan. Thus the use of high resolution, tree ring images allows for a more detailed assessment of the climate influence on tree growth.
% b)
	\item Cambial phenology. Growth onset and duration vary because of inter-annual differences in weather, with cambium reactivation in spring being highly dependent on temperature. 
% c)
	\item Radial growth increased by temperature, depends on \textbf{when} it is warmer. Long seasons at low temperature will produce fewer cell rows than at warmer temperature. 
	\item The growth rate has a more direct influence on tree growth than the growing season length. 
\end{enumerate}

% ============================
% this isn't a section, but I wanted it to be separate from the 2 first ones.
% ============================
\subsection*{Nature of the problem} 
\begin{enumerate}
	\item Past phenological trends don't predict future phenological changes. Highlights the importance of understanding the drivers that control phenology and growth,
	\item The Assumption that longer seasons lead to increased growth is called into question
	\item Impacts carbon source-sink projections
\end{enumerate}

% ============================
% Research questions
% ============================
\subsection *{Research questions}
\begin {enumerate}
	\item \textbf{Fuelinex}: How do extended growing seasons affect tree growth across different species, both immediately (in the same year as the extended season) and in subsequent years?
	\item \textbf {CookieSpotters}: How phenological traits regulate tree growth in urban ecosystems?
\end {enumerate}

% ============================
% Hypothesis
% ============================
\subsection *{Hypothesis}
\begin {enumerate}
	\item \textbf{Fuelinex}:  Growing season extension modifies a tree’s capacity to fill carbon and nitrogen storage pools and this could lead ot increased growth in the following season.
	\item \textbf{Fuelinex}: Species capable of accumulating nutrients after growth cessation while going through leaf senescence might exhibit growth increment in the following growing season
	\item \textbf{CookieSpotters}: Juvenile trees would grow more proportionally in response to longer seasons than mature trees.
\end {enumerate}

% ============================
% Objectives and outreach
% ============================
\subsection *{Objectives and outreach}
\begin {enumerate}
	\item \textbf{Fuelinex}: Assess tree species’ potential to prolong or stretch their activity schedule.
	\item \textbf{Fuelinex}:  Determine whether trees can absorb nutrients beyond their theoretical growing season.
	\item \textbf{Fuelinex}:  Examine if increased carbon pools translate into greater growth increment in the following growing season. 
	\item \textbf{CookieSpotters}: Investigate how the timing of phenological events affects growth across years for juvenile and mature trees
\end {enumerate}

%<><><><><><><><><><><><><><><><><><><><>
% Methods
%<><><><><><><><><><><><><><><><><><><><>
\section{Methods}

% ============================
% Fuelinex
% ============================
\subsection {Fuelinex}
\begin {enumerate}
	\item Full factorial design (Figure 1. Experimental design)
	\item 2-year experiment 
	\item Nutrient addition
	\item Data: phenology, shoot elongation, diameter, height, biomass, tree rings
	\item Analysis: TBD
\end {enumerate}

% ============================
% Wildchrokie
% ============================
\subsection {Wildchrokie}
\begin {enumerate}
	\item Common garden from 2015 to 2023 (Figure 2. Species studied, and number of trees/species)
	\item Data: phenology, height, tree rings
	\item Analysis: Hierarchical model to understand how tree ring width relates to GDD
\end {enumerate}

% ============================
% Treespotters
% ============================
\subsection {Treespotters}
\begin {enumerate}
	\item Citizen science project from 2015 to today (Figure 3. Species studied, and number of trees/species)
	\item Tree coring
	\item Data: phenology, tree rings
	\item Analysis: Hierarchical model to understand how tree ring width relates to GDD	
\end {enumerate}

\section*{Timeline}

% old text
%\section*{Chapter 2.1. Wildchrokie: Phenological observations coupling with tree-ring width measurements for a 6-year common garden experiment} While the assumption that a longer growing season leads to increased growth is an intuitive and common one, recent evidence shows that this may not be the case. 
%How phenological season length relates to growth. This is a major question in fundamental biology, but also critical to forecasts of climate change itself, since most carbon models assume that plants experiencing longer seasons will sequester more carbon, but recent studies have called this assumption into question.

\end{document}
