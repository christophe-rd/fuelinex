

\documentclass{article}
\usepackage[utf8]{inputenc}
\usepackage{authblk}
\usepackage{setspace}
\usepackage[margin=1.25in]{geometry}
\usepackage{graphicx}
\graphicspath{ {./figures/} }
\usepackage{subcaption}
\usepackage{amsmath}
\usepackage{lineno}
\usepackage{soul}
\usepackage{xcolor}
\sethlcolor{yellow}
\linenumbers

%%%%%% Bibliography %%%%%%
% Replace "sample" in the \addbibresource line below with the name of your .bib file.

%%%%%%%%%%%%%%%%%
\title{Thesis outline}
\date{\today}
\author{Christophe Rouleau-Desrochers}
\begin{document}
%%%%%%%%%%%%%%%%%%%%%%%%%%%%%%

\maketitle


%<><><><><><><><><><><><><><><><><><><><>
% Introduction
%<><><><><><><><><><><><><><><><><><><><>
\section{Introduction}

% ============================
% 1. CC  X Phenology
% ============================
\subsection{Climate change impacts on tree phenology} 
Climate change impacts on biological systems and how phenological trends are already shifting with warming temperatures. 
\begin{enumerate}
% 1)
	\item Warmer temperature led to earlier spring events for amphibians, birds, butterflies and wild plants (Walther, 2002)
% 2)
	\item Autumn phenological events are delayed, but the trend is not as clear as spring's. Description of mechanistic drivers of autumn phenology vs spring
% Include in main text: We have a good mechanistic understanding of the drivers that lead plants to leaf out early, but we don't for Autumn. \textit{Maybe talk about why the trend isn't clear (e.g. monitoring leaf fall and colouring is hard. Can be highly influenced by a single episode of wind, precipitation or frost (Gunderson, 2012)}  
% 3)
	\item Long-term trends suggest that the pace of spring events advancement is slowing down.  Counterinteraction of warmer winters that delays spring phenology because of non-met chilling requirements, which increase forcing requirements --> later budburst 
%Include in main text: Deacclimatation forms during spring
% 4) 
	\item Photoperiod perception interaction with warming requirements (Zohner 2016)
% 5)
	\item Potential impacts of spring frost on growth. \textit{Explain how the strategy to rely on photoperiodic cues can decrease spring frost risks}
% 6)
	\item Overall, earlier spring and delayed autumn lead to a longer phenological growing season (Korner, 2023 for pheno GS definition)
% 7)
	\item Drought events are increasing in frequency and severity, which influences tree growth
% Include in main text: earlier spring might increase water deficit later on in the GS. (Vitasse 2021)
% 8)
	\item Pros and cons of early Start of Season: \\  % this is just an idea for the outline that won't be in the main text. Mostly to organize potential positive and negeative consequences 
		\textbf{Pros}: 
			\begin {itemize}
				\item Potential competitive ability of carbon uptake at the individual and stand level (increased productivity) (Estiarte, 2015); 
				\item More days to reach fruit maturity. 
			\end {itemize}
		\textbf{Cons}: 
			\begin {itemize}
				\item Trophic mismatch (though limited support) (Loughnan 2024)
				\item Incre	ased summer drought-induced stress
				\item Increased pest and disease pressure
				\item Soil nutrient depletion (e.g. Reich 2006)
			\end {itemize}
	\item Pros and cons of delayed End of Season: \\
		\textbf{Pros}: 
			\begin {itemize}
				\item Photosynthesis can occur for longer, increasing carbon sequestration (Keenan, 2014) 
				\item May increase nutrient resorption efficiency (Richardson 2010)
				\item May delay frost exposure (Gunderson, 2012)
			\end {itemize}
		\textbf{Cons}: 
			\begin {itemize}
				\item Delayed leaf senescence could kill leaves (cold spell) before nutrient resorption (Estiarte, 2015 ; Augspurger, 2013)
				\item Phenological mismatches
				\item Disruption of dormancy cycles --chilling requirements not met(Korner, 2010)
				\item Extension of pest life cycles (Ayres, 2000)
			\end {itemize}
\end{enumerate}

% ============================
% 2. Tree-ring stuff 
% ============================
\subsection{Tree rings measurements as a proxy for growth}
Analyze tree rings to investigate the relationship between phenology and growth

\begin{enumerate}
% 1)
	\item  Tree ring images allow for a more detailed assessment of the climate influence on tree growth than diameter and height measurements 
% Include in main text: Diameter and height measurements are widely used to assess yearly biomass increment. However, these measurements are punctual and are often the cumulative result of many climatic events and constraints that occur during a tree's lifespan
% 2)
	\item Cambial phenology. Growth onset and duration vary because of inter-annual differences in weather, with cambium reactivation in spring being highly dependent on temperature. 
% 3)
	\item Radial growth increased by temperature, depends on \textbf{when} it is warmer. 
% Include in main text: Long seasons at low temperature will produce fewer cell rows than at warmer temperature.
	\item Growth rate has a more direct influence on tree growth than the growing season length. 
\end{enumerate}

% ============================
% this isn't a section, but I wanted it to be separate from the 2 first ones.
% ============================
\subsection{Nature of the problem} 
\begin{enumerate}
	\item Past phenological trends don't predict future phenological changes. Highlights the importance of understanding the drivers that control phenology and growth,
	\item The assumption that longer seasons lead to increased growth is called into question
	\item Impacts on carbon source-sink projections
\end{enumerate}

% ============================
% Research questions
% ============================
\subsection {Research questions}
\begin {enumerate}
	\item \textbf{Fuelinex}: How do extended growing seasons affect tree growth across different species, both immediately (in the same year as the extended season) and in subsequent years?
	\item \textbf {CookieSpotters}: How phenological traits regulate tree growth in urban ecosystems?
\end {enumerate}

% ============================
% Hypothesis
% ============================
\subsection{Hypothesis}
\begin {enumerate}
	\item \textbf{Fuelinex}: Growing season extension modifies a tree’s capacity to fill carbon and nitrogen storage pools and this could lead to increased growth in the following season.
	\item \textbf{Fuelinex}: Species capable of accumulating nutrients after growth cessation while going through leaf senescence might exhibit growth increment in the following growing season
	\item \textbf{CookieSpotters}: The magnitude of the growth response to longer seasons will differ between juvenile and mature trees.
\end {enumerate}

% ============================
% Objectives and outreach
% ============================
\subsection{Objectives and outreach}
\begin {enumerate}
	\item \textbf{Fuelinex}: Assess tree species’ potential to prolong or stretch their activity schedule.
	\item \textbf{Fuelinex}:  Determine whether trees can absorb nutrients beyond their theoretical growing season.
	\item \textbf{Fuelinex}:  Examine if increased carbon pools translate into greater growth increment in the following growing season. 
	\item \textbf{CookieSpotters}: Investigate how the timing of phenological events affects growth across years for juvenile and mature trees
\end {enumerate}

%<><><><><><><><><><><><><><><><><><><><>
% Methods
%<><><><><><><><><><><><><><><><><><><><>
\section{Methods}

% ============================
% Fuelinex
% ============================
\subsection{Fuelinex}
\begin {enumerate}
	\item Full factorial design (\hl{Figure 1. Experimental design})
	\item 2-year experiment 
	\item Nutrient addition
	\item Data: phenology, shoot elongation, diameter, height, biomass, tree rings
	\item Analysis: TBD
\end {enumerate}

% ============================
% Wildchrokie
% ============================
\subsection{Wildchrokie}
\begin {enumerate}
	\item Common garden from 2015 to 2023 (\hl{Table I. Species studied, and number of trees/species})
	\item Data: phenology, height, tree rings
	\item Analysis: Hierarchical model to understand how tree ring width relates to GDD
\end {enumerate}

% ============================
% Treespotters
% ============================
\subsection{Treespotters}
\begin {enumerate}
	\item Citizen science project from 2015 to today (\hl{Table II. Species studied, and number of trees/species})
	\item Tree coring
	\item Data: phenology, tree rings
	\item Analysis: Hierarchical model to understand how tree ring width relates to GDD	
\end {enumerate}

\section{Timeline}
\hl{Figure 2. Master's thesis timeline.}

% old text
%\section*{Chapter 2.1. Wildchrokie: Phenological observations coupling with tree-ring width measurements for a 6-year common garden experiment} While the assumption that a longer growing season leads to increased growth is an intuitive and common one, recent evidence shows that this may not be the case. 
%How phenological season length relates to growth. This is a major question in fundamental biology, but also critical to forecasts of climate change itself, since most carbon models assume that plants experiencing longer seasons will sequester more carbon, but recent studies have called this assumption into question.

\end{document}
