

\documentclass{article}
\usepackage[utf8]{inputenc}
\usepackage{authblk}
\usepackage{setspace}
\usepackage[margin=1.25in]{geometry}
\usepackage{graphicx}
\graphicspath{ {./figures/} }
\usepackage{subcaption}
\usepackage{amsmath}
\usepackage{lineno}
\linenumbers


%%%%%% Bibliography %%%%%%
% Replace "sample" in the \addbibresource line below with the name of your .bib file.
\usepackage[style=nejm, 
citestyle=numeric-comp,
sorting=none]{biblatex}
\addbibresource{sample.bib}

%%%%%% Title %%%%%%
% Full titles can be a maximum of 200 characters, including spaces. 
% Title Format: Use title case, capitalizing the first letter of each word, except for certain small words, such as articles and short prepositions
\title{fuelinex draft manuscript}

%%%%%% Authors %%%%%%
% Authors should be listed in order of contribution to the paper, by full first name, then middle initial (if any), followed by last name and separated by commas.
% Please do not use initials for first names. If you use your middle name as a full name, use an initial for the first name and spell out your full middle name.
% Use a superscript asterisk (*) to identify the corresponding author and be sure to include that person’s e-mail address. Use symbols (in this order: †, ‡, §, ||, ¶, #, ††, ‡‡, etc.) for author notes, such as present addresses, “These authors contributed equally to this work” notations, and similar information.
% You can include group authors, but please include a list of the actual authors (the group members) in the Supplementary Materials.
\author[1*$\dag$]{Christophe Rouleau-Desrochers}


%%%%%% Affiliations %%%%%%
\affil[1]{UBC}
%%%%%% Date %%%%%%
% Date is optional
\date{\today}

%%%%%% Spacing %%%%%%
% Use paragraph spacing of 1.5 or 2 (for double spacing, use command \doublespacing)
\onehalfspacing

\begin{document}

\maketitle

%%%%%% Abstract %%%%%%
\begin{abstract}
\end{abstract}
%%%%%% Main Text %%%%%%

\section{Introduction}

%%%%%%
\section{Materials and Methods}
%%%%%
\subsection{Species selection}
Acma, Alru, Bepa, Prvi, Quma were purchased from Peel's nursery and arrived on *** at Totem Field Studios (***coordinates). Alru is a fast growing species and they already arrived taller than the other species. We decided to cut a segment from the saplings and plant them in bare soil. See below for more info. 
Acne, Pist, Poba, Segi were species that were purchased in 2022 for 2023 Phaenoflex's experiment. We selected randomly these tres. At the time, these species were not used because:
\begin{itemize}
	\item Acne: root system very small (they came in carots compared to other trees that came in pots)
	\item Pist: smaller than other species
	\item Poba: after they flushed, then they were repotted and lost their leaves.
	\item Segi: smalller than other species
\end{itemize}
Again, for the same reason as the Alru, we took cuttings from Poba and replanted them in soil with the following methodology. The cuttings were stored in climate chambers with the corresponding temperature (see Hobo loggers) from February 13, 2024 to Feb 20, 2024. The tree cuttings were planted at that time. 
\par Notes from Justin: Assuming that all the information prior to planting is already noted down (e.g. what temperature the cuttings were stored at etc.)
30 cm long shoot tip cuttings of both red alder and balsam poplar were soaked at the cut wound for 15 minutes in a solution of 20 mL indole-butyric acid 0.4\% (Wilson Liquid root stimulator) diluted in 2 litres of warm tap water. (0.004\% concentration). 180 1-gallon pots were filled up to 1 inch from the lip with pre-moistened peat-based potting mix containing large pumice chunks. Soil was pressed firmly to compact. Cuttings were placed into the soil at the depth such that pre-drawned paint lines could still be visible just above the soil surface. 

\subsection{Tree measurements}
The following measurements were performed from Feb 7, 2024 to Feb 11, 2024.
1. Using red paint, we marked the trees on their trunk at ~3 cm from the soil. 2. The diameter was measured at the top of that marked. 3. Height measurements were also done using these marks. We measured height from the top of the red mark until the bottom of the highest apical bud. If there was none, we selected the highest lateral shoot. For the following measurements (winter 2025 and fall 2025), if the apical bud died, we added a colum "previous measurement valid" and input "y" if the measured shoot snapped off (because of insects, accidently broken it, etc.) and measured the highest shoot then.


\subsection{Shoot elongation}: To facilitate the shoot elongastion monitoring,  paper ruler was atached on the following species Acne, Bepa, Poba and Quma. We used A3 RiteIntheRain paper so they would survive rain. We also taped the end of each ruler with packaging tape for 2 reasons: 1. Increase the the fusion of the tape with the ruler and 2. Increase the durability of the fixation to the tree. We used Band Aid medical tape to fix the paper rulers to the trees in order for the trunk to be able to breath. Prior to the installation, using red pain, we marked where the reference point for the measurement. This is the bottom of the new-year apical shoot. For species on which we couldn't install the paper rulers, manual measurements of the shoot elongation was performed each Wednesday. 



\subsection {Fertilizer}
Using fertilizer premix from UBC's garden, I diluted by half to make it less concentrated. Dilution factor: 1:2. Then I added ~125mL to all trees on Friday, June 7, 2024. Another nutrient addition was performed to maintain the nutrient availability in the soil on 6 July 2024. See git issue \#14 for more details. \\
\textbf{2025} : On Friday 11 April 2025, 125mL of undiluted liquid fertlizer (same as 2024) was added to all trees, excluding the nitro boost treatment replicates. The latter didn't get any nutrients in spring and will get some only later on in the summer. Since we dropped the nitro treatments for the segi, all of these replicates got fertilizer. 

\subsection{Hobo loggers}
Hobo loggers (Temp/humidity) were set up in the climate chambers at the beginning of the Cool Spring treatments. They were then transferred to Totem Field at different locations and hidden behind a white sheet of paper to avoid the sun from hitting them directly. 
\par 
On June 7, 2024, Hobo loggers (Temp/light) were placed in 3 different blocks at Totem Field. They were placed at the top of PVC pipes at a height of 1m from the ground. They were placed in a position where the foliage covers of the trees would not shade them. I set 6/block. This was performed after I notice that there will be a big light difference. The plants that are the farthest from the greenhouse door receive far less light then the one closest to the door. They were configured on Sunday June 9, 2024. I also installed 4 loggers on the greenhouse roof in case the ones positionned at 1m above the soil don't record the light properly. 

\subsection{Spring and Fall treatments}
The Cool Spring treatment consisted of placing the CS replicates in climate chambers to delay the start of their growing season on March 6 2024. The WS replicates remained at Totem Field studios
\par The Warm Fall treatment consisted of placing WS/WF, CS/WF and WSWF\_nitro treatments in the climate chambers on 4 September 2024. The photoperiod was set every week on Wednesday to fit the local sunrise and sunset and was ramped until it reached full light. The temperature was set to fit the mean 30 years daily maximum temperature of one prior month. E.g. the the temperature for the first week of September was set to the temperature regime of the first week of August. The CF treatments remained at Totem Field Studios. 
\par For both climate chamber treatment, the trees were rotated and watered weekly to minimize the effects the climate chambers could have on the trees.

\subsection  {Senescence monitoring}
Every week, starting on September ***, senescence was monitored by two methods. The first being a visual assessment of the remaining green leaf cover. We used a systemic aproach to estimate what percentage of green leaf cover was remaining by comparing to what would be 100\% of cover..
From September **** to September 25, we used a chlorophylll content meter ***. On October 2, because of device failure, we switched to SPAD-502DL Plus (Konica Minolta) from Loren Rieseberg's lab. To calibrate the two instruments to values that are comparable, we used****. check: https://nph.onlinelibrary.wiley.com/doi/full/10.1046/j.0028-646X.2001.00289.x

\subsection{Shoot elongation measurements}
In 2024 and 2025, shoot elongation measurements were conducted using two distinct methods. But for both methods, the following were conducted: in 2024, we selected the shoot coming off the apical meristem when possible. If the bud died or if the shoot snapped off, we selected the closest lateral shoot. In the case of Prvi on which there are no obvious apical shoot (sometimes there will be 2 branches of equal height) we selected the highest one and if that one died, we selected the other. Then we went on the lateral shoot if both died.\\
Then using rain paint, we marked the base of the chosen bud.\\
In 2025, we preferably chose the continuous shoot that was measured the previous year. If that shoot died, we chose another shoot acording to the previous criterias. \\
Two methods:\\
  1. Paper rulers using rite in the rain paper and printed a 38cm ruler. For the species on which there was trunk/branch space on which we could safely install a paper ruler, we installed one. The species were: Acne, Bepa, Poba, Quma. Since the ruler was positioned at the bottom of the red mark, we didn't have to adjust it every time we measured shoot elongation. Then the shoot elongation was measured at the botttom of the apical bud. \\
  2. Metal rulers: using a metal scientific ruler, we measured the shoot from the bottom of the red mark to the bottom of the apical bud for deciduous species.  We measured until the top of the apical meristem for Pist.
   
Shoot elongation was measured weekly for all species. For determinate growth species, after two weeks of little or no change in elongation, we started monitoring them every second week. For indeterminate growth species, they kept on being monitored every week. 
\subsection{Leaf count}
In order to determine whether nutrient addition treatments in the fall affected leaf primordia formation, we counted the leaves on 27 May for the determinate growth species only i.e. Acne, Prvi and Quma. We counted the leaves only for the shoot coming out of the apical meristem. We always counted all the leaves on the current-year shoot measured for shoot elongation measurements.


\subsection{Biomass collection}
In the fall of 2025, when all the individuals from a species have lost all their leaves, we proceeded to remove the trees from their pots, remove the dirt by shaking them first and washing off the dirt off the roots gently with regular water gun. Then, we moved the trees, 1 bloc at a time in the drying ovens where they were left to dry for 72 hours at 70C. Not in paper bags. Then below and above ground biomass were separated by cutting the tree at 1cm above the highest root. We weighted biomass at a precision of 0.01gr
\subsection{Experimental Design}
%\includegraphics[]{../experimental_design/Fuelinex_Design_V4.jpg}

\subsection{Statistical Analysis}
%%%%%%
`section{Results}
%%%%%%
\section{Discussion}
%%%%%%

\printbibliography

\end{document}
