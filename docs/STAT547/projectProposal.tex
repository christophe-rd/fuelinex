\documentclass[11pt, oneside]{article}   	% use "amsart" instead of "article" for AMSLaTeX format
\usepackage{geometry}                		% See geometry.pdf to learn the layout options. There are lots.
\geometry{letterpaper}                   		% ... or a4paper or a5paper or ... 
%\geometry{landscape}                		% Activate for rotated page geometry
%\usepackage[parfill]{parskip}    		% Activate to begin paragraphs with an empty line rather than an indent
\usepackage{graphicx}				% Use pdf, png, jpg, or eps§ with pdflatex; use eps in DVI mode
								% TeX will automatically convert eps --> pdf in pdflatex		
\usepackage{amssymb}

%SetFonts

%SetFonts


\title{How a longer season affects growth during the current and subsequent years?}
\author{Christophe Rouleau-Desrochers and Ken Michiko Samson}
\date{\today}					

\begin{document}
\maketitle
\section*{Abstract}
The most frequently observed biological impact of anthropogenic climate change over the past decades is major changes in phenology---the timing of recurring life history events. These shifts changes when the growing season starts and when it ends, which modifies the growing season length. Earlier spring and delayed fall events support a long-lasting and intuitive assumption that earlier spring and delayed autumn events extend seasons and thus increase growth. However, research from recent years has cast doubt on this hypothesis by demonstrating that despite an earlier growth onset, longer seasons did not increase the growth rate nor overall annual increment in trees. To address this decoupling, we propose to use a full-factorial design of Cool/Warm, Spring/Fall treatments to test whether trees under experimental conditions can benefit from longer seasons during the first and the following year. More specifically, we want to demonstrate whether the treatment conditions during year one leads to a carry-over effect on the growth potential during year two. Using 15 replicates per treatment/species, for all 7 species, our modelling approach examines the relative effects of each treatment on the outcome of each year and for each species, potentially revelling how different species may respond to longer growing season which are expect to keep shifting with climate change.

\end{document}  