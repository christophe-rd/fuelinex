

\documentclass{article}
\usepackage[utf8]{inputenc}
\usepackage{authblk}
\usepackage{setspace}
\usepackage[margin=1.25in]{geometry}
\usepackage{graphicx}
\graphicspath{ {./figures/} }
\usepackage{subcaption}
\usepackage{amsmath}
\usepackage{lineno}
\linenumbers


%%%%%% Bibliography %%%%%%
% Replace "sample" in the \addbibresource line below with the name of your .bib file.
\usepackage[style=nejm, 
citestyle=numeric-comp,
sorting=none]{biblatex}
\addbibresource{sample.bib}

%%%%%% Title %%%%%%
% Full titles can be a maximum of 200 characters, including spaces. 
% Title Format: Use title case, capitalizing the first letter of each word, except for certain small words, such as articles and short prepositions
\title{fuelinex draft manuscript}

%%%%%% Authors %%%%%%
% Authors should be listed in order of contribution to the paper, by full first name, then middle initial (if any), followed by last name and separated by commas.
% Please do not use initials for first names. If you use your middle name as a full name, use an initial for the first name and spell out your full middle name.
% Use a superscript asterisk (*) to identify the corresponding author and be sure to include that person’s e-mail address. Use symbols (in this order: †, ‡, §, ||, ¶, #, ††, ‡‡, etc.) for author notes, such as present addresses, “These authors contributed equally to this work” notations, and similar information.
% You can include group authors, but please include a list of the actual authors (the group members) in the Supplementary Materials.
\author[1*$\dag$]{Christophe Rouleau-Desrochers}


%%%%%% Affiliations %%%%%%
\affil[1]{UBC}
%%%%%% Date %%%%%%
% Date is optional
\date{June 10, 2024}

%%%%%% Spacing %%%%%%
% Use paragraph spacing of 1.5 or 2 (for double spacing, use command \doublespacing)
\onehalfspacing

\begin{document}

\maketitle

%%%%%% Abstract %%%%%%
\begin{abstract}
\end{abstract}
%%%%%% Main Text %%%%%%

\section{Introduction}

%%%%%%
\section{Materials and Methods}
%%%%%
\subsection{Species selection}
Acma, Alru, Bepa, Prvi, Quma were purchased from Peel's nursery and arrived on *** at Totem Field Studios (***coordinates). Alru is a fast growing species and they already arrived taller than the other species. We decided to cut a segment from the saplings and plant them in bare soil. See below for more info. 
Acne, Pist, Poba, Segi were species that were purchased in 2022 for 2023 Phaenoflex's experiment. At the time, these species were not used because:
\begin{itemize}
	\item Acne: root system very small (they came in carots compared to other trees that came in pots)
	\item Pist: smaller than other species
	\item Poba: after they flushed, then they were repotted and lost their leaves.
	\item Segi: smalller than other species
\end{itemize}
Again, for the same reason as the Alru, we took cuttings from Poba and replanted them in soil with the following methodology. The cuttings were stored in climate chambers with the corresponding temperature (see Hobo loggers) from February 13, 2024 to Feb 20, 2024. The tree cuttings were planted at that time. 
\par Notes from Justin: Assuming that all the information prior to planting is already noted down (e.g. what temperature the cuttings were stored at etc.)
30 cm long shoot tip cuttings of both red alder and balsam poplar were soaked at the cut wound for 15 minutes in a solution of 20 mL indole-butyric acid 0.4\% (Wilson Liquid root stimulator) diluted in 2 litres of warm tap water. (0.004% concentration). 180 1-gallon pots were filled up to 1 inch from the lip with pre-moistened peat-based potting mix containing large pumice chunks. Soil was pressed firmly to compact. Cuttings were placed into the soil at the depth such that pre-drawned paint lines could still be visible just above the soil surface. 

\subsection{Tree measurements}


\subsection{Experimental Design}

\subsection{Statistical Analysis}
%%%%%%
`section{Results}
%%%%%%
\section{Discussion}
%%%%%%

\printbibliography

\end{document}
