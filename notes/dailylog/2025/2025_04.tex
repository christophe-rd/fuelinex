\begin{center}
\section*{\month}
\end{center}

\def\day{\textit{April 1st 2025}}
\def\weekday{\textit{Tuesday}}
\subsection*{\weekday, \day}
\textbf {Christophe}
After 4 hours of refreshing on NSERC's website, I finally got my response for CGS-M AND I GOT IT! Super happy. Then went to seminar which was interesting and got to speak to the speaker at lunch. Came back to the office, finished my cookies and kept on mounting my cores. Had an appointment so left at 3pm.

\def\day{\textit{April 2nd 2025}}
\def\weekday{\textit{Wednesday}}
\subsection*{\weekday, \day}
\textbf {Christophe}
Came in at 9:30 and worked on cookies on microscope till 11:30 and went back to the office. 

\def\day{\textit{April 3rd 2025}}
\def\weekday{\textit{Thursday}}
\subsection*{\weekday, \day}
\textbf {Christophe}
Lizzie's class in the morning. Monitored the trees and watered them for the ones inside of the greenhouse. Interviewed Nolan in the afternoon and worked on sanding the cores.

\def\day{\textit{April 4th 2025}}
\def\weekday{\textit{Friday}}
\subsection*{\weekday, \day}
\textbf {Christophe}
Came in at 9. Sanded the cores. Scanned the 4 left over cookies. Worked on cleaning script for the cookies and went to wine and cheese in the evening and came back to the lab to scan the first round of cores.

\def\day{\textit{April 7th 2025}}
\def\weekday{\textit{Monday}}
\subsection*{\weekday, \day}
\textbf {Christophe}
Came in at 9 and scanned the second round of cores. Worked on expenses and prepared meeting with Lizzie. 
\def\day{\textit{April 8th 2025}}
\def\weekday{\textit{Tuesday}}
\subsection*{\weekday, \day}
\textbf {Christophe}

\def\day{\textit{April 9th 2025}}
\def\weekday{\textit{Wednesday}}
\subsection*{\weekday, \day}
\textbf {Christophe}

\def\day{\textit{April 10th 2025}}
\def\weekday{\textit{Thursday}}
\subsection*{\weekday, \day}
\textbf {Christophe}

\def\day{\textit{April 11th 2025}}
\def\weekday{\textit{Friday}}
\subsection*{\weekday, \day}
\textbf {Christophe}

\def\day{\textit{April 12th 2025}}
\def\weekday{\textit{Saturday}}
\subsection*{\weekday, \day}
\textbf {Christophe}

\def\day{\textit{April 13th 2025}}
\def\weekday{\textit{Sunday}}
\subsection*{\weekday, \day}
\textbf {Christophe}

\def\day{\textit{April 14th 2025}}
\def\weekday{\textit{Monday}}
\subsection*{\weekday, \day}
\textbf {Christophe}

\def\day{\textit{April 15th 2025}}
\def\weekday{\textit{Tuesday}}
\subsection*{\weekday, \day}
\textbf {Christophe}

\def\day{\textit{April 16th 2025}}
\def\weekday{\textit{Wednesday}}
\subsection*{\weekday, \day}
\textbf {Christophe}

\def\day{\textit{April 17th 2025}}
\def\weekday{\textit{Thursday}}
\subsection*{\weekday, \day}
\textbf {Christophe}

\def\day{\textit{April 18th 2025}}
\def\weekday{\textit{Friday}}
\subsection*{\weekday, \day}
\textbf {Christophe}

\def\day{\textit{April 19th 2025}}
\def\weekday{\textit{Saturday}}
\subsection*{\weekday, \day}
\textbf {Christophe}

\def\day{\textit{April 20th 2025}}
\def\weekday{\textit{Sunday}}
\subsection*{\weekday, \day}
\textbf {Christophe}

\def\day{\textit{April 21st 2025}}
\def\weekday{\textit{Monday}}
\subsection*{\weekday, \day}
\textbf {Christophe}

\def\day{\textit{April 22nd 2025}}
\def\weekday{\textit{Tuesday}}
\subsection*{\weekday, \day}
\textbf {Christophe}

\def\day{\textit{April 23rd 2025}}
\def\weekday{\textit{Wednesday}}
\subsection*{\weekday, \day}
\textbf {Christophe}

\def\day{\textit{April 24th 2025}}
\def\weekday{\textit{Thursday}}
\subsection*{\weekday, \day}
\textbf {Christophe}

\def\day{\textit{April 25th 2025}}
\def\weekday{\textit{Friday}}
\subsection*{\weekday, \day}
\textbf {Christophe}

\def\day{\textit{April 26th 2025}}
\def\weekday{\textit{Saturday}}
\subsection*{\weekday, \day}
\textbf {Christophe}

\def\day{\textit{April 27th 2025}}
\def\weekday{\textit{Sunday}}
\subsection*{\weekday, \day}
\textbf {Christophe}

\def\day{\textit{April 28th 2025}}
\def\weekday{\textit{Monday}}
\subsection*{\weekday, \day}
\textbf {Christophe}

\def\day{\textit{April 29th 2025}}
\def\weekday{\textit{Tuesday}}
\subsection*{\weekday, \day}
\textbf {Christophe}

\def\day{\textit{April 30th 2025}}
\def\weekday{\textit{Wednesday}}
\subsection*{\weekday, \day}
\textbf {Christophe}
