

\documentclass{article}
\usepackage[utf8]{inputenc}
\usepackage{authblk}
\usepackage{setspace}
\usepackage[margin=1.25in]{geometry}
\usepackage{graphicx}
\graphicspath{ {./figures/} }
\usepackage{subcaption}
\usepackage{amsmath}
\usepackage{lineno}
\linenumbers



%%%%%% Title %%%%%%
\title{Fuelinex biomass collection protocol}

%%%%%% Authors %%%%%%
\author[1*$\dag$]{Christophe Rouleau-Desrochers}


%%%%%% Affiliations %%%%%%
\affil[1]{UBC}
%%%%%% Date %%%%%%
% Date is optional
\date{31 July 2025}

%%%%%% Spacing %%%%%%
% Use paragraph spacing of 1.5 or 2 (for double spacing, use command \doublespacing)
\onehalfspacing

\begin{document}

\maketitle

Once all the leaves have fallen from all the individuals within a given species can the biomass collection start. 
\begin{enumerate}
	\item Within the same day, measure diameter and height of \textbf{all} trees of that species at the red mark.
	\item By species, bloc and by treatment, process the trees 5 at a time.
	\item Taking the base of the trunk, remove the tree from its pot. 
	\item Take a photo of \textbf{1 individual/Species/Bloc/Treatment} of the whole tree from the bottom of the root system to the top of the tree. Make sure the tag is visible
	\item Taking the water hose, remove a maximum of the soil from the root system by being extremely delicate to not break the fine roots. 
	\item \textit{Using a hand saw, cut the tree above the root collar to get a twig of 2cm. }
	\item \textit{Place the corresponding tag on a flask and put the twig inside it. }
	\item \textit{Place the corresponding tag on the roots.}
	\item \textit{Fill the flask with alcohol and close it}
	\item Place the tree in a plastic bine to be brought to Forestry.
	\item Place the tree in the drying oven (80C). Mark down the time and date and which individuals have been placed in the oven. 
	\item After (time tbd), remove the trees from the oven and weight them using a balance. 
\end{enumerate}
\end{document}
