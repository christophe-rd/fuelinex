

\documentclass{article}
\usepackage[utf8]{inputenc}
\usepackage{authblk}
\usepackage{setspace}
\usepackage{natbib}
\usepackage{hyperref}
%\usepackage{cite}
\usepackage[margin=1in]{geometry}
\usepackage{array}
\usepackage{graphicx}
\usepackage{caption}
\graphicspath{ {./figures/} }
\usepackage{subcaption}
\usepackage{amsmath}
\usepackage{lineno}
\usepackage{soul}
\usepackage{xcolor}
\sethlcolor{yellow}
\linenumbers

%%%%%%%%%%%%%%%%%
\title{Evidence that growing season length and tree growth are decoupled in an urban arboretum}
\date{\today}
\author{Christophe Rouleau-Desrochers,  Neil Pederson, Victor van der Meersch, E.M. Wolkovich}
\begin{document}
%%%%%%%%%%%%%%%%%%%%%%%%%%%%%%

\maketitle

% The body of the abstract may be up to 400 words maximum.   
% The abstract should not contain any headings.   
% All abstracts are expected to report on work relevant to the field of ecology.   
% All abstracts are expected to report on new contributions (i.e., contributions that have not been previously published). A project that reviews current understanding (e.g., published research), such as meta-analyses, are considered “new work” if that review has not been previously published.  
% The abstract must report specific new knowledge (quantitative, qualitative, or conceptual). The results, outcomes, or knowledge may be preliminary, but they may not be vague. Abstracts without explicitly stated novel results, outcomes, or knowledge will be rejected.  
% Abstracts must be clear. Poorly written abstracts will be rejected.   
% Abstracts must be written in English and must follow standard grammar and punctuation rules. Abstracts that do not meet this guideline will be rejected.   


% Provide sufficient background information for the reader to understand the motivation for the work to be presented.  

% Clearly articulate the goals and objectives of the work. Where appropriate, specific research questions and hypotheses should also be clearly articulated (e.g., research projects, meta-analyses, etc).  

% Clearly articulate the approaches or methods employed to arrive at the results, outcomes, or conclusions produced by the study. For abstracts reporting on a research project, the specific methods used should be summarized; for more conceptual, theoretical, applied, or other projects, the general approach or framework must be summarized.  

% Clearly summarize key outcomes or contributions from the work; these may be in the form of quantitative results (e.g., for research-focused studies) or qualitative outcomes or knowledge produced.  

% Conclude with one or more ecologically relevant take-home messages.

%<><><><><><><><><><><><><><><><><><><><><><><><><><><><><><><><><><><><><><><><>
% ABSTRACT %
% 175 words max
%<><><><><><><><><><><><><><><><><><><><><><><><><><><><><><><><><><><><><><><><>

% Background
Anthropogenic climate change affects many natural systems at the global scale. The most frequently observed biological impact of climate change---shifts in the timing of recurring life history events (phenology)---is likely to have cascading/additional/knock-on effects. For trees, shifted phenology has extended the vegetative growing season, which is widely expected to increase tree growth, with important effects on forest carbon sequestration dynamics. However, multiple recent studies have failed to find this relationship and suggested shifts in drought or competition may prevent increased growth. Here, we address this decoupling by leveraging two unique datasets of vegetative phenology and growth (tree rings) data, one from a common garden and the other from a citizen science program located in an urban Arboretum, where drought and competition are limited
% Goals
With these two datasets, we aim to provide an explanation for the recently observed decoupling between the growing season length and tree growth.
% Methods
First, with the common garden project, we monitored leaf phenology for three years across four species and four provenances (75 individuals), and we collected cross sections spanning seven years of growth data through tree rings. Second, we leverage nine years of phenology data collected by citizen scientists across 11 species of mature trees (50 individuals), which we relate to their tree rings. We analyzed how the growing season length drives tree growth across our studied species using a three-level Bayesian hierarchical model.
% Results
Across our 14 deciduous tree species of different age classes over 10 years of growing season length data spanning 111 to 157 days, we found contrasting evidence that trees grow more during longer seasons. Indeed, nine of our species did not change their growth; two grew less, and three grew more with longer seasons. Fast-growth species should opportunistically shift their growth with changing conditions. In contrast, our results suggest that some of the most responsive species to changing season length were slower-growing, conservative species. In addition, our juvenile and mature tree comparison shows that juvenile trees were less flexible than the mature trees, which differs from some previous studies. Moreover, with the common garden study, we show an absence of clear local adaptation, suggesting provenance effects may be weak for growth under current conditions, which could inform assisted-migration strategies.
% Take home
Together, these two complementary datasets indicate that longer growing seasons do not uniformly increase tree growth in an urban Arboretum, and that factors other than drought and competition may underlie this observed decoupling. This could substantially affect future forest carbon sequestration dynamics in the context of a rapidly changing climate. 

\end{document}





