

\documentclass{article}
\usepackage[utf8]{inputenc}
\usepackage{authblk}
\usepackage{setspace}
\usepackage{natbib}
\usepackage{hyperref}
%\usepackage{cite}
\usepackage[margin=1in]{geometry}
\usepackage{array}
\usepackage{graphicx}
\usepackage{caption}
\graphicspath{ {./figures/} }
\usepackage{subcaption}
\usepackage{amsmath}
\usepackage{lineno}
\usepackage{soul}
\usepackage{xcolor}
\sethlcolor{yellow}
\linenumbers

%%%%%%%%%%%%%%%%%
\title{Evidence that growing season length and tree growth are decoupled in an urban arboretum}
\date{\today}
\author{Christophe Rouleau-Desrochers,  Neil Pederson, Victor van der Meersch, E.M. Wolkovich}
\begin{document}
%%%%%%%%%%%%%%%%%%%%%%%%%%%%%%

\maketitle

% The body of the abstract may be up to 400 words maximum.   
% The abstract should not contain any headings.   
% All abstracts are expected to report on work relevant to the field of ecology.   
% All abstracts are expected to report on new contributions (i.e., contributions that have not been previously published). A project that reviews current understanding (e.g., published research), such as meta-analyses, are considered “new work” if that review has not been previously published.  
% The abstract must report specific new knowledge (quantitative, qualitative, or conceptual). The results, outcomes, or knowledge may be preliminary, but they may not be vague. Abstracts without explicitly stated novel results, outcomes, or knowledge will be rejected.  
% Abstracts must be clear. Poorly written abstracts will be rejected.   
% Abstracts must be written in English and must follow standard grammar and punctuation rules. Abstracts that do not meet this guideline will be rejected.   


% Provide sufficient background information for the reader to understand the motivation for the work to be presented.  

% Clearly articulate the goals and objectives of the work. Where appropriate, specific research questions and hypotheses should also be clearly articulated (e.g., research projects, meta-analyses, etc).  

% Clearly articulate the approaches or methods employed to arrive at the results, outcomes, or conclusions produced by the study. For abstracts reporting on a research project, the specific methods used should be summarized; for more conceptual, theoretical, applied, or other projects, the general approach or framework must be summarized.  

% Clearly summarize key outcomes or contributions from the work; these may be in the form of quantitative results (e.g., for research-focused studies) or qualitative outcomes or knowledge produced.  

% Conclude with one or more ecologically relevant take-home messages.

%<><><><><><><><><><><><><><><><><><><><><><><><><><><><><><><><><><><><><><><><>
% ABSTRACT %
% 175 words max
%<><><><><><><><><><><><><><><><><><><><><><><><><><><><><><><><><><><><><><><><>

% Background
Anthropogenic climate change affects many natural systems at the global scale. The most frequently observed biological impact of climate change---shifts in the timing of recurring life history events (phenology)---is likely to have cascading/additional/knock-on effects. For trees, shifted phenology has extended the vegetative growing season, which is widely expected to increase tree growth, with important effects on forest carbon sequestration dynamics. However, multiple recent studies have failed to find this relationship and suggested shifts in drought or competition may prevent increased growth. 

% Goals
Here, we leverage two unique datasets of vegetative phenology and growth (tree rings) data, one from a common garden and the other from a citizen science program. With these two datasets, we aim to provide an explanation for the recently observed decoupling between growing season length and tree growth.

% Methods
The common garden project has four species of four provenances and we have three years of phenology and 7 years of growth data (tree rings) on juvenile trees. At the same location, we have nine years of phenology data and over 30 years of growth from fully mature deciduous trees. With these data, we built Bayesian hierarchical models to understand how the growing season length drives tree growth across our studied species of different provenances. Our observational projects provide the rare opportunity to investigate the relationship between growth and vegetative phenology in an urban Arboretum, where drought nor competition are unlikely to limit growth

%Here, we leverage two unique datasets of vegetative phenology and growth (tree rings) data, one from a common garden and the other from a citizen science program, collected in an urban arboretum where drought and competition are limited. Our two data sets provide the unique perspective of investigating the growth and growing season length relationship decoupling for two different age classes. One project examines juvenile trees, crucial for forest regeneration and the other looks at fully mature trees. 

% Results
Across our 14 tree species over 10 years of growing season length data spanning 111 to 157 days, we found mixed evidence that trees grow more during longer seasons. Indeed, the mature trees were less flexible than the juveniles, but life history strategies appears to play a role in our trees' reponses to longer seasons. Indeed, slow growth, long-lived species responded less to changing season length wheras the fast growth, short-lived species could take advantage of these shifting conditions. This pattern was absent from our juvenile trees   some species appear to grow more with longer and warmer seasons, others do not grow more and sometimes even grow less compared to shorter seasons. In addition, the common garden data support the recently observed decoupling between growing season length and growth, but suggest it may be driven by other constraints than currently proposed.

% Take home

\end{document}





