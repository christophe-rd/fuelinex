

\documentclass{article}
\usepackage[utf8]{inputenc}
\usepackage{authblk}
\usepackage{setspace}
\usepackage{natbib}
\usepackage{hyperref}
\usepackage[margin=1in]{geometry}
\usepackage{array}
\usepackage{graphicx}
\usepackage{caption}
\graphicspath{ {./figures/} }
\usepackage{subcaption}
\usepackage{amsmath}
\usepackage{lineno}
\usepackage{soul}
\usepackage{xcolor}
\sethlcolor{yellow}
\linenumbers

%%%%%%%%%%%%%%%%%
\title{Proposed Research Outline}
\date{\today}
\author{Christophe Rouleau-Desrochers}
\begin{document}
%%%%%%%%%%%%%%%%%%%%%%%%%%%%%%

\maketitle

%Provide a detailed yet concise description of your proposed research project for the period during which you are to hold the award. Be as specific as possible. Provide background information to position your proposed research within the context of the current knowledge in the field. State the significance of the proposed research to a field or fields in the NSE. State the objectives and hypothesis and outline the experimental or theoretical approach to be taken (citing literature pertinent to the proposal) and the methods and procedures to be used.
%If the proposed research is a continuation of your thesis, clearly state the differences between work done for your thesis and the research activities outlined in this proposal.


%<><><><><><><><><><><><><><><><><><><><>
% CONTEXT %
%<><><><><><><><><><><><><>\section{Introduction}
Human activity, notably their greenhouse gas emissions, are likely to have long-lasting consequences on worldwide ecosystem processes. While the changes that happened to climate in the past decades are attributed to human activity, the precision of the  consequences on a variety of environmental processes are debatable. While this is true, impact of climate change on spring and fall phenological events are clear and they are shifting, but again, the consequences on species fitness remain unclear. More precisely, spring phenological events have been advancing from 0.5 (Wolfe2005Climate) to 4.2 days/decade (Chmielewski and Rotzer2001;Fu2014Recent), and autumn phenophases (e.g. budset and leaf colouring) show either though much weaker the the spring's (Gallinat2015Autumn; Medvigy2014Macroscale, Menzel2006European). The former is mainly driven by temperature (Chuine2010, Cleland2007, Penuelas et Fiella2009), while the drivers of the later are far less known for many reasons (such as power data quality and overall less attention), but the belief is that autumn phenophases, are driven by shortening photoperiod (Flynn and Wolkovich2018; Korner and Basler2010, Cooke2010TheDynamic) and colder temperatures (Cooke2012, Delpierre2016, Lang1087). While there is still uncertainty about the extent of fall event shifts, there is a long-lasting ---and intuiative---  assumption that these shifts lead to longer seasons which translate into increased growth. Reserch from the past three years have casted doubt on this hypothesis. For instance Dow 2022, showed that despite an earlier growth onset, growth rate nor overall annual increment were increased by a theroritical longer season. 
\par
This work led to my research questions. What are the mechanisms pushing down trees from grow more despite theoritical longer growing seasons. Overall, there are two possible explanations: internal (via physiological constraints) or external (environmental) limits to growth. Tree growth could be limited physiologically since these mechanisms are under strict genetic control and there is debate whether internal constraints are specific to species or are universal (zohner2023). As climate change has a complex and rather hard-to-predict nature, the external limits to growth are hard to quantify at the individual level as they affect communities as a whole. Drought, spring frost and heat waves are commonly known to be the main mechanisms that could limit tree growth under climate change (Drobyshev2008influence). To achieve a better comprehension of these mechanisms, experiments are paramount as they have the huge advantage of excluding variables that covary in nature, highlighting the ones of interest. With experiments, we can tease out for example, co-occuring warmer earlier spring events from severe drought later in the season, a common reality in natural environments for which teasing apart the impacts on carbon sequestration of these two situations is merely impossible. However, experiments come with a high cost of time and labor and while studying saplings (as mature trees can't be used for experiments) is of critical importance for forest regenaration projections, they can hardly be translated to mature trees which hold for the overwhelming carbon biomass proportion of forests. For this, drone imagery paired with machine learning hold the promise of acquiring immensely large sample sizes beyond was is achievable using traditional ground work, while increasing significantly the spatial and temporal resolution of sattelites. Therefore, I propose to use a combination or experiments to test internal (Chapter 1), along with external (Chapter 2) limits to growth and observational data of a mixed-forest community using UAVs imagery and AI (Chapter 3) to address the paradox of the absence of increased growth despite apparent better conditions.   

% Etienne's text: A major empirical challenge when studying canopy tree responses to changing climate is that in any landscape, responses vary strongly across species, and even within species across environmental gradients. Robust quantification of these complex species-environment responses requires large sample sizes well beyond what is achievable with traditional plot-based surveys, which contain relatively few canopy trees. Satellite remote sensing observations are too coarse to reliably quantify responses at individual tree level or identify tree species.


%<><><><><><><><><><><><><><><><><><><><>
% CHAPTER 1 %
%<><><><><><><><><><><><><><><><><><><><>
\section*{Chapter 1: extended growing season experiment (Fuelinex) continuation}
To understand 

%<><><><><><><><><><><><><><><><><><><><>
% CHAPTER 2 %
%<><><><><><><><><><><><><><><><><><><><>
\section*{Chapter 2: drought and spring frost experiment}
With climate change, not only the growing season length shifts, but trees will experience shifts in the timing of moisture deficits from lower precipitation and higher evapotranspiration. This is commonly referred to as drought stress (dox2022severe) and tree-ring (not sure) research shows that summer droughts advances growth cessation---leading to an earlier end of season. When droughts are of longer period and/or higher intensity, they may lead to xylem embolism, which affects water transport, reduces water use efficiency, increases the possibility of disease and insect pests and can lead to tree death which has been known to occur in many regions across the Northern Hemisphere (Kang2023An earlier). A common phenomenon of climate change induced shifts in phenology are increasing the frequency and severity of late spring frosts. This can lead to tissue loss for which trees can recover by reinvesting in a second cohort of leaves. However, the lost time that trees cannot photosynthesize along with the increased investment in the second cohort may lead to significant disadvantages, but it's unclear whether trees going through spring frosts grow less. \\
To investigate these processes, I will conduct an experiment consisting of three drought treatments, occuring at different timings during the growing season and an additional two treatments testing for spring frost events both at the start of the growing season when the buds have just started to burst and later, once the leaves have fully leaf out. I will use twelve deciduous tree species (six congeneric pairs to avoid potential confounding effects of shared evolutionary history) native to North America. The spring frost treatments will consist of placing the trees in growth chambers at a temperature 5 degrees warmer than ambient condition to trigger budburst. When the trees started to burst, I will place the first treatment for one hour in freezing growth chambers at a budkilling temperature set to the specific specied LT50 values. For the second spring frost treatment, I will wait for the leaves to be elongated and then place the trees under the same freezing condition as the first treatment (Zohner2018Increased autumn).\\
For the drought treatments, the climate conditions will be set at 5 degrees warmer than ambient condition, at very low air humidity to maximize evapotranspiration rate. Once all the trees of a specific species have reached their respective wilting point (values at which soil water is not extractable by the plant), the trees will be removed from the chambers and put back to ambient condition at constant irrigation. The three drought treatments won't differ in their conditions, but instead in their timing since little attention has been payed to the importance of the moment of the drought. Therefore, the first treatment will be conducted just after leaf-out and before summer solstice. The second will be start one week before solstice --- timing of peak growth for a lot of species***. The last drought treatment will occur just at the beginning of budset which is the theoritical end of early wood growth and onset of late wood growth. Phenological phases, shoot elongation will be monitored weekly throughout the growing season. Biomass will be estimated using allometric equations at the start and end of the growing season. In order to grasp a high temporal resolution of growth responses to treatments, 5 replicates per species/treatments will be equipped with micro magnetic dendrometers that will provide valuable previously impossible to get onto how trees respond to growth at an extremely high temporal resolution. 

%<><><><><><><><><><><><><><><><><><><><>
% CHAPTER 3 %
%<><><><><><><><><><><><><><><><><><><><>
\section*{Chapter3 : cambial phenology X drone imagery phenological observations}
Getting a better understanding of accross species differences in growth synchrony with leaf phenology is paramount to refine carbon sequestration models in the face of Anthropogenic climate change. Thus, I aim to launch a large-scale project using cutting-edge technology to gatter a large amount of data of trees' growth onset and end from a mixed-forest community located at Station bilogique des Laurentides (St-Hypollyte (Qc)), during three growing seasons. Using this site will allow me to follow up work previously done by my lab as well as creating a partnership with Dr. Etienne Laliberté from the Plant Functional Ecology Laboratory (PFEL) who curently uses this site for his research. To monitor leaf phenology from budburst to leaf drop, I will use high-frequency repeat overflights using Unmaned Aerian Vehicles (UAVs) over the canopy of these for communities to monitor every single trees over the course of the growing season. Partnering with the PFEL will allow me to automatically acquire hundreds of close-up photos of individual tree crowns per day to monitor critical phenophases. Along with this imagery, I will use 200 DC3 Perimeter Dendrometer placed randomly throughtout the site on 40 trees per species. I believe that using high resolution data accross space and time will allow me to infer a strong relationship between leaf phenophases and growth seasonality

%<><><><><><><><><><><><><><><><><><><><>
% REFERENCES
%<><><><><><><><><><><><><><><><><><><><>
\section{References}
\bibliography{ExportedItems.bib}
\bibliographystyle{pnas} % set citation style 

\end{document}
