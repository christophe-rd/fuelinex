
\documentclass[12pt]{article}
%\renewcommand{\baselinestretch}{1.5}
\usepackage{sectsty,setspace}
\usepackage[top=1.87cm, bottom=1.87cm, left=1.87cm, right=1.87cm]{geometry} 
\usepackage{epstopdf} % for pdf creation
\usepackage{amsmath,latexsym,amssymb,wasysym} % improving structure
\usepackage{natbib}
\usepackage{hyperref} % to get the little green boxes around the refs
\usepackage{times} % setting font to times
\usepackage{fancyhdr} % for custom headers/footers

\setstretch{1}
%\setlength{\baselineskip}{0.167in} 
\setlength\parindent{0pt} % no indents throughout

%%%%%%%%%%%%%%%%%
% Setup header
\begin{document}
\pagestyle{fancy}
\fancyhf{} % clear header/footer
\fancyhead[R]{Christophe Rouleau-Desrochers} % name on right side
\fancyfoot[C]{\thepage}
\renewcommand{\headrulewidth}{0pt}

%\title{Climate change and forests: remote sensing and experiments to understand tree growth}
%\date{\today}
%\author{Christophe Rouleau-Desrochers}
%\begin{document}
%%%%%%%%%%%%%%%%%%%%%%%%%%%%%%

\section*{Climate change and tree growth: drones and experiments to refine forest carbon sequestration projections}
%\maketitle
%Provide a detailed yet concise description of your proposed research project for the period during which you are to hold the award. Be as specific as possible. Provide background information to position your proposed research within the context of the current knowledge in the field. State the significance of the proposed research to a field or fields in the NSE. State the objectives and hypothesis and outline the experimental or theoretical approach to be taken (citing literature pertinent to the proposal) and the methods and procedures to be used.
%If the proposed research is a continuation of your thesis, clearly state the differences between work done for your thesis and the research activities outlined in this proposal.
%<><><><><><><><><><><><><><><><><><><><>
% CONTEXT %
%<><><><><><><><><><><><><><><><><><><><>
\textbf {Context:} There is increasing evidence that anthropogenic climate change, and particularly increased temperature, affect many natural systems at the global scale \cite{parmesan_poleward_1999,rosenzweig_attributing_2008}. The most observed biological impact of climate change over the past decades are major changes on spring and fall phenology --- the timing of recurring life history events \cite{parmesan_globally_2003}. Understanding the consequences of these shifts on ecosystems requires understanding how much the growing season has changed \cite{duputie_phenological_2015}. Spring phenological events (e.g. budburst and leafout) have been advancing from 0.5 \cite{wolfe_climate_2005} to 4.2 days/decade \cite{chmielewski_response_2001,fu_recent_2014} and are mainly driven by temperature \cite{chuine_why_2010,cleland_shifting_2007,penuelas_responses_2001}. In contrast, autumn phenophases (e.g. budset and leaf colouring) are delayed, though to a much lesser extent than spring's \cite{gallinat_autumn_2015,jeong_macroscale_2014} and are driven by shortening photoperiod \cite{cooke_dynamic_2012,flynn_temperature_2018,korner_phenology_2010} and colder temperatures \cite{cooke_dynamic_2012,delpierre_temperate_2016}. These shifts support a long-lasting and intuitive assumption that earlier spring and delayed fall events lead to longer seasons ---and thus increased growth \cite{keenan_net_2014}. However, research from the past three years has cast doubt on this hypothesis \cite{dow_warm_2022,green_limits_2022,silvestro_longer_2023}. Recently, \cite{dow_warm_2022}** showed that despite an earlier growth onset, neither growth rate nor overall annual increment was increased by longer seasons. This could have large consequences on carbon cycle models with potential feedback on future climate change.\\
Understanding these findings resumes answering what are the mechanisms that limit trees from growing more despite theoretically longer growing seasons? My research focuses on this via considering two possible drivers explaining this phenomenon: external (environmental) \cite{kolar_response_2016} or internal (via physiological constraints)\cite{zohner_effect_2023} limits to growth. The complex nature of climate change makes predicting the external limits to growth hard to quantify at the individual level, as they affect communities as a whole. Drought, spring frost and heat waves are commonly mentioned as the main extreme events that could limit tree growth under climate change \cite{tyree_xylem_2002, choat_triggers_2018, li_widespread_2023,trenberth_global_2014,intergovernmental_panel_on_climate_change_detection_2014,chiang_evidence_2021,polgar_leafout_2011,reinmann_compensatory_2023}. To better comprehend these mechanisms, experiments are paramount because they can robustly tease apart the external vs internal drivers (e.g. warmer springs from severe drought later in the season ---a common co-occurring reality in natural environments) \cite{morin_changes_2010,primack_observations_2015}. This is essential to refine forest carbon sequestration projections \cite{green_limits_2022,cabon_cross-biome_2022}. However, experiments are most often performed on juvenile trees, which are critical for their role in forest regeneration projections, but their responses can hardly be translated to mature trees, which hold the overwhelming carbon biomass proportion of forests \cite{augspurger_differences_2003,silvestro_longer_2023,vitasse_ontogenic_2013}. To investigate how growing season shifts impact mature trees in their natural environments, Unmanned aerial vehicle (UAV) imagery paired with machine learning have the capacity to acquire huge sample sizes largely beyond what is achievable via traditional observational ground work, and at a better spatial and temporal resolution than satellites \cite{berra_assessing_2019,piao_plant_2019,teng_bringing_2025}. Here, I propose a combination of two experiments to test internal (Chapter 1) and external (Chapter 2) limits to growth along with a large-scale mixed-forest observational data project using UAV imagery and machine learning (Chapter 3) to address the paradox of the absence of increased growth despite apparently improved growing season conditions. \\
%<><><><><><><><><><><><><><><><><><><><>
% CHAPTER 1 %
%<><><><><><><><><><><><><><><><><><><><>
\textbf{Chapter 1: Extended growing season experiment (MSc thesis continuation)}
Shifts in phenological phases have consequences on growth during the current growing season, but experiments to date have yet to quantify if longer seasons have lagging effects over the following years \cite{chapin_ecology_1990,landhausser_partitioning_2012,lawrence_variable_2018,martens_first-year_2007,schott_premature_2013}. Therefore, for the first two years of my proposed award tenure, I will expand my Master's work by analyzing 2025 data and extend the project for a third consecutive year (2026). Along with the continuation of phenological phase monitoring over the whole growing season, I will also collect the biomass and tree cross-sections at the end of the 2026 growing season. I will also collaborate with researchers the WSL from ETH Zurich to do cellular scans of the trees' cross-sections to understand how treatment affects cell count and morphology. By using growth increment metrics and tree ring cell analysis collected for three seasons, I will investigate how the first year treatments affect growth over the two following years. With the expansion of my Master's work, I aim to assess different tree species’ potential to stretch their activity schedules and wether or not this translates into increased growth over multiple growing seasons. \\
%<><><><><><><><><><><><><><><><><><><><>
% CHAPTER 2 %
%<><><><><><><><><><><><><><><><><><><><>
\textbf{Chapter 2: Drought and spring frost experiment}
With climate change, not only will growing season length shift, but trees will also experience shifts in the timing of moisture deficits from lower precipitation and higher evapotranspiration that may lead to drought stress \cite{dox_wood_2022}. Tree-ring research shows that summer droughts advance growth cessation---leading to an earlier end of season \cite{kang_earlier_2023} and potential tissue loss \cite{kramer_why_2012}. In addition, warming springs advance budburst, but come with increased frequency and severity of late spring frosts resulting in tissue loss \cite{baumgarten_no_2023,kramer_why_2012}. Trees can recover by reinvesting in a second cohort of leaves \cite{baumgarten_no_2023,dandrea_winters_2019}. However, the lost time that trees cannot photosynthesize along with the increased investment in the second cohort of leaves may lead to significant disadvantages; however, it is unclear whether trees exposed to spring frosts also grow less \cite{chamberlain_late_2021,baumgarten_no_2023}. 
To investigate how these two abiotic drivers affect trees, I will conduct an experiment during the second year of the award tenure that consists of three drought treatments, occurring at different timings during the growing season and an additional two spring frost treatments early and late in the spring. I will use 15 replicates of 12 deciduous North American tree species (six congeneric pairs to avoid potential confounding effects of shared evolutionary history), spanning different life history strategies, for all five treatments and a control, summing a total of 1080 individuals (a sample size consistent with my current successful experiment). For spring frost treatments, I will place the trees in growth chambers early in the season at warm temperatures to trigger budburst. When the trees start to burst, I will place the first treatment for one hour in freezing growth chambers, following methods in \citet{chamberlain_late_2021}**. For the second spring frost treatment, I will wait for the leaves to be fully elongated and then place the trees under the same freezing conditions as the first treatment \cite{zohner_increased_2018}.
For the drought treatments, I will move the trees in growth chambers at a warmer temperature and lower air humidity than ambient conditions to maximize evapotranspiration rates. Once the trees have reached their respective wilting point (values at which soil water is not extractable by the plant), I will remove them from the chambers, one species at a time and move them back to ambient conditions and constant irrigation. The three drought treatments will differ in their timing of occurence to test the importance of drought timing. Thus, the first treatment will be conducted just after leaf-out; the second will start one week before solstice --- period of peak growth for a lot of species \cite{anderson-teixeira_carbon_2021,dorangeville_drought_2018,mcmahon_general_2015}; the last drought treatment will happen near the end of the season, just before growth cessation. I will monitor phenological phases and shoot elongation every week throughout the growing season. Using allometric equations, I will estimate biomass at the start and end of the growing season. In order to grasp a high temporal resolution of growth responses to treatments, I will equip a subset of trees with magnetic dendrometers that will provide valuable insight into growth temporality in response to treatments. \\
%<><><><><><><><><><><><><><><><><><><><>
% CHAPTER 3 %
%<><><><><><><><><><><><><><><><><><><><>
\textbf{Chapter 3 : Cambial phenology $\times$ drone imagery phenological observations}
An improved understanding of differences in growth synchronicity with leaf phenology across species is paramount to refining carbon cycles models in the context of Anthropogenic climate change \cite{klein_coordination_2016,kramer_importance_2000,richardson_climate_2013,swidrak_comparing_2013}. Thus, for the three years of my award tenure, I aim to launch a large-scale project using cutting-edge drone $\times$  artificial intelligence technologies \cite{ball_accurate_2023,teng_bringing_2025,ulku_deep_2022} to gather a large amount of data on tree growth onset and end from a mixed-forest community located at Station biologique des Laurentides (St-Hypollyte (Qc)), during three consecutive growing seasons. Using this site will allow me to follow up on work previously done by my laboratory \cite{flynn_temperature_2018} as well as creating a partnership with Dr. Etienne Laliberté from the Plant Functional Ecology Laboratory (PFEL) who curently uses this site for his research \cite{cloutier_influence_2024}. To monitor leaf phenology from budburst to leaf drop, I will use high-frequency repeated overflights using UAVs over the canopy to monitor every canopy tree over the course of the growing season. Then I will use BalSAM, a promising model to accurately and efficiently segment tree crown from repeated UAV images \cite{teng_bringing_2025}. This will allow me to gather large amount of accurate phenological data from single trees within the forest community \cite{teng_bringing_2025}. With this data, I will be able to accurately infer the start and end of the growing season for each species and individual within this forest community \cite{berra_assessing_2019,fawcett_monitoring_2021}. Then, I will use 200 DC3 Perimeter Dendrometers placed randomly throughout the site on 40 trees per species. Using high-resolution data across space and time will allow me to robustly infer a relationship between leaf phenophases and growth seasonality.\\
%<><><><><><><><><><><><><><><><><><><><>
% OUTREACH %
%<><><><><><><><><><><><><><><><><><><><>
\textbf{Outreach}
Given the widespread impacts of climate change on ecosystems, understanding how forest communities respond to prolonged growing seasons is crucial. Observing the reactions of deciduous tree species to extended seasons may reveal potential benefits for some species and harm for others. These shifts are likely to influence forest stand dynamics across North America with potential feedback with future climate change. Therefore, using two different experiments and a large-scale remote sensing project, I aim to understand how the growth dynamics of North American tree species will change with longer growing seasons.  

%<><><><><><><><><><><><><><><><><><><><>
% REFERENCES
%<><><><><><><><><><><><><><><><><><><><>
\newpage
\bibliography{bibproposed.bib}
\bibliographystyle{nature} % set citation style 

\end{document}
