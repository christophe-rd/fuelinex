

\documentclass{article}
\usepackage[utf8]{inputenc}
\usepackage{authblk}
\usepackage{setspace}
\usepackage{natbib}
\usepackage{hyperref}
%\usepackage{cite}
\usepackage[margin=1in]{geometry}
\usepackage{array}
\usepackage{graphicx}
\usepackage{caption}
\graphicspath{ {./figures/} }
\usepackage{subcaption}
\usepackage{amsmath}
\usepackage{lineno}
\usepackage{soul}
\usepackage{xcolor}
\sethlcolor{yellow}
\linenumbers

%%%%%%%%%%%%%%%%%
\title{Proposed Research Outline}
\date{\today}
\author{Christophe Rouleau-Desrochers}
\begin{document}
%%%%%%%%%%%%%%%%%%%%%%%%%%%%%%

\maketitle

%Provide a detailed yet concise description of your proposed research project for the period during which you are to hold the award. Be as specific as possible. Provide background information to position your proposed research within the context of the current knowledge in the field. State the significance of the proposed research to a field or fields in the NSE. State the objectives and hypothesis and outline the experimental or theoretical approach to be taken (citing literature pertinent to the proposal) and the methods and procedures to be used.
%If the proposed research is a continuation of your thesis, clearly state the differences between work done for your thesis and the research activities outlined in this proposal.


%<><><><><><><><><><><><><><><><><><><><>
% ABSTRACT %
%<><><><><><><><><><><><><><><><><><><><>

%\section {References}
%\bibliography{Exported_Items.bib}
%\bibliographystyle{nature} % set citation style 



\end{document}
