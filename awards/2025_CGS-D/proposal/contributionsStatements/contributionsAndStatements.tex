\documentclass[12pt]{article}
\usepackage{sectsty,setspace}
\usepackage[top=1.87cm, bottom=1.87cm, left=1.87cm, right=1.87cm]{geometry} 
\usepackage{epstopdf}
\usepackage{amsmath,latexsym,amssymb,wasysym}
\usepackage{times}
\usepackage{fancyhdr}

\setstretch{1}
\setlength{\baselineskip}{0.167in}
\setlength\parindent{0pt}

\pagestyle{fancy}
\fancyhf{}
\fancyhead[R]{Christophe Rouleau-Desrochers}
\fancyfoot[C]{\thepage}
\renewcommand{\headrulewidth}{0pt}

\begin{document}

%%%%%%%%%%%%%%%%%%%%%%%%%%%%%%%%%
\subsection*{I --- Contributions to research and development}
%%%%%%%%%%%%%%%%%%%%%%%%%%%%%%%%%

\textbf{a. Articles published or accepted in peer-reviewed journals} \\[2pt]
Wolkovich, E.M., \textbf{Rouleau-Desrochers, C.}, Garcia de Cortazar-Atauri, I., Walker, M.A., Lacombe, T. (2025) Building a more predictive model of Terroir for the Anthropocene. \textit{Plants, People, Planet}. DOI: 10.1002/ppp3.70055.  \\
Role: developed the meta-analysis conceptual figure and a global map, and contributed to manuscript editing and interpretation. \\[6pt]

\textbf{c. Non-peer-reviewed contributions (for example, specialized publications, technical reports, preprints, conference presentations, posters)} \\[2pt]
\textbf{Rouleau-Desrochers, C.}, Patry, É., Gingras, G., Ucler, J., Vézina-Doré, M., Paillé, O., Moghaddam, Y.Y. (2024) ``Un été parti en fumée.'' \textit{Le Point Bio}.  \\
Role: lead author; conducted interviews with researchers and wrote the manuscript (work completed during undergraduate studies). \\

%%%%%%%%%%%%%%%%%%%%%%%%%%%%%%%%%%%%%%%
\subsection*{II --- Most significant contributions to research and development}
%%%%%%%%%%%%%%%%%%%%%%%%%%%%%%%%%%%%%%%

\textbf{1. Predictive Terroir synthesis (Plants, People, Planet, 2025).}  
This paper tackles an urgent applied problem: current vine-planting choices and limited genotype × environment data undermine adaptation planning under climate change. I translated conceptual ideas into a succinct meta-analysis figure and a map that show where data are missing and why those gaps matter for predictive models. My role combined figure creation and manuscript editing — activities that directly shaped the paper's message and its accessibility to both ecologists and applied growers. \\

\textbf{2. Public communication on Québec wildfires (Le Point Bio, 2024).}  
I wrote and co-produced a public-facing article that explained causes and health impacts of the 2023 Québec wildfires shortly after the events. Drawing on my boreal-forest background, I led expert interviews and framed complex findings in language for non-specialist readers. This work reflects my commitment to translating science into meaningful public outreach and helped me practice rapid, accurate science communication in a high-stakes context.

%%%%%%%%%%%%%%%%%%
\subsection*{III --- Applicant’s statement}
%%%%%%%%%%%%%%%%%%

\textbf{Research experience}  

The first time I really fell into natural-science research was in summer 2021 with Dr. Pierre Drapeau. Over three months, my colleagues and I travelled across Québec investigating woodpecker damage to wooden electrical infrastructure — a project that taught me rigorous field protocols, database management and the basics of scientific coding. In 2022, I held my first Undergraduate Student Research Award (USRA) and carried out intensive fieldwork on Pileated woodpecker habitat in remote boreal forest: long days of banding, nest-searching and tree-classification built my field skillset. In parralel to this, I leverage 20 years of nesting-phenology data to understand how Pileated woodpecker nesting phenology is affected by climate change---my first independent statistical project--- which gave me hands-on experience in data cleaning, model selection and interpreting of long-term ecological signals. A second USRA at UBC (supervised by Drs. Frederik Baumgarten and E.M. Wolkovich) exposed me to large-scale experiments: I conducted a smaller project on defoliation treatments, helped maintain a 1,000-tree trial, set up growth-chamber treatments and learned to deploy dendrometers. I am a co-author on this ongoing project.

In 2024, I led my own greenhouse experiment (now the foundation of my Master’s chapter) and joined field campaigns in Smithers (BC) and at Mount Rainier (WA). This built up my skill set in experimental designs, plant monitoring and climate data recording. In 2025, in parallel to working for a second year on my greenhouse experiment ---which will become part of my first PhD chapter---, I visited the Arnold Arboretum of Harvard University to collect tree cores. This would complete the data collection of my second Master's chapter. Both the cores and cross sections from a common garden project allowed me to gain valuable knowledge of dendrochronological methods. Collectively, these experiences taught me experimental design, tree-ring analysis methods, intensive field logistics, and quantitative analysis — skills I continually extend through workshops and applied projects. \\

\textbf{Relevant activities and distinctions.}  

Prior to my academic journey I completed degrees related to the restaurant and wine industries, and over eight years in that field, I built a strong work ethic and rigour. The transition from food and beverage to academia was caused by the realization that my ties to nature were stronger than I thought while working for six months a vineyard in New Zealand. Thereafter, I enrolled in a Bachelor's in biology, majoring in ecology. As an undergraduate, I applied the work ethic I learned in hospitality to my studies and research. In 2025, during the first year of my Master's degree, I deepened my quantitative skills by taking a three-week workshop on hierarchical Bayesian models and following the Statistical Rethinking course; I also present and refine my models in a biweekly Bayesian working group.

My dedication and constant efforts during my undergraduate studies at Université du Québec à Montréal were recognized with the Governor General's Academic Silver Medal, Dean's List honours, 13 competitive scholarships totalling \$76,750, and a perfect GPA of 4.30/4.30. In addition to my excellent academic record, I have always maintained active outreach: I was interviewed at the 2023 UQÀM scholarship ceremony, featured on Biochambers for our growth-chamber work, presented to Treespotters citizen scientists at Harvard, and hosted an open house at the Canadian Society for Ecology and Evolution (2024).

Across two years leading my experiment, I developed strong leadership qualities by as I trained and supervised 12 undergraduate students in growth-chamber protocols, phenological monitoring and greenhouse management. I also led Smithers fieldwork (BC) in 2025. In addition of my mentor role in my laboratory, I was a teaching assistant for FRST303 (1962 students) and FRST304 (1216 students), during which I ran office hours and gave supplementary materials to help students grasp ecological and plant-physiology concepts. Since 2021, I have mentored twelve international students during their first months in Canada, tutored seven peers in biochemistry and statistics, and volunteered with BUDR to advise students on applying to graduate school ---often drawing on my own non-linear path.

My background in restaurant management was valuable for my greenhouse experiment as I already gained technical and organizational knowledge for successful projects. However, this unique context posed certain challenges related to working with living organisms (e.g. pest management, irrigation control, climate data collection), which I overcame by building a strong network of professors and technicians who could give me resources to overcome challenges.


My wildlife photography work complements my outreach: my photos have appeared on the cover of \textit{Le Point Bio (2024)}, in the Biodiversity Research Center annual competition, in a recent PNAS article, and in \textit{Le Rappel}.



\end{document}
