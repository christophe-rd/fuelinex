\documentclass[12pt]{article}
\usepackage{sectsty,setspace}
\usepackage[top=1.87cm, bottom=1.87cm, left=1.87cm, right=1.87cm]{geometry} 
\usepackage{epstopdf}
\usepackage{amsmath,latexsym,amssymb,wasysym}
\usepackage{times}
\usepackage{fancyhdr}

\setstretch{1}
%\setlength{\baselineskip}{0.167in}
\setlength\parindent{0pt}

\pagestyle{fancy}
\fancyhf{}
\fancyhead[R]{Christophe Rouleau-Desrochers}
\fancyfoot[C]{\thepage}
\renewcommand{\headrulewidth}{0pt}

\begin{document}

%%%%%%%%%%%%%%%%%%%%%%%%%%%%%%%%%actionable
\subsection*{I --- Contributions to research and development}
%%%%%%%%%%%%%%%%%%%%%%%%%%%%%%%%%

\textbf{a. Articles published or accepted in peer-reviewed journals} \\[2pt]
Wolkovich, E.M., \textbf{Rouleau-Desrochers, C.}, Garcia de Cortazar-Atauri, I., Walker, M.A., Lacombe, T. (2025) Building a more predictive model of Terroir for the Anthropocene. \textit{Plants, People, Planet}. DOI: 10.1002/ppp3.70055.  \\
I developed the meta-analysis conceptual figure and led the global mapping analysis that identified key data gaps; I assisted with the writing and edited the manuscript describing methods and implications. \\[3pt]
%%%
\textbf{c. Non-peer-reviewed contributions (for example, specialized publications, technical reports, preprints, conference presentations, posters)} \\[2pt]
\textbf{Rouleau-Desrochers, C.}, Patry, É., Gingras, G., Ucler, J., Vézina-Doré, M., Paillé, O., Moghaddam, Y.Y. (2024) ``Un été parti en fumée.'' \textit{Le Point Bio}.  \\
My role as the lead author was to conduct the interviews with researchers and to write the manuscript (work completed as part of my undergraduate studies). 

%%%%%%%%%%%%%%%%%%%%%%%%%%%%%%%%%%%%%%%
\subsection*{II --- Most significant contributions to research and development}
%%%%%%%%%%%%%%%%%%%%%%%%%%%%%%%%%%%%%%%

\textbf{1. Predictive model of Terroir under climate change (Plants, People, Planet, 2025).}  
This paper tackles an urgent applied problem: current vine-planting choices and limited genotype$\times$environment data undermine adaptation planning under climate change. I synthesized fragmented genotype$\times$environment datasets into a proposed meta-analysis framework and designed a global data-gap map that reframed the paper’s core argument: predictive models must explicitly account for missing data to guide adaptation. Beyond figure preparation, I contributed to shaping the conceptual structure of the manuscript, ensuring that the analyses were presented in a way accessible to both ecologists and practitioners. My work provided the methodological backbone that made the study’s recommendations actionable for viticulture planning under climate change.\\
\textbf{2. Public communication on Québec wildfires (Le Point Bio, 2024).}  
I wrote and co-produced a public-facing article that explained the causes and health impacts of the 2023 Québec wildfires shortly after the events. Drawing on my boreal-forest ecology background, I led expert interviews and framed complex findings in an accessible jargon. This work reflects my commitment to turning research into actionable public understanding and outreach, and helped me practice rapid, accurate science communication in a high-stakes context.

%%%%%%%%%%%%%%%%%%
\subsection*{III --- Applicant’s statement}
%%%%%%%%%%%%%%%%%%

\textbf{Research experience}: 

The first time I really fell into natural-science research was during the summer of 2021 with Dr. Pierre Drapeau. Over three months, my colleagues and I travelled across Québec investigating woodpecker damage to wooden electrical infrastructures---a project that taught me rigorous field protocols, database management and the basics of scientific coding.  In 2022, I held my first Undergraduate Student Research Award (USRA) and carried out intensive fieldwork on Pileated woodpecker habitat in remote boreal forest: long days of banding, nest-searching and tree-classification built my field skillset. Alongside this, I leveraged 20 years of nesting data to understand how Pileated woodpecker nesting phenology is affected by climate change---my first independent statistical project---which gave me hands-on experience in data cleaning, model selection and interpretation of long-term ecological signals. A second USRA at UBC in 2023 (supervised by Drs. Frederik Baumgarten and E.M. Wolkovich) exposed me to large-scale experiments. During the summer, I designed and implemented a defoliation experiment to test tree resilience, maintained a 1,000-tree trial, deployed dendrometers and analyzed growth responses to treatments. These activities allowed me to frame research questions about climate–growth interactions, refine experimental protocols, and collect data for better predictive models of tree responses under climate change.

In 2024, I conceived and led a greenhouse experiment (foundation of my first MSc chapter), where I formulated hypotheses on plant-climate interactions, designed treatments to test them, and integrated climate and phenological data into statistical models. Alongside this, I also joined field campaigns in Smithers (BC) and Mount Rainier (WA) to study forest regeneration dynamics, where I improved my forestry methods and learned the basics of dendrochronology. In 2025, while continuing my greenhouse experiment---which will later become my first PhD chapter---I visited the Arnold Arboretum of Harvard University to collect tree cores, process cross-sections from a common garden trial, and analyze tree rings. These activities completed the data collection for my second MSc chapter and deepened my expertise in dendrochronological methods. Collectively, these experiences have equipped me with strong skills in experimental design, tree-ring analysis, intensive field logistics, and quantitative methods, which I continue to strengthen through workshops and applied projects. \\

\textbf{Relevant activities and distinctions}:  

Prior to my current academic path, I earned degrees in the restaurant and wine industries and spent nearly ten years working in the field, where I built a strong work ethic and sense of rigour. The transition from the food and beverage industry to academia happened while working for six months in a vineyard in New Zealand; I realized that my ties to nature were stronger than I expected. Thereafter, I enrolled in a Bachelor's degree in biology, majoring in ecology. As an undergraduate, I applied the work ethic I learned in hospitality to my studies and research. In 2025, during the first year of my MSc degree, I deepened my quantitative skills by taking a biomathematics class, attending a three-week workshop on hierarchical Bayesian models and following the Statistical Rethinking course. I also present and refine my models in a biweekly Bayesian working group where peers comment and suggest improvements.

My dedication and constant efforts during my undergraduate studies at Université du Québec à Montréal were recognized with the Governor General's Academic Silver Medal, Dean's List of honours, 13 competitive scholarships totalling \$76,750, and a perfect GPA of 4.30/4.30. In addition to my excellent academic record, I have always maintained active outreach: I was interviewed at the 2023 UQÀM scholarship ceremony, featured on Biochambers for my growth-chamber work, presented to Treespotters citizen scientists at Harvard, and hosted an open house at the Canadian Society for Ecology and Evolution (2024).

Across the two years leading my experiment, I developed strong leadership qualities as I trained and supervised 12 undergraduate students in growth-chamber protocols, phenological monitoring and greenhouse management, two of whom are currently pursuing graduate studies. I also led the team for Smithers fieldwork in 2025. In addition to my mentor role in my laboratory, I was a teaching assistant for FRST303 (course enrollment: 1962) and FRST304 (course enrollment: 1216), during which I ran office hours and gave supplementary materials to help students grasp ecological and plant physiology concepts. Outside my lab work, since 2021, I have mentored twelve international students during their first months in Canada, tutored seven peers in biochemistry and statistics, and volunteered with BUDR to advise students on applying to graduate school ---often drawing on my own non-linear path. 
My background in restaurant management was valuable for my greenhouse experiment, as I already had technical and organizational knowledge for successful projects. However, this unique context posed certain challenges related to working with living organisms (e.g. pest management, irrigation control, climate data collection), which I overcame by building a strong network of professors and technicians who could provide me with resources to overcome obstacles.
My wildlife photography work complements my outreach: my photos have appeared on the cover of \textit{Le Point Bio (2024)}, in the Biodiversity Research Center (UBC) annual competition, in a recent 
Proceedings of the National Academy of Sciences (PNAS) article, and in \textit{Le Rappel}.


\end{document}
