
\documentclass[12pt]{article}
%\renewcommand{\baselinestretch}{1.5}
\usepackage{sectsty,setspace}
\usepackage[top=1.87cm, bottom=1.87cm, left=1.87cm, right=1.87cm]{geometry} 
\usepackage{epstopdf} % for pdf creation
\usepackage{amsmath,latexsym,amssymb,wasysym} % improving structure
\usepackage{natbib}
%\usepackage{hyperref} % to get the little green boxes around the refs
\usepackage{times} % setting font to times
\usepackage{fancyhdr} % for custom headers/footers

\setstretch{1}
%\setlength{\baselineskip}{0.167in} 
\setlength\parindent{0pt} % no indents throughout

%%%%%%%%%%%%%%%%%%%%%%%%%%%%%%
% NSERC's formatting
%%%%%%%%%%%%%%%%%%%%%%%%%%%%%%
% Setup header
\begin{document}
\pagestyle{fancy}
\fancyhf{} % clear header/footer
\fancyhead[R]{Christophe Rouleau-Desrochers} % name on right side
\fancyfoot[C]{\thepage}
\renewcommand{\headrulewidth}{0pt}


%%%%%%%%%%%%%%%%%%%%%%%%%%%%%%

%%%%%%%%%%%%%%%%%%%%%%%%%%%%%%
% Regular formatting
%%%%%%%%%%%%%%%%%%%%%%%%%%%%%%
%\title{Improving forest carbon models under climate change using drones and experiments}
%\date{\today}
%\author{Christophe Rouleau-Desrochers}
%\begin{document}
%%%%%%%%%%%%%%%%%%%%%%%%%%%%%%

\section*{Improving Forest Carbon Models under Climate Change Using Drones and Experiments}
%Provide a detailed yet concise description of your proposed research project for the period during which you are to hold the award. Be as specific as possible. Provide background information to position your proposed research within the context of the current knowledge in the field. State the significance of the proposed research to a field or fields in the NSE. State the objectives and hypothesis and outline the experimental or theoretical approach to be taken (citing literature pertinent to the proposal) and the methods and procedures to be used.
%If the proposed research is a continuation of your thesis, clearly state the differences between work done for your thesis and the research activities outlined in this proposal.
%<><><><><><><><><><><><><><><><><><><><>
% CONTEXT %
%<><><><><><><><><><><><><><><><><><><><>
\textbf{Context:} There is increasing evidence that anthropogenic climate change, and particularly increased temperature, affects many natural systems at the global scale \cite{parmesan_poleward_1999,rosenzweig_attributing_2008}. The most frequently observed biological impact of climate change over the past decades are major changes on spring and autumn phenology --- the timing of recurring life history events \cite{parmesan_globally_2003,cleland_shifting_2007,lieth_phenology_1974,woolway_phenological_2021,menzel_european_2006}. Understanding the consequences of these shifts on ecosystems requires understanding how much the growing season has changed \cite{duputie_phenological_2015}. Spring phenological events (e.g. budburst and leafout) have been advancing from 0.5 \cite{wolfe_climate_2005} to 4.2 days/decade \cite{chmielewski_response_2001,fu_recent_2014} and are mainly driven by temperature \cite{chuine_why_2010,cleland_shifting_2007,penuelas_responses_2001}. In contrast, autumn phenology (e.g. budset and leaf colouring) is delayed, though to a much lesser extent than spring \cite{gallinat_autumn_2015,jeong_macroscale_2014}, and is driven by shortening photoperiod \cite{cooke_dynamic_2012,flynn_temperature_2018,korner_phenology_2010} and colder temperatures \cite{cooke_dynamic_2012,delpierre_temperate_2016}. These shifts support a long-lasting and intuitive assumption that earlier spring and delayed autumn events lead to longer seasons---and thus increased growth \cite{keenan_net_2014}. However, research from the past three years has cast doubt on this hypothesis \cite{dow_warm_2022,green_limits_2022,silvestro_longer_2023}. Recently, Dow \textit{et al}. (2022) showed that despite an earlier growth onset, neither growth rate nor overall annual increment was increased by longer seasons. This could substantially affect carbon-cycle model projections and thus feedbacks to future climate \cite{richardson_climate_2013,swidrak_comparing_2013}. 
Understanding these findings requires answering why trees do not grow more despite longer growing seasons. I hypothesize two possible drivers explaining this phenomenon: external (environmental) \cite{kolar_response_2016} or internal (via physiological constraints)\cite{zohner_effect_2023} limits to growth. The complex nature of climate change makes predicting the external drivers to growth hard to quantify at the individual level, as these drivers affect communities as a whole. Drought, spring frost and heat waves are commonly mentioned as the main extreme events that could limit tree growth under climate change \cite{tyree_xylem_2002, choat_triggers_2018, li_widespread_2023,trenberth_global_2014,intergovernmental_panel_on_climate_change_detection_2014,chiang_evidence_2021,polgar_leafout_2011,reinmann_compensatory_2023}. To better understand these mechanisms, experiments are paramount to robustly tease apart the external vs internal drivers (e.g. warmer springs from severe drought later in the season---a common co-occurring reality in natural environments) \cite{morin_changes_2010,primack_observations_2015}. This is essential to refine forest carbon sequestration projections \cite{green_limits_2022,cabon_cross-biome_2022}. However, experiments are most often performed on juvenile trees, which are critical for their role in forest regeneration projections, but their responses can hardly be translated to mature trees, which hold the overwhelming carbon biomass proportion of forests \cite{augspurger_differences_2003,silvestro_longer_2023,vitasse_ontogenic_2013}. To investigate how growing season shifts impact mature trees in their natural environments, unmanned aerial vehicle (UAV) imagery paired with machine learning has the capacity to acquire huge sample sizes, beyond what is achievable via traditional observational ground work, and at a better spatial and temporal resolution than satellites \cite{berra_assessing_2019,piao_plant_2019,teng_bringing_2025}. Thus, I propose to use a combination of two experiments to test internal (Chapter 1) and external (Chapter 2) limits to growth along with a large-scale mixed-forest observational data project using UAV imagery and machine learning (Chapter 3). This will allow me to address the paradox of the absence of increased growth despite apparently improved growing season conditions. 

%<><><><><><><><><><><><><><><><><><><><>
% CHAPTER 1 %
%<><><><><><><><><><><><><><><><><><><><>
\textbf{Chapter 1. Extended growing season experiment---internal drivers (MSc thesis continuation)}:
Shifts in phenological phases have consequences on growth during the current growing season---hypothesis previously tested in my thesis---but experiments to date have not quantified if longer seasons have lagging effects over the following years \cite{chapin_ecology_1990,landhausser_partitioning_2012,lawrence_variable_2018,martens_first-year_2007,schott_premature_2013}. Therefore, my objective here will be to expand my Master's thesis work by analyzing 2025 data and extend the project for a third consecutive year (2026) in order assess different tree species’ potential to stretch their activity schedules and determine whether or not this translates into increased growth over multiple growing seasons. For the first year of my proposed award tenure, I will continue to monitor phenology and growth, but I will also collect tree cross-sections at the end of the third growing season. Following up on an ongoing collaboration with ETH Zurich, I will perform cellular scans of the tree cross-sections at WSL (ETH Zurich). These valuable tree-ring data will allow me to understand how treatments affect cell count and morphology for each growing season \cite{silvestro_longer_2023}. 

%<><><><><><><><><><><><><><><><><><><><>
% CHAPTER 2 %
%<><><><><><><><><><><><><><><><><><><><>
\textbf{Chapter 2. Drought and spring frost experiment---external drivers}:
With climate change, not only will growing season length shift, but trees will also experience shifts in the timing of moisture deficits leading to drought stress, and increased late-spring frosts \cite{dox_wood_2022}. Tree-ring research shows that both droughts and frosts can result in important tissue loss \cite{kramer_why_2012,baumgarten_no_2023,dandrea_winters_2019}. However, it is unclear whether trees exposed to droughts and spring frosts also grow less \cite{chamberlain_late_2021,baumgarten_no_2023,lian_summer_2020,zhang_drought_2021}.
Here, my objective will be to investigate how these two abiotic drivers affect tree growth. For this, I will conduct an experiment during the second year of the award tenure that consists of three drought treatments and an additional two spring frost treatments. I will use 15 replicates of 12 deciduous North American tree species (six congeneric pairs to avoid potential confounding effects of shared evolutionary history), spanning different life history strategies, for all five treatments and a control, summing a total of 1080 individuals (a sample size consistent with my successful thesis experiment already well underway). 
For the drought treatments, I will move the trees to growth chambers at a warmer temperature and lower air humidity than ambient conditions to maximize evapotranspiration rates until they reach their respective wilting point (values at which soil water is not extractable by the plant). Then the trees will be moved back to ambient conditions and at constant irrigation. The three drought treatments will differ in their timing of occurrence to test the importance of drought timing. Thus, the first treatment will be conducted just after leaf-out; the second, one week before the solstice---period of peak growth for a lot of species \cite{anderson-teixeira_carbon_2021,dorangeville_drought_2018,mcmahon_general_2015}; the last will happen near the end of the season, just before growth cessation. For spring frost treatments, I will place the trees in growth chambers early in the season at warm temperatures to trigger budburst. When the trees start to burst, I will place the first treatment for one hour in freezing growth chambers. For the second spring frost treatment, I will wait for the leaves to be fully elongated and then place the trees under the same freezing conditions as the first treatment \cite{zohner_increased_2018}. Both treatments will follow successful methods in Chamberlain \textit{et al}. (2021). For all treatments, including the control, I will monitor phenology throughout the growing season, estimate biomass at the start and end of the growing season---using allometric equations, and I will equip a subset of trees with magnetic dendrometers that will provide valuable insight into the exact timing of changing growth in response to treatments. 

%<><><><><><><><><><><><><><><><><><><><>
% CHAPTER 3 %
%<><><><><><><><><><><><><><><><><><><><>
\textbf{Chapter 3. Growth timing$\times$phenology drone imagery}:
An improved understanding of the differences in growth synchronicity with leaf phenology across species is critical in forest ecology, yet this relationship has received little attention \cite{klein_coordination_2016,kramer_importance_2000}. Thus, for the three years of my award tenure, I will launch a large-scale project using cutting-edge UAVs equipped with multispectral sensors$\times$artificial intelligence technologies \cite{ball_accurate_2023,teng_bringing_2025,ulku_deep_2022} to gather a large amount of data on tree growth onset and end from a mixed-forest community. My objectives are to increase the accuracy of the data used by carbon cycle models---paramount in the context of rapid climate change \cite{richardson_climate_2013,swidrak_comparing_2013}--- and improve the field of phenology remote sensing by challenging novel methods and building on them. The work will take place at a research station in St-Hypollyte (Qc) for three consecutive growing seasons. Using this site will allow me to follow up on work previously done by my laboratory \cite{flynn_temperature_2018} as well as create a partnership with Dr. Etienne Laliberté, whose laboratory currently uses this site for their research \cite{cloutier_influence_2024}. I will perform high-frequency UAV overflights to capture leaf phenology of every canopy tree throughout the growing season. Then I will use BalSAM, a promising model to accurately and efficiently segment tree crowns from repeated UAV images \cite{teng_bringing_2025}. This will allow me to gather a large amount of highly precise data from single trees within the forest community \cite{teng_bringing_2025}. With this data, I will be able to infer the start and end of the growing season for each species and individual within this forest community \cite{berra_assessing_2019,fawcett_monitoring_2021}. Then, I will deploy 50 dendrometers distributed across the site to record stem diameter changes on canopy trees. Using high-resolution data across space and time will allow me to robustly infer a relationship between leaf phenology and growth seasonality.\\
%<><><><><><><><><><><><><><><><><><><><>
% OUTREACH %
%<><><><><><><><><><><><><><><><><><><><>
\textbf{Outreach:}
Given the widespread impacts of climate change on ecosystems, understanding how forest communities respond to prolonged growing seasons is crucial. Observing the responses of deciduous tree species to extended seasons may reveal potential benefits for some species and harm for others. These shifts are likely to influence forest dynamics across North America, with potential feedback with future climate change. Therefore, using two experiments and a large-scale remote sensing project, I aim to understand how the growth dynamics of North American tree species will change with longer growing seasons.  

%<><><><><><><><><><><><><><><><><><><><>
% REFERENCES
%<><><><><><><><><><><><><><><><><><><><>
\newpage

\bibliography{bibproposed.bib}
\bibliographystyle{nature} % set citation style 

\end{document}
