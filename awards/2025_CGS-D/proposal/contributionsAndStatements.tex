

\documentclass{article}
\usepackage[utf8]{inputenc}
\usepackage{authblk}
\usepackage{setspace}
\usepackage{natbib}
\usepackage{hyperref}
%\usepackage{cite}
\usepackage[margin=1in]{geometry}
\usepackage{array}
\usepackage{graphicx}
\usepackage{caption}
\graphicspath{ {./figures/} }
\usepackage{subcaption}
\usepackage{amsmath}
\usepackage{lineno}
\usepackage{soul}
\usepackage{xcolor}
\sethlcolor{yellow}
\linenumbers

%%%%%%%%%%%%%%%%%
\title{Contributions and Statements}
\date{\today}
\author{Christophe Rouleau-Desrochers}
\begin{document}
%%%%%%%%%%%%%%%%%%%%%%%%%%%%%%

\maketitle

% See here for details: https://www.nserc-crsng.gc.ca/OnlineServices-ServicesEnLigne/instructions/201/pgs-pdf_eng.asp#a10
\section*{Part I – Contributions to research and development}
\par
\textbf{a. Articles published or accepted in peer-reviewed journals} \\
Wolkovich, E.M., \textbf{Rouleau-Desrochers, C.}, Garcia de Cortazar-Atauri, I., Walker, M. A., Lacombe,
T. 2025. Building a more predictive model of Terroir for the Anthropocene. \textit{Plants, People, Planet}: https://doi.org/10.1002/ppp3.70055 \\
\par
\textbf{b. Other peer-reviewed contributions (for example, communications, papers in peer-reviewed conference proceedings, posters)—do not repeatedly list the same proceeding from multiple conferences, proceedings for future conferences or your thesis} \\
\par
\textbf{c. Non-peer-reviewed contributions (for example, specialized publications, technical reports, preprints, conference presentations, posters, articles submitted) \\
Technology transfer} \\ 
\textbf{Rouleau-Desrochers, C.}, Patry, É., Gingras, G., Ucler, J.,Vézina-Doré, M., Paillé, O., Moghaddam, Y. Y. 2024. Un été parti en fumée. \textit{Le Point Bio}
\par
\textbf{d. Contributions resulting from your participation in industrially relevant R\&D activities} \\ 
\par
\textbf{e. Awarded and submitted patents and copyrights (for example, software, but not publications)}


\section*{Part II – Most significant contributions to research and development}
% here I need to descrive the most significant contribution to the articles
The article published in \textit{Le Point Bio} helped increasing our understanding of the causes and mechanisms underlying the huge wildfires that deforested large regions of the province of Quebec in 2023. For this, my colleagues and I interviewed experts in the field to get a unique and fresh perpective of the fires that just happened when we started the project. I believe my contribution was essential because I could bring my own knowledge and experience with boreal ecosystems which led me to prepare and ask the questions to the researchers. This article, in addition, was not written to be published in the scientific litterautre and be peer-reviewed. It was written to be as widely accessible as possible to break the boundaries between academia and its scientific jargon and the public. Along with me being the first author of this article, my photo was also selected to be on the cover of the magazine. \\

As for the article published in \textit{Plants, People, Planet}, my contribution was to help the readers' understanding of the article concept by implementing a meta-analysis figure and a map in order to convey the message better. The article highlights the problem the to best use grapevine diversity, growers need to better understand how the environment interacts with different grapevine varieties. Increasing genetic diversity in vineyards is of critical importance in the context of rapid climate change because the climate of some regions will shift towards the climate of others. Hence, my work as a co-author on this paper was to both convey the message that increasing the quality and quantity of data is required to better predict the future of vineyards. I also designed a map which demonstrates the huge imbalance of genotype X environment study which are essential to predictive models. I also edited the manuscript at each step of peer-review revision process. The article was published in \textit{Plants, People, Planet} because of its rigorous and non-predatory peer-review process, along with the fact that it is a not-for-profit organization that promotes global plant-focused research and its impacts on humans which was perfect for the article's subject. My collaboration was mostly through the first author, E.M. Wolkovich which was through interacting and thinking through manuscript revision and figure design. 

\section*{Part III – Applicant’s statement}
% Describe your professional, academic and extracurricular activities, interactions, and collaborations that best demonstrate your relevant experience and achievements obtained within and beyond academia. Examples of these include:

\subsection*{Research experience}
% 2020
My scientific abilities really kick-started when I worked in a vineyard for 6 months in New Zealand. During that time working in the fields, I developped my passion for plants and this is when I understood that my interests grew deeper than simply letting the plants grow. I wanted to know why they grow in a certain way, the difference between the varieties and how they are influenced by the environment. This led me, some years later, to work on the manuscript described above. 

%2021
The first time I ever encountered research was during the summer of 2021. My the help of my colleagues, I travelled for the whole summer in Quebec, inspecting wooden electric posts to look for woodpeckers' damage on them. This has huge financial and logistical impacts on Hydro Quebec, the electricity provider of Quebec. This gave me a set of invaluable tools, such as rigourous data collection, database use and management and coding expertise I am still using in my research today.

%2022
During the summe of 2022, I worked as a USRA research assistant working again on woodpeckers, but most specifically on Pileated woopecker's vital domain in a remote region of Quebec's boreal forest. This 3-month fieldwork-intensive experience, led me to develop strong field capabilities, bird capture knowledge and an excellent training in forest tree categorization. Along with this, I received training for Forest work, outlining the safety procedures to take when working in remote hard-to-access locations. 

%2023
My second USRA tennure was held at the University of British Columbia under the supervision of Dr. Frederik Baumgarten and Dr. E.M. Wolkovich. During that summer, I developped deep knowledge of setting up an large-scale experiment using 1200 trees, setting up experimental conditions using growth chambers and cutting-edge magnetic micro-dendrometer to monitor the primary growth of tree saplings under experimental conditions. More specifically, I was reponsible of the defoliation treatments, aiming to understand how pest defoliation at different timing during the growing season affects tree growth. I was offered by Dr. Baumgarten to be co-author on this large-scale experiment, a project we are currently working on. In addition to this, I was also offered to lead a follow-up experiment, which was to occur in 2024-2025.

%2024
My third and last summer I held a USRA was to lead my own experiment that will later become my main Master's chapter. During that year, I designed and set-up a large-scale experiment of 630 trees with the aim of understanding how longer growing seasons affect tree growth across different species. This also gave me extremely valuable knowledge and experience in team management as I had lead five people who helped me conducting my experiment. During that summer, I also participated pursued field for a long-term project of forest recruitment in Smithers (BC) and on a project of seed masting in Mount Rainier National Park (WA, USA). The former increased my tree id skills and trained me into specialized techniques of long-term seedling growth monitoring. For the later, this project led me to gain valuable knowledge in dendrochronology techniques, fieldwork experience and launching over 400 dendrometers across multiple forest stands. 

%2025
In 2025, I really started digging deeper into the statistical world of Bayesian statistics by taking a workshop on hierarchical models, along with training myself with the course given by the textbook \textit{Statistical Rethinking}. Along with this, I worked on sample tree cross-section processing, scanning and tree ring measurements that will form my second chapter. I also visited Boston (MA, USA) to collect tree cores at the Arnold Arboretum of Harvard University from trees that were closely followed for 10 years by a citizen science program called the Treespotters during which they monitored phenological phases of 50 trees. During that summer, I also led the fieldwork at Smithers (BC), training a new undergraduate researcher in the lab on the methods used in forestry to understand forest regeneration. Parralel to all of this, I trained new undergraduate researchers on methods used in experiments such as growth chamber experiment, phenological monitoring and greenhouse management. 

\subsection*{Relevant activities – CGRS D}

\begin{itemize}
    \item Relevant training, such as: Academic training, Lived experience, Traditional teachings

    \item Scholarships, awards and distinctions

    \item Academic record

    \item Professional, academic and extracurricular activities, as well as collaborations with supervisors, colleagues, peers, students and members of the community, such as:
    \begin{itemize}
        \item Teaching, mentoring, supervising and/or coaching
        \item Managing projects
        \item Participating in science and/or research promotion
        \item Participating in community outreach, volunteer work and/or civic engagement
        \item Chairing committees and/or organizing conferences and meetings
        \item Participating in departmental or institutional organizations, associations, societies and/or clubs
    \end{itemize}
\end{itemize}


%\section {References}
%\bibliography{Exported_Items.bib}
%\bibliographystyle{nature} % set citation style 



\end{document}
