

\documentclass{article}
\usepackage[utf8]{inputenc}
\usepackage{authblk}
\usepackage{setspace}
\usepackage{natbib}
\usepackage{hyperref}
%\usepackage{cite}
\usepackage[margin=1in]{geometry}
\usepackage{array}
\usepackage{graphicx}
\usepackage{caption}
\graphicspath{ {./figures/} }
\usepackage{subcaption}
\usepackage{amsmath}
\usepackage{lineno}
\usepackage{soul}
\usepackage{xcolor}
\sethlcolor{yellow}
\linenumbers

%%%%%%%%%%%%%%%%%
\title{Contributions and Statements}
\date{\today}
\author{Christophe Rouleau-Desrochers}
\begin{document}
%%%%%%%%%%%%%%%%%%%%%%%%%%%%%%

\maketitle

% See here for details: https://www.nserc-crsng.gc.ca/OnlineServices-ServicesEnLigne/instructions/201/pgs-pdf_eng.asp#a10

% role in these articles should be 1-2 sentences
\section*{Part I – Contributions to research and development}
\subsection*{a. Articles published or accepted in peer-reviewed journals} 
Wolkovich, E.M., \textbf{Rouleau-Desrochers, C.}, Garcia de Cortazar-Atauri, I., Walker, M. A., Lacombe,
T. (2025) Building a more predictive model of Terroir for the Anthropocene. \textit{Plants, People, Planet}: https://doi.org/10.1002/ppp3.70055 \\
The work was not done in process of completing my undergraduate degree, but was part of my second USRA work. My role as a co-author was with the figure creations and manuscript editing. \\
\textbf{b. Other peer-reviewed contributions (for example, communications, papers in peer-reviewed conference proceedings, posters)—do not repeatedly list the same proceeding from multiple conferences, proceedings for future conferences or your thesis} \\
\textbf{c. Non-peer-reviewed contributions (for example, specialized publications, technical reports, preprints, conference presentations, posters, articles submitted)} \\
\textbf{Rouleau-Desrochers, C.}, Patry, É., Gingras, G., Ucler, J.,Vézina-Doré, M., Paillé, O., Moghaddam, Y. Y. 2024. Un été parti en fumée. \textit{Le Point Bio} \\
The work was done in process of completing my undergraduate degree. My role as a the first author was to lead the article writting and conduct the interviews along with my peers.\\
\textbf{d. Technology transfer} \\ 
\textbf{e. Contributions resulting from your participation in industrially relevant R\&D activities} \\ 
\textbf{f. Awarded and submitted patents and copyrights (for example, software, but not publications)}


\section*{Part II – Most significant contributions to research and development}
% here I need to descrive the most significant contribution to the articles
As for the article published in \textit{Plants, People, Planet}, my contribution was to implement the article's concepts into a meta-analysis figure and a map to convey the message better. The article highlights the problem that in order to maximize grapevine diversity's potential, growers need to better understand how the environment interacts with different grapevine varieties. Increasing genetic diversity in vineyards is critical in the context of rapid climate change because some regions will shift towards the climate of others. Hence, my work as a co-author on this paper was to help convey the message that increasing the quality and quantity of data is required to better predict the future of vineyards. I also designed a map which demonstrates the huge geographical imbalance of genotype X environment experiments which are paramount to build robust predictive models. I also edited the manuscript at each step of the peer-review revision process. The article was published in \textit{Plants, People, Planet} because of its rigorous and non-predatory peer-review process. The New Phytologist Foundation is a not-for-profit organization that promotes global plant-focused research and its impacts on humans, which was perfect for the article's subject. My collaboration was mostly through working alongside the first author, E.M. Wolkovich by interacting and creating figure design. 

The article published in \textit{Le Point Bio} gave the opportunity to bridge the gap before the scientific community and the public. We refined our understanding of the causes and mechanisms underlying the vast and severe wildfires that deforested large regions of the province of Quebec in 2023. For this, my colleagues and I interviewed experts in the field to get a unique and very fresh perspective on the fires that just happened a couple of weeks before the start of the project. I believe my contribution was essential because I could bring my expertise with boreal ecosystems which led me to prepare and ask the questions to the researchers. Along with me being the first author of this article, my photo was also selected to be on the cover of the magazine. \\

\section*{Part III – Applicant’s statement}
% Describe your professional, academic and extracurricular activities, interactions, and collaborations that best demonstrate your relevant experience and achievements obtained within and beyond academia. Examples of these include:

\subsection*{Research experience}
%2021
The first time I really got into natural science research was during the summer of 2021 under the supervision of Pierre Drapeau. Along with my colleagues, I travelled for three months various regions of Quebec, inspecting wooden electric posts to look for woodpeckers' damage on them. This has huge financial and logistical impacts on Hydro Quebec, Quebec's electricity provider. This gave me a set of invaluable tools, such as rigourous data collection methods, database use and management and coding expertise which I am still using in my research today.

%2022
The summer of 2022 marked the first time I received an Undergraduate Student Research Award (USRA) and during which I worked as a research assistant on a project studying Pileated woopecker's vital domain in a remote region of Quebec's boreal forest. This 3-month intensive fieldwork experience led me to develop strong field capabilities, bird capture knowledge and an excellent training in forest tree categorization. Along with this, I received training for Forest work, outlining the safety procedures to take when working in remote, hard-to-access locations. In parralel to this fieldwork and until the end of 2023, I also conducted a research project on Pileated woodpecker's nesting phenology, leveraging 20 years of nesting data and performing my first statistical analysis. It's with this project that my interest for phenology started which led me to my following research experience.

%2023
My second USRA holding was held at the University of British Columbia under the supervision of Dr. Frederik Baumgarten and Dr. E.M. Wolkovich. During that summer, I developed valuable experience and skills in setting up and maintenance of a large-scale experiment using 1200 trees. My responsabilities were to set up experimental conditions using growth chambers and cutting-edge magnetic micro-dendrometer to monitor the primary growth of tree saplings. In addition to this, I was responsible of the defoliation treatments, aiming to understand how pest defoliation at different timings during the growing season affects tree growth. I was offered by Dr. Baumgarten to be a co-author on this in-progress experiment. In addition to this, Dr. Wolkovich offered to lead a follow-up experiment, which was to occur in 2024-2025.

%2024
During the whole year of 2024, I lead my own experiment that would later become my main Master's chapter and this conincided in my third USRA holding. I ordered trees, designed a full-factorial experiment, set-up a large-scale experiment of 630 trees with the aim of understanding how longer growing seasons affect tree growth across different species. During that summer, I also participated in fieldwork for a long-term lab project of forest recruitment in Smithers (BC), as well as joining a team for four weeks to work on a project of seed masting in Mount Rainier National Park (WA, USA). The former increased my tree identification skills and allowed me to get trained into specialized techniques of long-term seedling growth monitoring. For the later, I gained valuable knowledge in dendrochronology techniques, fieldwork experience as well as technical experience related to the launch of 400 dendrometers across multiple forest stands. 

%2025
Along with this, I worked on tree cross-section processing, scanning and tree ring measurements that will form my second chapter. I also visited Boston (MA, USA) to collect tree cores at the Arnold Arboretum of Harvard University to complete data collection of my second chapter. During that summer, I also led the fieldwork at Smithers (BC), training am undergraduate researcher on the methods used in forestry to understand forest regeneration. In parralel to all of this ---for the second year of my greenhouse experiment--- I trained six undergraduate researchers on methods used in experiments such as growth chamber experiment, phenological monitoring and greenhouse management. 

\subsection*{Relevant activities – CGRS D}
% 2020

\begin{itemize}
    \item Relevant training, such as: Academic training, Lived experience, Traditional teachings
I was a teaching assistant providing students help with ecological and plant physiology theory under the supervision of Drs Sally Aitken and Suzanne Simard. Since 2021, I have developped a strong interest in helping students in need. Therefore, throughout my undergraduate degree, I mentored twelve international students with help and guidance during their first months of studies in Canada. I also worked as a part-time tutor where I helped seven students with their biochemistry and statistics classes. This interest remained and since I started my Master's degree, I am also volunteering to help undergraduate students succeed in their goals to go through graduate school.
In 2025, I really delve deeper into Bayesian statistics by taking a three-week workshop on hierarchical models, along with training myself with the \textit{Statistical Rethinking} course and attending and presenting my models at a biweekly working group on Bayesian models. 
    \item Scholarships, awards and distinctions
For my undergraduate grades at Université du Québec à Montréal, I received the Governor General's Academic Silver Medal (2 medals for roughly 8000 students and had the privilege of being included in the Dean's list of excellence. I also received 13 academic ---competitive--- excellence awards.
    \item Academic record
I graduated with a GPA of 4.30/4.30.
    \item Professional, academic and extracurricular activities, as well as collaborations with supervisors, colleagues, peers, students and members of the community, such as:
    \begin{itemize}
        \item Teaching, mentoring, supervising and/or coaching
         In addition to the rich technical and organizational knowledge I gained managing a large-scale greenhouse experiment for 2 years, I also developped leadership qualities as I trained and lead 12 undergraduate students who helped me conducting my experiment.
        \item Managing projects
        \item Participating in science and/or research promotion
        \item Participating in community outreach, volunteer work and/or civic engagement
        \item Chairing committees and/or organizing conferences and meetings
        \item Participating in departmental or institutional organizations, associations, societies and/or clubs
    \end{itemize}
    \item Other (maybe not includee here:
    My scientific interests and skills really kick-started when working in a vineyard for 6 months in New Zealand. My passion for plants developped while working in these fields. During this work, I realized that my interests grew deeper than the simple joy of harvesting the fruits off the vine. I wanted to understand why the vines grow in a certain way, the difference among the varieties and how they are influenced by the environment. This led me, some years later, to work on the manuscript described above. 

\end{itemize}


%\section {References}
%\bibliography{Exported_Items.bib}
%\bibliographystyle{nature} % set citation style 



\end{document}
