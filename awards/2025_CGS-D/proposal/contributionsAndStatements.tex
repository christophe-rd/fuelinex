

\documentclass{article}
\usepackage[utf8]{inputenc}
\usepackage{authblk}
\usepackage{setspace}
\usepackage{natbib}
\usepackage{hyperref}
%\usepackage{cite}
\usepackage[margin=1in]{geometry}
\usepackage{array}
\usepackage{graphicx}
\usepackage{caption}
\graphicspath{ {./figures/} }
\usepackage{subcaption}
\usepackage{amsmath}
\usepackage{lineno}
\usepackage{soul}
\usepackage{xcolor}
\sethlcolor{yellow}
\linenumbers

%%%%%%%%%%%%%%%%%
\title{Contributions and Statements}
\date{\today}
\author{Christophe Rouleau-Desrochers}
\begin{document}
%%%%%%%%%%%%%%%%%%%%%%%%%%%%%%

\maketitle

% See here for details: https://www.nserc-crsng.gc.ca/OnlineServices-ServicesEnLigne/instructions/201/pgs-pdf_eng.asp#a10
\section*{Part I – Contributions to research and development}
\par
\subsection*{a. Articles published or accepted in peer-reviewed journals} \\
Wolkovich, E.M., \textbf{Rouleau-Desrochers, C.}, Garcia de Cortazar-Atauri, I., Walker, M. A., Lacombe,
T. 2025. Building a more predictive model of Terroir for the Anthropocene. \textit{Plants, People, Planet}: https://doi.org/10.1002/ppp3.70055 \\
\par
\textbf{b. Other peer-reviewed contributions (for example, communications, papers in peer-reviewed conference proceedings, posters)—do not repeatedly list the same proceeding from multiple conferences, proceedings for future conferences or your thesis} \\
\par
\textbf{c. Non-peer-reviewed contributions (for example, specialized publications, technical reports, preprints, conference presentations, posters, articles submitted) \\
Technology transfer} \\ 
\textbf{Rouleau-Desrochers, C.}, Patry, É., Gingras, G., Ucler, J.,Vézina-Doré, M., Paillé, O., Moghaddam, Y. Y. 2024. Un été parti en fumée. \textit{Le Point Bio}
\par
\textbf{d. Contributions resulting from your participation in industrially relevant R\&D activities} \\ 
\par
\textbf{e. Awarded and submitted patents and copyrights (for example, software, but not publications)}


\section*{Part II – Most significant contributions to research and development}
% here I need to descrive the most significant contribution to the articles
The article published in \textit{Le Point Bio} gave the opportunity to bridge the gap before the scientific community and the public. We refined our understanding of the causes and mechanisms underlying the vast and severe wildfires that deforested large regions of the province of Quebec in 2023. For this, my colleagues and I interviewed experts in the field to get a unique and very fresh perspective on the fires that just happened a couple of weeks before the start of the project. I believe my contribution was essential because I could bring my own knowledge and experience with boreal ecosystems which led me to prepare and ask the questions to the researchers. Along with me being the first author of this article, my photo was also selected to be on the cover of the magazine. \\
As for the article published in \textit{Plants, People, Planet}, my contribution was to help the readers' understanding of the article's concept by implementing thr proposed suggestion into a meta-analysis figure and a map to convey the message better. The article highlights the problem that in order to maximise grapevine diversity's potential, growers need to better understand how the environment interacts with different grapevine varieties. Increasing genetic diversity in vineyards is critical in the context of rapid climate change because some regions will shift towards the climate of others. Hence, my work as a co-author on this paper was to help convey the message that increasing the quality and quantity of data is required to better predict the future of vineyards. I also designed a map which demonstrates the huge geographical imbalance of genotype X environment experiments which are paramount to build robust predictive models. I also edited the manuscript at each step of the peer-review revision process. The article was published in \textit{Plants, People, Planet} because of its rigorous and non-predatory peer-review process. The New Phytologist Foundation is a not-for-profit organization that promotes global plant-focused research and its impacts on humans, which was perfect for the article's subject. My collaboration was mostly through working alongside the first author, E.M. Wolkovich by interacting and creating figure design. 

\section*{Part III – Applicant’s statement}
% Describe your professional, academic and extracurricular activities, interactions, and collaborations that best demonstrate your relevant experience and achievements obtained within and beyond academia. Examples of these include:

\subsection*{Research experience}
% 2020
My scientific interests and skills really kick-started when working in a vineyard for 6 months in New Zealand. My passion for plants developped while working in these fields. During this work, I realized that my interests grew deeper than the simple joy of harvesting the fruits off the vine. I wanted to understand why the vines grow in a certain way, the difference among the varieties and how they are influenced by the environment. This led me, some years later, to work on the manuscript described above. 

%2021
The first time I really got into natural science research was during the summer of 2021 under the supervision of Pierre Drapeau. Along with my colleagues, I travelled for three months various regions of Quebec, inspecting wooden electric posts to look for woodpeckers' damage on them. This has huge financial and logistical impacts on Hydro Quebec, Quebec's electricity provider. This gave me a set of invaluable tools, such as rigourous data collection methods, database use and management and coding expertise which I am still using in my research today.

%2022
The summer of 2022 marked the first time I received an Undergraduate Student Research Award (USRA) and during which I worked as a research assistant on a project studying Pileated woopecker's vital domain in a remote region of Quebec's boreal forest. This 3-month intensive fieldwork experience led me to develop strong field capabilities, bird capture knowledge and an excellent training in forest tree categorization. Along with this, I received training for Forest work, outlining the safety procedures to take when working in remote, hard-to-access locations. In parralel to this fieldwork and until the end of 2023, I also conducted a research project on Pileated woodpecker's nesting phenology, leveraging 20 years of nesting data and performing my first statistical analysis. It's with this project that my interest for phenology started which led me to my following research experience.

%2023
My second USRA holding was held at the University of British Columbia under the supervision of Dr. Frederik Baumgarten and Dr. E.M. Wolkovich. During that summer, I developed valuable experience and skills in setting up and maintenance of a large-scale experiment using 1200 trees. My responsabilities were to set up experimental conditions using growth chambers and cutting-edge magnetic micro-dendrometer to monitor the primary growth of tree saplings. In addition to this, I was responsible of the defoliation treatments, aiming to understand how pest defoliation at different timings during the growing season affects tree growth. I was offered by Dr. Baumgarten to be a co-author on this in-progress experiment. In addition to this, Dr. Wolkovich offered to lead a follow-up experiment, which was to occur in 2024-2025.

%2024
My third and last summer I held a USRA was to lead my own experiment that will later become my main Master's chapter. During that year, I designed and set-up a large-scale experiment of 630 trees with the aim of understanding how longer growing seasons affect tree growth across different species. This also gave me extremely valuable knowledge and experience in team management as I had lead five people who helped me conducting my experiment. During that summer, I also participated pursued field for a long-term project of forest recruitment in Smithers (BC) and on a project of seed masting in Mount Rainier National Park (WA, USA). The former increased my tree id skills and trained me into specialized techniques of long-term seedling growth monitoring. For the later, this project led me to gain valuable knowledge in dendrochronology techniques, fieldwork experience and launching over 400 dendrometers across multiple forest stands. 

%2025
In 2025, I really started digging deeper into the statistical world of Bayesian statistics by taking a workshop on hierarchical models, along with training myself with the course given by the textbook \textit{Statistical Rethinking}. Along with this, I worked on sample tree cross-section processing, scanning and tree ring measurements that will form my second chapter. I also visited Boston (MA, USA) to collect tree cores at the Arnold Arboretum of Harvard University from trees that were closely followed for 10 years by a citizen science program called the Treespotters during which they monitored phenological phases of 50 trees. During that summer, I also led the fieldwork at Smithers (BC), training a new undergraduate researcher in the lab on the methods used in forestry to understand forest regeneration. Parralel to all of this, I trained new undergraduate researchers on methods used in experiments such as growth chamber experiment, phenological monitoring and greenhouse management. 

\subsection*{Relevant activities – CGRS D}

\begin{itemize}
    \item Relevant training, such as: Academic training, Lived experience, Traditional teachings

    \item Scholarships, awards and distinctions

    \item Academic record

    \item Professional, academic and extracurricular activities, as well as collaborations with supervisors, colleagues, peers, students and members of the community, such as:
    \begin{itemize}
        \item Teaching, mentoring, supervising and/or coaching
        \item Managing projects
        \item Participating in science and/or research promotion
        \item Participating in community outreach, volunteer work and/or civic engagement
        \item Chairing committees and/or organizing conferences and meetings
        \item Participating in departmental or institutional organizations, associations, societies and/or clubs
    \end{itemize}
\end{itemize}


%\section {References}
%\bibliography{Exported_Items.bib}
%\bibliographystyle{nature} % set citation style 



\end{document}
