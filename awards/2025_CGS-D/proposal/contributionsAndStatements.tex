


\documentclass[12pt]{article}
%\renewcommand{\baselinestretch}{1.5}
\usepackage{sectsty,setspace}
\usepackage[top=1.87cm, bottom=1.87cm, left=1.87cm, right=1.87cm]{geometry} 
\usepackage{epstopdf} % for pdf creation
\usepackage{amsmath,latexsym,amssymb,wasysym} % improving structure
\usepackage{times} % setting font to times
\usepackage{fancyhdr} % for custom headers/footers

\setstretch{1}
\setlength{\baselineskip}{0.167in} 
\setlength\parindent{0pt} % no indents throughout

%%%%%%%%%%%%%%%%%
% Setup header
\pagestyle{fancy}
\fancyhf{} % clear header/footer
\fancyhead[R]{Christophe Rouleau-Desrochers} % name on right side
\fancyfoot[C]{\thepage}
\renewcommand{\headrulewidth}{0pt}

%%%%%%%%%%%%%%%%%%%%%%%%%%%%%%
\begin{document}
%\section*{Contributions and Statements}


%%%%%%%%%%%%%%%%%
%\title{Contributions and Statements}
%\date{\today}
%\author{Christophe Rouleau-Desrochers}
%\begin{document}
%%%%%%%%%%%%%%%%%%%%%%%%%%%%%%

%\maketitle

% See here for details: https://www.nserc-crsng.gc.ca/OnlineServices-ServicesEnLigne/instructions/201/pgs-pdf_eng.asp#a10

% role in these articles should be 1-2 sentences

%%%%%%%%%%%%%%%%%%%%%%%%%%%%%%%%%
\subsection*{I --- Contributions to research and development}
%%%%%%%%%%%%%%%%%%%%%%%%%%%%%%%%%
\textbf{a. Articles published or accepted in peer-reviewed journals} \\
Wolkovich, E.M., \textbf{Rouleau-Desrochers, C.}, Garcia de Cortazar-Atauri, I., Walker, M. A., Lacombe,
T. (2025) Building a more predictive model of Terroir for the Anthropocene. \textit{Plants, People, Planet}: https://doi.org/10.1002/ppp3.70055 \\
The work was not done in the process of completing my undergraduate degree, but was part of my second USRA work. My role as a co-author was with the figure creation and manuscript editing. \\

\textbf{c. Non-peer-reviewed contributions (for example, specialized publications, technical reports, preprints, conference presentations, posters, articles submitted)} \\
\textbf{Rouleau-Desrochers, C.}, Patry, É., Gingras, G., Ucler, J.,Vézina-Doré, M., Paillé, O., Moghaddam, Y. Y. 2024. Un été parti en fumée. \textit{Le Point Bio} \\
The work was done in the process of completing my undergraduate degree. My role as the first author was to lead the article writing and conduct the interviews, along with my peers.

%%%%%%%%%%%%%%%%%%%%%%%%%%%%%%%%%%%%%%%
\subsection*{II --- Most significant contributions to research and development}
% here I need to descrive the most significant contribution to the articles
%%%%%%%%%%%%%%%%%%%%%%%%%%%%%%%%%%%%%%%
The article published in \textit{Plants, People, Planet}, highlights a crucial and urging problem in the winemaking industry related to the which winegrape varieties are currently used and where they are planted. In the context of climate change, wine growers must maximize grapevine diversity's potential, and for this, increased data quantity and quality is required to better understand how the environment interacts with different grapevine varieties. My role was to implement the article's concepts into a meta-analysis conceptual figure and a map to convey the article's main message. To improve how genetic diversity is used in a changing climate, we propose to use meta-analysis approach that has huge potential in refining our understanding of how the environment interacts with grape varieties. I also designed a map which demonstrates the urging need of increasing genotype X environment experiments in more diverse wine regions around the world. These data are paramount in building robust predictive model, such as the proposed meta-analysis approach. We published the article in \textit{Plants, People, Planet} because of its rigorous and non-predatory peer-review process. This not-for-profit journal promotes global plant-focused research and its impacts on humans, which was in line with the proposed article's subject. My collaboration was mostly through working alongside the first author, E.M. Wolkovich by creating different iterations of figures. \\

The article published in \textit{Le Point Bio} aimed to bridge the gap ---too rarely connected--- between the scientific community and the public. Our article offers a comprehensive and complete overview of the causes and mechanisms underlying the catastrophic wildfires that deforested large regions of the province of Quebec in 2023, but also their impacts on human health. For this, my colleagues and I interviewed experts in the field to get a unique and very fresh perspective on the fires that just happened a couple of weeks before the start of the project. My contribution was essential because I could bring my expertise with boreal forest ecosystems which led me to prepare and conduct the interviews with the researchers. Along with being the first author of the article, my photo was also selected to feature on the cover of the magazine. 

%%%%%%%%%%%%%%%%%%
\subsection*{III --- Applicant’s statement}
% Describe your professional, academic and extracurricular activities, interactions, and collaborations that best demonstrate your relevant experience and achievements obtained within and beyond academia. Examples of these include:
%%%%%%%%%%%%%%%%%%

\textbf{Research experience}: The first time I really got into natural science research was during the summer of 2021 under the supervision of Pierre Drapeau. Along with my colleagues, I travelled in various regions of Quebec for three months investigating the impact of woodpeckers' damage on wooden electric ---a situation with huge financial and logistical consequences on electricy network across the province. This first internship gave me a set of invaluable tools, such as rigourous data collection methods, database management and coding expertise which formed the base research expertise. The summer of 2022 marked the first time I received an Undergraduate Student Research Award (USRA) and during which I worked as a research assistant on a project studying Pileated woopecker's vital domain in a remote region of Quebec's boreal forest. This 3-month intensive fieldwork experience led me to develop strong field capabilities, bird capture experience and an excellent training in forest tree categorization. In parallel to this fieldwork and until the end of 2023, I also conducted a research project on Pileated woodpecker's nesting phenology, leveraging 20 years of nesting data and performing my first statistical analysis. I held my second USRA in 2022 at the UBC under the supervision of Dr. Frederik Baumgarten and Dr. E.M. Wolkovich. During that summer, I developed valuable skills in working on a large-scale experiment using 1000 trees. In addition to this, I conducted a smaller project on defoliation treatments and was responsible for setting up experimental conditions using growth chambers and working with dendrometers. I am currently working as a co-author alongside Dr. Baumgarten on this in-progress project. In 2024, for my third consecutive USRA holding, I lead my own experiment that would later become my main Master's chapter. During that summer, I also participated in fieldwork for a long-term lab project of forest recruitment in Smithers (BC), and joined a team during four weeks to work on a project of seed masting in Mount Rainier National Park (WA, USA). Both of these fieldwork experiences trained me in specialized techniques of long-term seedling growth monitoring, dendrochronology techniques, and dendrometers launch across multiple forest stands.  At the end of 2024, we decided to follow-up my experiment and increase the diversity of data to form my future first PhD chapter. In parallel to working for a second year on my greenhouse experiment, I visited the Arnold Arboretum of Harvard University to collect tree cores would complete data collection of my second Master's chapter. Both the cores and cross sections allowed me to gain valuable knowledge of dendrochronological methods.

\textbf{Relevant activities}: Prior to my academic journey, I pursued different degrees related to the restaurant management and wine industry. During these eight years, I developed a strong sense of rigour and work ethic. The transition from food and beverage to academia was caused by the realization that my ties to nature were stronger than I thought while working for six months a vineyard in New Zealand. Therefore, after this field experience, I registered in a Bachelor's degree in biology majoring in ecology. As an undergraduate, I was strongly driven to perform, by applying the work ethic I gained in the restaurant industry. During the first year of my Master's degree, I In 2025, I delve deeper into Bayesian statistics by taking a three-week workshop on hierarchical models, along with training myself by following the \textit{Statistical Rethinking} course. I also attended and presented my models at a biweekly working group on Bayesian models. 

My dedication and constant efforts during my undergraduate studies at Université du Québec à Montréal were recognized through several prestigious distinctions. I was awarded the Governor General's Academic Silver Medal --- granted to only two students out of roughly 8,000--- and earned a place on the Dean's List of Excellence. In addition to the awards listed in the \textit{Awards} section, I received six further academic and competitive excellence scholarships, for a total of \$76,750 during my undergraduate studies. Despite my unconventional path to academia, I am proud to have graduated with a perfect GPA of 4.30/4.30.

During the two years of my experiment, I developed strong leadership qualities as I trained and led 12 undergraduate students on growth chamber experiments, phenological monitoring and greenhouse management. During that summer of 2025, I also led the fieldwork at Smithers (BC) and trained an undergraduate researcher on the methods used in forestry to study forest regeneration dynamics. One of these students recently graduated and was successful in their graduate study application. In addition of my mentor role in my laboratory, I was a teaching assistant in FRST303 (1962 students) and FRST304 (1217 students), providing students with help with ecological and plant physiology theory under the supervision of Drs Sally Aitken and Suzanne Simard, respectively. 

My background in restaurant management was valuable for my greenhouse experiment as I already gained technical and organizational knowledge for successful projects. However, this unique context posed certain challenges related to working with living organisms (e.g. pest management, irrigation control, climate data collection), which I overcome by building a strong network of professors and technicians that could give me resources to overcome challenges.
        
My research work was the subject of a research outreach article published on Biochambers website. This company's lead scientist interviewed my colleague and I on the work conducted in our lab and how we were using growth chambers to answer our research questions. My second chapter leverages phenology data collected by a citizen science project called the Treespotters for 10 years. Thus, when my colleague and I went to core the trees in Boston (MA), I gave the citizen scientists a presentation on tree core sampling and outlined the scope of my project and how I will be using their data

Since 2021, I have developed a strong interest in helping students in need. Therefore, throughout my undergraduate degree, I mentored twelve international students with help and guidance during their first months of studies in Canada, and worked as a part-time tutor during which I helped seven students with their biochemistry and statistic classes. This interest remained and since the start of my Master's degree, I am volunteering to help undergraduate students succeed in their goals of pursuing graduate school through the BUDR program. I recently gave a presentation on my unsual path to graduate school and provided undegraduate students with advices on how to join a laboratory.

In relation to the first Cocodet award I received, I was interviewed during the 2023 UQÀM annual scholarship award ceremony, during which I was given the opportunity to talk about my future research goals and questions. As a member of the Canadian Society for Ecology and Evolution and during the 2024 annual conference, I hosted an open house featuring the methods and research goals of my greenhouse experiment. 

My artistic expression manifests itself through photography and more specifically on wildlife. Several of my photos were published in different medias: the cover of the 2024 edition of \textit{Le Point Bio}; the Biodiversity Research Center (BRC) annual photo competition; a recent article in the Proceeding of the National Academy of Science (PNAS); on a recent publication on non-for-profit organisation \textit{Le Rappel}



%\section {References}
%\bibliography{Exported_Items.bib}
%\bibliographystyle{nature} % set citation style 



\end{document}
