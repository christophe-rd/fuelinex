

\documentclass{article}
\usepackage[utf8]{inputenc}
\usepackage{authblk}
\usepackage{setspace}
\usepackage{natbib}
\usepackage{hyperref}
\usepackage[margin=1in]{geometry}
\usepackage{array}
\usepackage{graphicx}
\usepackage{caption}
\graphicspath{ {./figures/} }
\usepackage{subcaption}
\usepackage{amsmath}
\usepackage{lineno}
\usepackage{soul}
\usepackage{xcolor}
\sethlcolor{yellow}
\linenumbers

%%%%%%%%%%%%%%%%%
\title{Proposed Research Outline}
\date{\today}
\author{Christophe Rouleau-Desrochers}
\begin{document}
%%%%%%%%%%%%%%%%%%%%%%%%%%%%%%

\maketitle

%Provide a detailed yet concise description of your proposed research project for the period during which you are to hold the award. Be as specific as possible. Provide background information to position your proposed research within the context of the current knowledge in the field. State the significance of the proposed research to a field or fields in the NSE. State the objectives and hypothesis and outline the experimental or theoretical approach to be taken (citing literature pertinent to the proposal) and the methods and procedures to be used.
%If the proposed research is a continuation of your thesis, clearly state the differences between work done for your thesis and the research activities outlined in this proposal.


%<><><><><><><><><><><><><><><><><><><><>
% ABSTRACT %
%<><><><><><><><><><><><><><><><><><><><>
\section{Context}
\begin {enumerate}
	\item In temperate and boreal forests, temperature plays a crucial role in setting the physiological activity boundaries for trees (ref)
	\item Growing season extension in the past decades \citep{korner_phenology_2010,menzel_growing_1999}
	\item Trends of spring and autumn phenological events and their drivers \citep{walther_ecological_2002}
	\item Mechanisms that could limit growth despite having a longer growing season: heat waves, drought \citep{tyree_xylem_2002, choat_triggers_2018, li_widespread_2023, trenberth_global_2014,intergovernmental_panel_on_climate_change_detection_2014,chiang_evidence_2021}, spring frosts \citep{polgar_leafout_2011,reinmann_compensatory_2023}
	\item How these shifts translate into effects on trees/forests are not totally clear \citep{estiarte_alteration_2015,keenan_net_2014,gunderson_forest_2012,piao_plant_2019,dow_warm_2022}
	\item Global growing season shifts consequences on forest ecosystems and services (ref)
	\item I conducted an experiment on growing season extension and will leverage tree ring data and will conduct a follow-up experiment on environmental stressors on tree growth, will use high-resolution drone imagery and conduct a meta-analysis to test this hypothesis
	\item Significance of proposed research: importance of young trees on forest recruitment and increasing the precision of projections of carbon sequestration under future climate change (ref)
	\item** Question: which limits a tree to benefit from longer growing seasons? Internal limits or environmental stressors?
\end {enumerate}

%<><><><><><><><><><><><><><><><><><><><>
% CHAPTER 1 %
%<><><><><><><><><><><><><><><><><><><><>
\section*{Chapter 1: extended growing season experiment (Fuelinex) continuation}
\begin {enumerate}
	\item \textbf{Hypothesis:} Growing season extension modifies a tree’s capacity to sequestrate carbon and nitrogen, and this could lead to increased growth in the following season \citep{chapin_ecology_1990,landhausser_partitioning_2012,lawrence_variable_2018,martens_first-year_2007,schott_premature_2013}
	\item \textbf{Objectives:} Assess tree species’ potential to prolong or stretch their activity schedule and if tree ring can help answer this question
	\item \textbf{Methods:} to investigate the impact of manipulated spring and fall temperatures on phenological responses, I successfully conducted experiments in 2024 and 2025.
	\begin{itemize}
			\item Year 1 of award tenure
			\item 2-year experiment over 2024-2025 using 7 species
			\item Full factorial design of spring and fall warming with two levels each (control/warmed)= four treatments plus an additional two treatments to test fall nutrient effects, 15 replicates each for a total of 630 individual trees.
			\item Biomass increment assessment across the different treatments
			\item Tree ring analysis, including cell count and their characteristics, to infer a relationship between these criteria and the different treatments
		\end{itemize}
\end {enumerate}
%<><><><><><><><><><><><><><><><><><><><>
% CHAPTER 2 %
%<><><><><><><><><><><><><><><><><><><><>
\section*{Chapter 2: drought and spring frost experiment}
\begin {enumerate}
	\item \textbf{Hypothesis:} the timing of spring frosts and drought events\citep{babst_twentieth_2019,buermann_widespread_2018,dorangeville_drought_2018,helcoski_growing_2019,mcmahon_general_2015} will modify trees' growth response 
	\item \textbf{Objectives:} understand how the timing of spring and drought events affects trees' growth across 10 species. 
	\item \textbf{Methods:} Fuelinex explored the impact of extended seasons on growth, whereas this experiment will investigate how and if environmental stressors are responsible for the unexpected absence of growth response to longer seasons. 
		\begin {itemize}
			\item Year 2 and 3 of award tenure
			\item 10 species, 15 replicates/species
			\item Treatments: 3 drought (late spring, summer solstice and late summer, 2 spring frost (at budburst and after leaf out) and 1 control. 
			\item Micro-dendrometer to get high-resolution data on primary growth response of trees.
			\item Phenological monitoring 
			\item Shoot elongation for secondary growth response
			\item Biomass increment assessment across the different treatments
		\end{itemize}
\end {enumerate}

%<><><><><><><><><><><><><><><><><><><><>
% CHAPTER 3 %
%<><><><><><><><><><><><><><><><><><><><>
\section*{Chapter3 : cambial phenology X drone imagery phenological observations}
\begin {enumerate}
	\item \textbf{Hypothesis:} synchrony between growth onset and leaf phenological phases will vary across species and sites.
	\item \textbf{Objectives:} Understand how species-specific leaf and wood phenology relate using drone imagery and deep learning methods  
	\item \textbf{Methods:} Satellites lack spatial and temporal resolution, where UAVs equipped with RGB and multi-spectral sensors can get high resolution of mature stands. UAV imagery can estimate SOS and EOS \citep{berra_assessing_2019,fawcett_monitoring_2021}
		\begin{itemize}
			\item Year 1, 2 and 3 of award tenure
			\item High-resolution drone imagery will be used to monitor forest phenology across 2 growing seasons at the Station biologique des Laurentides (St-Hypollyte (Qc)), following previous work from my current lab at this location and collaboration with the Plant Functional Ecology Laboratory
			\item Seven deciduous tree species will be used 
			\item Soil temperature, water content will be continuously measured using soild moisture and temperature sensors
			\item 200 DC3 Perimeter Dendrometer to infer a relationship between growth seasonality and leaf phenology
		\end{itemize}
\end {enumerate}

\section{References}
\bibliography{ExportedItems.bib}
\bibliographystyle{pnas} % set citation style 

\end{document}
