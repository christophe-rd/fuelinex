

\documentclass{article}
\usepackage[utf8]{inputenc}
\usepackage{authblk}
\usepackage{setspace}
\usepackage{natbib}
\usepackage{hyperref}
%\usepackage{cite}
\usepackage[margin=1in]{geometry}
\usepackage{array}
\usepackage{graphicx}
\usepackage{caption}
\graphicspath{ {./figures/} }
\usepackage{subcaption}
\usepackage{amsmath}
\usepackage{lineno}
\usepackage{soul}
\usepackage{xcolor}
\sethlcolor{yellow}
\linenumbers

%%%%%%%%%%%%%%%%%
\title{Proposed Research Outline}
\date{\today}
\author{Christophe Rouleau-Desrochers}
\begin{document}
%%%%%%%%%%%%%%%%%%%%%%%%%%%%%%

\maketitle

%Provide a detailed yet concise description of your proposed research project for the period during which you are to hold the award. Be as specific as possible. Provide background information to position your proposed research within the context of the current knowledge in the field. State the significance of the proposed research to a field or fields in the NSE. State the objectives and hypothesis and outline the experimental or theoretical approach to be taken (citing literature pertinent to the proposal) and the methods and procedures to be used.
%If the proposed research is a continuation of your thesis, clearly state the differences between work done for your thesis and the research activities outlined in this proposal.


%<><><><><><><><><><><><><><><><><><><><>
% ABSTRACT %
%<><><><><><><><><><><><><><><><><><><><>
\section{Context}
\begin {enumerate}
	\item In temperate and boreal forests, temperature plays a crucial role in setting the physiological activity boundaries for trees.
	\item Growing season extension in the past decades
	\item Trends of spring and autumn phenological events and their drivers %\citep{walther_ecological_2002}
	\item Mechanisms that could limit growth despite having a longer growing season: heat waves, drought, spring frosts
	\item How these shifts translate into effects on trees/forests are not totally clear
	\item Global growing season shifts consequences on forest ecosystems and services
	\item I will conduct an experiment using saplings under experimental conditions, will use high-resolution drone imagery and conduct a meta-analysis to test this hypothesis
	\item Significance of proposed research: importance of young trees on forest recruitment and increase the precision of projections of carbon sequestration under future climate change. 
\end {enumerate}

%<><><><><><><><><><><><><><><><><><><><>
% CHAPTER 1 %
%<><><><><><><><><><><><><><><><><><><><>
\section*{Chapter 1: Experiment on drought and spring frosts on tree biomass}
\begin {enumerate}
	\item Year 1 and 2 of award tenure
	\item Hypothesis: the timing of spring frosts and drought events will modify trees' growth reponse 
	\item Objectives: understand how the timing of spring and drought events affect trees' growth accross 10 species. 
	\item 10 species, 15 replicates/species
	\item Treatments: 2 drought, 2 spring frost and 1 control. 
	\item Micro-dendrometer to get high-resolution data on primary growth response of trees.
	\item Phenological observations 
	\item Shoot elongation for secondary growth response
	\item Biomass
\end {enumerate}

%<><><><><><><><><><><><><><><><><><><><>
% CHAPTER 2 %
%<><><><><><><><><><><><><><><><><><><><>
\section*{Chapter: cambial phenology X drone imagery phenological observations}
\begin {enumerate}
	\item Year 2 and 3 of award tenure
	\item Hypothesis: synchrony between growth onset and leaf phenological phases will change across species and sites.
	\item Objectives: Understand how species-specific leaf and wood phenology relate using drone imagery and deep learning methods (Veras et al., 2022)
	\item Relevance of using high-resolution drone imagery to monitor forest community imagery (4 sites on the east coast following a latitudinal gradient) 
	\item Four dominant tree species within these 4 populations
	\item Microcores sampling using Trephor weekly for the whole growing season (Dox 2022, Rossi 2006)
	\item Soil moisture measurements and temperature sensors
\end {enumerate}

%<><><><><><><><><><><><><><><><><><><><>
% CHAPTER 3 %
%<><><><><><><><><><><><><><><><><><><><>
\section*{Chapter 3: meta-analyses on something or something else or explicitly mention that I will follow-up with Fuelinex...}
\begin {enumerate}
	\item should figure something out
\end {enumerate}

%\section {References}
%\bibliography{Exported_Items.bib}
%\bibliographystyle{nature} % set citation style 



\end{document}
