%------------------------------------------------------------------------------
\documentclass[11pt,letter]{article}
\usepackage[top=0.4in, bottom=0.7in, left=1in, right=1in]{geometry}
\renewcommand{\baselinestretch}{1}
\usepackage{graphicx}
\usepackage{natbib}
\usepackage{amsmath}
\usepackage{hyperref}
\parindent=0pt
\parskip=1 pt

%------------------------------------------------------------------------------
\title{Proposal}
\author{Christophe Rouleau-Desrochers}

\begin{document}
\pagenumbering{gobble}

\maketitle

%%%%%%%%%%%%%%%%%%%%%%%%%
% Lay Abstract %
%%%%%%%%%%%%%%%%%%%%%%%%%

%%%%%%%%%%%%%%%%%%%%%%%%%
% Substantive Eligibility %
%%%%%%%%%%%%%%%%%%%%%%%%%
% Please describe how your research matches the areas of interest specified for the award (max 1800 characters). This section must be filled out.
% 	sustainable approaches to and development of the general urban environment, including water, energy and transportation infrastructure in British Columbia;
% 	environmental protection of oceans, beaches and waterfronts that impact British Columbia; and
% 	sustainable approaches to resource-intensive industry in British Columbia.
\section *{Substantive Eligibility} 
In the context of rapid anthropogenic climate warming, we need to take rapid and radical actions to mitigate the future consequences of climate change. Summers are hotter, and extreme events such as heat domes and severe drought events are happening more often (Zhang23). These phenomena and their consequences on urban environments and forest ecosystems are widely reported (Allen10;McCarthy10). Indeed, cities are forefront victims of climate change, but they are also important sources of greenhouse gases (DAS2024;Corburn09). People living in cities are already experiencing increased extreme weather events (e.g. heat waves) (DAS2024). Cities such as Vancouver can mitigate the future impacts of climate change. For instance, urban trees provide cooling effects which can save lives during summer heat waves (Ettinger24;Zandler24). They also provide carbon sequestration, improve air quality, and increase human health and well-being (Moreno24). The latter is especially important because recent work has shown that humans exposed to trees are showing considerable mental and physical health benefits (WOLF20;Turner‐Skoff19). My work could provide information to choose trees that will be better suited to warmer and drier cities. 
\par
In regards to forest ecosystems, recent evidence suggests that wildfires will increase with ongoing climate change (Wasserman23). More frequent and intense wildfires will likely decrease BC old growth forest (PRICE21). However, we are aware of extremely effective methods to mitigate wildfire impacts on ecosystems. For example, Indigenous nations such as the Lytton First Nation, have successfully practiced prescribed burning to reduce fire intensity and severity for centuries (LEWIS18). This highlights the critical importance of considering Indigenous knowledge in our climate change response policies. Additionally, recent work shows the critical importance of forest species composition on fire intensity where ecosystems with a higher proportion of deciduous tree species usually have lower chances of burning. This brings another key aspect of my research, which is by increasing the use of deciduous trees in the forestry industry, we could reduce fire severity and provide better choices to increase forest productivity. %18march: 2000 characters


%%%%%%%%%%%%%%%%%%%%%%%%%
%  EDI & Indigeneity %
%%%%%%%%%%%%%%%%%%%%%%%%%.
% Are there equity or Indigeneity considerations for your research (yes/no)? Please explain (max 1800 characters). This section must be filled out.
\section *{EDI and Indigeneity}
Bridging the gap between the Western view of science and indigenous knowledge is not simple, but I believe, necessary. BC Indigenous communities have an extremely valuable understanding of the forest ecosystem functioning. They have lived in west-coast boreal forests for centuries and have successfully mitigated wildfires' severity in ways we have yet to implement in our western-viewed forest management. However, wildfires are among the many threats forests will face under future climate change and the importance of conserving their indigenous ancestral lands is critical. Climate change-induced pest infestation and summer drought are expected to severely affect BC forests. Modeling and experimental studies can guide decisions on tree species for reforestation, potentially incorporating assisted migration to enhance forest resilience. Assisted migration involves moving tree genotypes to areas where they are expected to thrive in the near future. I argue that preserving BC's old-growth forests and ancestral lands could benefit from greater knowledge of which species and genotypes will boost reforestation efforts, helping forests better withstand future environmental changes.
\par 
While I believe my research could help refine the way we do forestry in BC, I believe that understanding tree growth better could also benefit inpoverished urban areas. The positive impacts of green spaces in cities are obvious, however the setbacks are more subtle but can have tremendous impacts on people's lives. Historically, impoverished neighbourhoods in which trees were planted showed increased and rapid gentrification. Over time, this has led to a disparity in green cover between wealthy and poorer areas, resulting in segregated urban landscapes. Consequently, the people living in urban areas poor in tree density are the most affected by increased summer heat waves and their corresponding lethal toll. It is in this context that my research could provide increase knowledge on a more informative and precise selection of the best trees that should be planted in BC cities so everyone can be exposed to the numerous benefits that trees bring in cities. 

%%%%%%%%%%%%%%%%%%%%%%%%%
%  Research Statement % 2 pages
%%%%%%%%%%%%%%%%%%%%%%%%%.
    % Define and clearly describe the research objectives;
\section {Research Statement}
In temperate and boreal forests, temperature plays a crucial role in setting the boundaries for seasonal physiological activity. Thus, with rising temperatures from anthropogenic climate change, the climatically possible growing season has lengthened in many ecosystems worldwide by up to 11 days. Plants have tracked this through shifts in phenology—the study of recurring life history events—which are expected to continue with increasing temperatures. In particular, trees have shifted earlier in the spring and later in the fall. They may use these extra days to fix more carbon and increase growth during the current growing season. This is a long-lasting assumption that was supported both by ecosystem-scale studies and warming experiments. This phenomenon could potentially have global consequences by partially offsetting anthropogenic warming (Friedlingstein23), but could also affect trees on the individual level because longer seasons would push them to increase growth (Grossiord22). However, very recent work has questioned this assumption with proposed mechanisms including drought or heat stress and plants' internal limits to growth (WOLKOVICH UNPUBLISHED; DOW22, GREEN AND KEENAN22). 

In my first chapter, Fuelinex, I aim to assess tree species’ potential to prolong or stretch their activity schedule and if increased carbon pools translate into greater growth increments in the following growing season. I will also investigate potential variations in these responses across deciduous and evergreen species, to test whether different patterns emerge within these distinct groups.
\par 
Recent research has shown how critical phenological traits are in regulating tree growth in urban ecosystems (SIMOVIC24). Thus, in my second chapter, I will investigate how the timing of phenological events affects growth across years for juvenile and mature deciduous trees within an urban ecosystem. This chapter will be split into two projects. The first one consists of a common garden study of trees grown from saplings of four different provenances. I will investigate growth patterns for juvenile individuals at similar life stage than the trees that are often planted in urban landscapes. For my second project I will use the citizen science phenological data on mature trees to better understand how phenological stages and growth are related for trees at advanced life stages. 
    % Provide a justification and description of the research methods undertaken, and its relevance to the subject matter as described above;
To investigate the impact of manipulated spring and fall temperatures on phenological responses, I successfully conducted experiments in 2024 across seven different tree species under controlled conditions, including species that span both fast and short-life strategies (e.g., \textit{Populus balsamifera}) and slow growth and longer lifespan species (e.g., \textit{Quercus macrocarpa}) and including both deciduous and evergreen species. I used a full factorial design of spring and fall warming with two levels each (control/warmed) resulting in four treatments plus an additional two treatments to test fall nutrient effects, using 15 replicates each for a total of 630 individual trees. Throughout the growing season of 2024, I tracked phenological events weekly from the start of the spring treatments through the end of the fall treatments. During the growing season of 2025, the same measurements will be performed. In fall 2025, after the trees have grown in ambient temperatures for the season, I will assess growth on the individual (total biomass) and the cellular level (number of cells and their characteristics), using dendrochronological methods.
\par
For the first project of my second chapter, leaf phenology was monitored for 5 years and tree cookies were collected in early 2023. Then, I will analyze how the tree ring width of each of these years relates to leaf phenology and how these traits vary across years. For my second project, I will use the phenological data collected for the past 10 years by a citizen science program called the Treespotters. To relate these observations to growth, I will collect tree cores from 53 mature deciduous trees monitored by the Treespotters. With these two projects, I aim to draw a more complete picture of the underlying mechanisms driving tree growth and its relation to growing season length and temperature. This information could be crucial for future tree-planting policies in British Columbia. For instance, many trees planted in Metro Vancouver are native to North America, including (e.g. Red maple (\textit{Acer rubrum}), River birch (\textit{Betula nigra}), Yellow birch (\textit{Betula alleghaniensis}), Northern Red Oak (\textit{Quercus rubra}). These four species are among the eleven I will be studying in this project and are widely planted across the Greater Vancouver area. By using experimental data and observational data to demonstrate the impact of growing season length on tree growth—considering both urban-planted juvenile and mature trees—my research has the potential to enhance city green architecture and inform reforestation efforts following forest disturbances.
    % Describe methods of dissemination of research results;
My first chapter, Fuelinex, should be written and ready to be sent for revision in mid-2026. I aim to present my experiments at the annual international phenology conference in 2026.
The second chapter, which consists of two projects, will be condensed into one scientific article in which I investigate how the timing of phenological events affects growth across years for juvenile and mature deciduous trees. Once this publication is sent for revision, I will present my results at the 2025 edition of EcoEvo, in Squamish (BC) and at the 2026 conference of the Ecological Society of America (location TBD) to share the importance of preserving and improving tree planting policies in cities.
    % Outline current progress of thesis/dissertation, including preliminary results;
All treatments for my first chapter have been conducted in 2024 and preliminary results suggest that the replicates subjected to spring and autumn warming experienced longer seasons and growth increment. Trees will be measured until the completion of their growing seasons after which their biomass will be assessed. 
For my second chapter, tree cookies have been processed for dendrochronological measurements. Tree ring widths have been measured and model development is ongoing. Treespotters data is already collected and tree cores will be collected in the end of April of 2025. Preliminary analyses of the phenological measurements indicate longer seasons when spring is early and fall is warmer. 
Chapter 1 data collection and analyses will be completed in early 2026. The results and discussion will be written in February and March 2026. The introduction, methods and abstract will be written in April 2026 and the paper will be sent for peer-review revision in June 2026.
Chapter 2 data collection and analyses will be completed in August 2025. The paper will be written in September to November and will be sent for peer-review revision in December 2025.
    % Outline plan for completion of thesis/dissertation;
Being 7 months into my Master’s, I have already completed the majority of the work for my main chapter. The second chapter is also well underway. I am currently building robust statistical models for the tree cookies which will be further applied to the data extracted from the tree cores that will be collected in April.
    % Speak to the feasibility of completion of thesis/dissertation.
I am confident that my thesis will be completed by August 2026. As a large proportion of my data collection has already been done and my statistical analyses are underway, I believe that the chapter writing will feasibly be completed between September 2025 and March 2026. These chapters will then be sent for revision by my supervisor and my committee meeting in April 2026 and my defense will take around that period. I am confident that the final submission should be done in July 2026.
% References
\bibliographystyle{nature} % set citation style 
\bibliography{2024-CGSM.bib}


\end{document}
