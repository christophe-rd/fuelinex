%------------------------------------------------------------------------------
\documentclass[11pt,letter]{article}
\usepackage[top=0.4in, bottom=0.7in, left=1in, right=1in]{geometry}
\renewcommand{\baselinestretch}{1}
\usepackage{graphicx}
\usepackage{natbib}
\usepackage{amsmath}
\usepackage{hyperref}
\parindent=0pt
\parskip=1 pt

%------------------------------------------------------------------------------
\title{Proposal}
\author{Christophe Rouleau-Desrochers}

\begin{document}
\pagenumbering{gobble}

\maketitle

%%%%%%%%%%%%%%%%%%%%%%%%%
% Lay Abstract %
%%%%%%%%%%%%%%%%%%%%%%%%%

%%%%%%%%%%%%%%%%%%%%%%%%%
% Substantive Eligibility %
%%%%%%%%%%%%%%%%%%%%%%%%%
% Please describe how your research matches the areas of interest specified for the award (max 1800 characters). This section must be filled out.
% 	sustainable approaches to and development of the general urban environment, including water, energy and transportation infrastructure in British Columbia;
% 	environmental protection of oceans, beaches and waterfronts that impact British Columbia; and
% 	sustainable approaches to resource-intensive industry in British Columbia.
\section *{Substantive Eligibility} 
In a context of rapid climate warming caused by increased anthropogenic greenhouse gas emissions, we need to take rapid and drastic actions to mitigate the future consequences that climate change will cause. Summers are hotter and extreme events such as heat domes and severe drought events are happening more often (REF). These phenomena and their consequences on urban environments and forest ecosystems are widely reported. Indeed, cities are forefront victims of climate change, but they are also important sources of greenhouse gases. People living in cities are already experencing increased extreme weather events, heat waves and rising sea levelsgit add, which are expected to increase in frequency and severity over the next decades. (DAS2024). While decreasing greenhouse gas emissions is a worldwide objective, a local one we can pursue in cities such as Vancouver is by mitigating future impacts of climate change. For instance urban trees provide cooling effects which are especially important during summer heat waves. They also provide carbon sequestration, improve air quality, and increase human health and well-being. The latter is especially important because recent work has shown that human exposed to trees are showing considerable mental health benefits. 
\par
In regards to forest ecosystems, increasing evidence shows that the response of trees to climate change will vary a lot by species and location, enlightening the importance of understanding how increased temperature and length of the growing season will affect tree species. Additionally, a common attribution of climate change on forest ecosystems is increased wildfire, a critical and devastating consequence for which we are aware of the mitigation strategies to use. For example, Indigenous nations have practiced  prescribed burning for centuries and we are just recently realizing that this technique can be extremely effective at reducing wildfire intensity and severity. This highlights the critical importance of considering Indigenous knowledge in our climate change response policies. In addition to this, recent work has also shown the critical importance of the tree species on fire intensity where forests with a higher proportion of deciduous tree species usually have lower chances of burning. This brings another key aspect of my research, which is by using deciduous tree in the forestry industry we can reduce fire severity, but understanding how climate change will affect their growth, I aim to provide the Forestry industry with knowledge of which trees will experience increased growth with ongoing climate climate change. %18march: 2000 characters


%%%%%%%%%%%%%%%%%%%%%%%%%
%  EDI & Indigeneity %
%%%%%%%%%%%%%%%%%%%%%%%%%.
% Are there equity or Indigeneity considerations for your research (yes/no)? Please explain (max 1800 characters). This section must be filled out.
\section *{EDI and Indigeneity}
Bridging the gap between the Western view of science and indigenous knowledge is not simple. BC Indigenous communities have an extremely valuable understanding of the forest ecosystem functioning. They have lived in west-coast boreal forests for centuries and have mitigated the risks of wildfires in a way we have yet to implement in the ways we do forestry. However, wildfires are only one of the future consequences of climate change on forest ecosystem and the importance of conserving their ancestrial lands is critical. Climate change-induced pest infestation and summer drought are expected to severely affect BC forest. Modelling approaches and experimental studies can in the decisions on the trees to use when reforesting and potentially use assisted migration to increase the resilience of BC forests. Assisted migration is an effective and widely used technique where trees of certain genotypes are moved to other locations because we expect them to perform better at certain latitudes in a close future. Here, I argue that the preservation of BC old-growth forests and ancestral lands could benefit from increased knowledge of the species and genotypes that we should use in reforestation in order to increase forest resilience so they can face future environmental change better.\\
While I believe my research could help refine the way we do forestry in BC, I believe that understanding tree growth better could also benefit inpoverished urban areas. The positive impacts of green spaces in cities are obvious, however the setbacks are more subtle but can have tremendous impacts on people's lives. Historically, impoverished neighbourhoods in which trees were planted showed increased and rapid gentrification. Over time, this has led to a disparity in green cover between wealthy and poorer areas, resulting in segregated urban landscapes. Consequently, the people living in urban areas poor in tree density are the most affected by increased summer heat waves and their corresponding lethal toll. It is in this context that my research could provide increase knowledge on a more informative and precise selection of the best trees that should be planted in BC cities so everyone can be exposed to the numerous benefits that trees bring in cities. 

%%%%%%%%%%%%%%%%%%%%%%%%%
%  Research Statement % 2 pages
%%%%%%%%%%%%%%%%%%%%%%%%%.
    % Define and clearly describe the research objectives;
\section {Research Statement}
In temperate and boreal forests, temperature plays a crucial role in setting the boundaries for seasonal physiological activity. Thus, with rising temperatures from anthropogenic climate change, the climatically possible growing season has lengthened in many ecosystems worldwide by up to 11 days. Plants have tracked this through shifts in phenology—the study of recurring life history events—which are expected to continue with increasing temperatures. In particular, trees have shifted earlier in the spring and later in the fall. They may use these extra days to fix more carbon and increase growth during the current growing season.\\
First, I aim to assess tree species’ potential to prolong or stretch their activity schedule and if increased carbon pools translate into greater growth increments in the following growing season. I will also investigate potential variations in these responses across
deciduous and evergreen species, to test whether different patterns emerge within these distinct groups.
\par 
In my second chapter, I will investigate how timing of phenological events affects growth across years for juvenile and mature deciduous trees. It will be split into two projects. The first one consists of a common garden study of trees grown from saplings (similar life stage to urban tree planting) of four different provenances. My second project consists of using citizen science phenological data and investigating how phenological stages relate to growth for deciduous mature tree species. 
    % Provide a justification and description of research methods undertaken, and its relevance to the subject matter as described above;
To investigate the impact of manipulated spring and fall temperatures on phenological responses, I successfully conducted experiments in 2024 across seven different tree species under controlled conditions, including species that span both fast and short-life strategies (e.g., \textit{Populus balsamifera}) and slow growth and longer lifespan species (e.g., \textit{Quercus macrocarpa}) and including both deciduous and evergreen species. I used a full factorial design of spring and fall warming with two levels each (control/warmed) resulting in four treatments plus an additional two treatments to test fall nutrient effects, using 15 replicates each for a total of 630 individual trees. Throughout the growing season of 2024, I tracked phenological events weekly from the start of the spring treatments through the end of the fall treatments. During the growing season of 2025, the same measurements will be performed. In fall 2025, after the trees have grown in ambient temperatures for the season, I will assess growth on the individual (total biomass) and the cellular level (number of cells and their characteristics), using dendrochronological methods.
For the first project of my second chapter, leaf phenology was monitored for 5 years and tree cookies were collected in early 2023. Then, I will analyze how the tree ring width of each of these years relates to leaf phenology and how these traits vary across years in the provenance of their source population. For my second project, I will use the phenological data collected for the past 10 years by a citizen science program called the Treespotters. Then, I will collect tree cores from 53 mature deciduous trees monitored by the Treespotters. With these two projects, I am to draw an exhaustive picture of the underlying mechanisms driving tree growth and its relation to growing season length and temperature. This is critical information most of the trees planted in Metro Vancouver are native to North America (e.g. Red maple (\textit{Acer rubrum}), River birch (\textit{Betula nigra}), Yellow birch (\textit{Betula alleghaniensis}), Northern Red Oak (\textit{Quercus rubra}), tree species that will be used in my chapters. By providing evidence for the effects of growing season length on tree growth with experimental data, by using urban planted saplings and mature trees, I believe my projects have the potential to improve the ways we design city green architecture and we reforest after forest disturbances. 
    % Describe methods of dissemination of research results;
The first chapter investigating the impact of longer growing seasons on the following year’s growth under experimental condition should be written and ready for publishing in mid-2026. I will also present Fuelinex experiments at the annual international phenology conference in 2026.
The second chapter, which consists of two projects, will be condensed in one scientic article consisting of studies aiming to answer the broad question of the relationship between the growing season length and growth for both juvenile and mature trees. Once this publication is set for revision, I will present my results to the 2025 edition of EcoEvo and of the conference of Ecological Society of America to share the importance of preserving and planning tree species in cities.
    % Outline current progress of thesis/dissertation, including preliminary results;
All treatments for my first chapter have been conducted in 2024. During the growing season of 2025, the same measurements will be performed. In fall 2025, I will assess growth on the individual (total biomass) and the cellular level (number of cells and their characteristics), using dendrochronological methods. Phenological measurements have been cleaned and preliminary analyses have been conducted.
For my second chapter, tree cookies have been prepared for dendrochronological measurements. Tree ring widths have been measured and model development is ongoing. Treespotters data have already been collected and tree cores will be collected in the end of April of 2025. Preliminary analyses of the phenological measurements indicate longer seasons when spring is early and fall is warmer.  
• Chapter 1 data collection and analyses will be completed in early 2026. The results and discussion will be written in February and March 2026. The introduction, methods and abstract will be written in April 2026 and the paper will be sent for peer-review revision in June 2026.
• Chapter 2 data collectiona and analyses will be completed in August 2025. The paper will be written in September to November and will be sent for peer-review revision in December of 2025.
    % Outline plan for completion of thesis/dissertation;
Being 7 months into my Master’s, I have already successfully completed all the treatments and half the data collection from my main chapter. The second chapter is also well underway. I currently building robust stastistical models for the tree cookies which will be further applied to the data extracted from the tree cores that will be collected in April.
    % Speak to the feasibility of completion of thesis/dissertation.
I am confident that my thesis will be completed by August 2026. As a large proportion of my data collection has already been done and that my statistical analyses are underway, I believe that the chapter writing will feasibly be completeted between September 2025 and March 2026. These chapters will then be sent to revision by my supervisor and my committee meeting in April 2026 and my defense will take place during that period. I am confident that the final submission should be done in July 2026.
% References
\bibliographystyle{nature} % set citation style 
\bibliography{2024-CGSM.bib}


\end{document}
