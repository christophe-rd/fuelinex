%------------------------------------------------------------------------------
\documentclass[11pt,letter]{article}
\usepackage[top=0.4in, bottom=0.7in, left=1in, right=1in]{geometry}
\renewcommand{\baselinestretch}{1}
\usepackage{graphicx}
\usepackage{natbib}
\usepackage{amsmath}
\usepackage{hyperref}
\parindent=0pt
\parskip=1 pt

%------------------------------------------------------------------------------
\title{Proposal}
\author{Christophe Rouleau-Desrochers}

\begin{document}
\pagenumbering{gobble}

\maketitle

%%%%%%%%%%%%%%%%%%%%%%%%%
% Lay Abstract %
%%%%%%%%%%%%%%%%%%%%%%%%%
Climate change is intensifying extreme weather events, posing significant challenges to urban and forest ecosystems. Rising temperatures, heat waves, and increasing wildfire frequency threaten biodiversity and human well-being. In urban environments, tree cover plays a critical role in mitigating heat waves by providing cooling effects, sequestering carbon, and improving air quality. Additionally, exposure to trees has been linked to substantial mental and physical health benefits. With my Master's research I aim to identify tree species best suited for warmer and drier conditions in British Columbia, optimizing urban tree planting strategies and reforestation efforts for climate resilience.
In forest ecosystems, increasing wildfire frequency and severity endanger old-growth forests, necessitating adaptive management strategies. Indigenous communities have long utilized prescribed burning to reduce fire intensity, demonstrating the value of traditional ecological knowledge. Furthermore, emerging research suggests that forests with higher proportions of deciduous trees exhibit lower fire severity. My study aims to give BC forestry industry tree species options that will be best suited to future climate, both to increase wildfire resilience and forest productivity, advocating for the integration of deciduous trees in reforestation efforts. This research has broader implications for environmental justice, as green infrastructure disparities disproportionately affect low-income and marginalized communities. By refining urban forestry strategies and incorporating climate-resilient species, my work aims to support equitable access to urban green spaces while promoting sustainable forest management practices that align with both scientific and Indigenous knowledge systems.

%%%%%%%%%%%%%%%%%%%%%%%%%
% Substantive Eligibility %
%%%%%%%%%%%%%%%%%%%%%%%%%
% Please describe how your research matches the areas of interest specified for the award (max 1800 characters). This section must be filled out.
% 	sustainable approaches to and development of the general urban environment, including water, energy and transportation infrastructure in British Columbia;
% 	environmental protection of oceans, beaches and waterfronts that impact British Columbia; and
% 	sustainable approaches to resource-intensive industry in British Columbia.
\section *{Substantive Eligibility} 
Given the accelerating pace of human-induced climate change, urgent and radical actions are essential to minimize its future impacts. Summers are hotter and extreme events such as heat domes and severe drought events are more frequent \citep{zhang_increased_2023}. These phenomena and their consequences on urban environments and forest ecosystems are widely reported \citep{allen_global_2010, mccarthy_climate_2010}. Indeed, cities are forefront victims of climate change \citep{das_unraveling_2024, corburn_cities_2009}. Urban residents are already facing a rise in extreme weather events, such as heat waves \citep{das_unraveling_2024}. Cities such as Vancouver can mitigate the future impacts of climate change. For instance, urban trees provide cooling effects which can save lives during summer heat waves \citep{ettinger_street_2024, zandler_cooling_2024}. They also provide carbon sequestration, improve air quality, and increase human health and well-being \citep{wolf_urban_2020}. This is particularly important, as recent studies show that exposure to trees significantly benefits both mental and physical health  \citep{wolf_urban_2020, turnerskoff_benefits_2019}. My proposed work could offer valuable insights into selecting tree species best adapted to warmer and drier urban environments.

Regarding forest ecosystems, recent evidence suggests that wildfires will increase with ongoing climate change \citep{wasserman_climate_2023}. More frequent and intense wildfires will likely decrease BC old-growth forest. \citep{price_conflicting_2021} However, we are aware of extremely effective methods to mitigate wildfire impacts on ecosystems. For example, Indigenous nations such as the Lytton First Nation, have successfully practiced prescribed burning to reduce fire intensity and severity for centuries. \citep{lewis_return_2018} This highlights the critical importance of considering Indigenous knowledge in our climate change response policies. Additionally, recent work shows the critical importance of forest species composition on fire intensity where ecosystems with a higher proportion of deciduous tree species usually have lower chances of burning (e.g., \citep{park_impact_2024}). This highlights another crucial aspect of my research: increasing the use of deciduous trees in the forestry industry could help reduce fire severity while offering better options for enhancing forest productivity.



%%%%%%%%%%%%%%%%%%%%%%%%%
%  EDI & Indigeneity %
%%%%%%%%%%%%%%%%%%%%%%%%%.
% Are there equity or Indigeneity considerations for your research (yes/no)? Please explain (max 1800 characters). This section must be filled out.
\section *{EDI and Indigeneity}
Bridging the gap between the Western view of science and indigenous knowledge is not simple, but I believe, necessary. BC Indigenous communities have an extremely valuable understanding of the forest ecosystem functioning. They have lived in west-coast boreal forests for centuries and have successfully mitigated wildfires' severity in ways we have yet to implement in BC forest management. However, wildfires are among the many threats forests will face under future climate change, and the importance of conserving their indigenous ancestral lands is critical. Climate change-driven pest infestations and summer droughts are projected to have severe impacts on BC forests. \citep{williams_climate_2002} Modelling and experimental studies may have the power to better guide decisions on tree species for reforestation, potentially incorporating assisted migration to enhance forest resilience. \citep{aitken_time_2016}

While I believe my research could help refine the way we do forestry in BC, it could also improve tree species composition in impoverished urban areas. The positive impacts of green spaces in cities are obvious, however, the setbacks are more subtle but can have tremendous impacts on people's lives. Low-income communities and people of color often experience climate injustice. One example is green 'climate gentrification,' where new green infrastructure leads to residential displacement. \citep{anguelovski_why_2019} Consequently, the people living in urban areas poor in tree density are the most affected by increased summer heat waves and their corresponding lethal toll. \citep{anguelovski_why_2019}

I argue that preserving BC's old-growth forests and ancestral lands, alongside enhancing knowledge of species and genotypes to improve reforestation efforts, will help forests better withstand future environmental changes. Additionally, my research could contribute valuable insights for the precise selection of better drought and heat-resistant trees to be planted in BC cities, ensuring that everyone can enjoy the benefits of trees in urban environments.
%%%%%%%%%%%%%%%%%%%%%%%%%
%  Research Statement % 2 pages
%%%%%%%%%%%%%%%%%%%%%%%%%.
    % Define and clearly describe the research objectives;
\section* {Research Statement}
\textbf{Introduction:}
In temperate and boreal forests, temperature plays a crucial role in setting the boundaries for seasonal physiological activity. Thus, with rising temperatures from anthropogenic climate change, the climatically possible growing season has lengthened in many ecosystems worldwide by up to 11 days. \citep{korner_phenology_2010, menzel_growing_1999} Plants have tracked this through shifts in phenology—the study of recurring life history events—which are expected to continue with increasing temperatures \citep{wolkovich_warming_2012}. In particular, trees have shifted earlier in the spring and later in the fall. They may use these extra days to fix more carbon and increase growth during the current growing season. \citep{keenan_net_2014, wang_interactive_2020} This is a long-lasting assumption that was supported both by ecosystem-scale studies and warming experiments. This phenomenon could potentially have global consequences by partially offsetting anthropogenic warming, but could also affect trees on the individual level because longer seasons would push them to increase growth. \citep{grossiord_warming_2022} However, recent research challenges this assumption, suggesting that factors such as drought stress and internal growth limitations may counteract the expected benefits. \citep{dow_warm_2022, green_limits_2022}  % Lizzie's paper doesn't work here...
\par
\textbf{Research objectives}: \\
Chapter 1. In my first chapter, I aim to assess tree species’ potential to prolong or stretch their activity schedule and if increased carbon pools translate into greater growth increments in the following growing season. I will also investigate potential variations in these responses across deciduous and evergreen species, to test whether different patterns emerge within these distinct groups.\\
Chapter 2. Recent research has shown how critical phenological traits are in regulating tree growth in urban ecosystems. \citep{simovic_functional_2024} Thus, in my second chapter, I will investigate how the timing of phenological events affects growth across years for juvenile and mature deciduous trees within an urban ecosystem. This chapter will be split into two projects. The first one consists of a common garden study of trees grown from saplings of four different provenances. I will investigate growth patterns for juvenile individuals at similar life stages to the trees that are often planted in urban landscapes. For my second project, I will use the citizen science phenological data on mature trees to better understand how phenological stages and growth are related for trees at advanced life stages.
\par
    % Provide a justification and description of the research methods undertaken, and its relevance to the subject matter as described above;
\textbf{Research methods}: \\
Chapter 1. To investigate the impact of manipulated spring and fall temperatures on phenological responses, I successfully conducted experiments in 2024 across seven different tree species under controlled conditions, including species that span both fast and short-life strategies (e.g., \textit{Populus balsamifera}) and slow growth and longer lifespan species (e.g., \textit{Quercus macrocarpa}) and including both deciduous and evergreen species. \citep{jonsson_annual_2010}  I used a full factorial design of spring and fall warming with two levels each (control/warmed), resulting in four treatments plus an additional two treatments to test fall nutrient effects, using 15 replicates each for a total of 630 individual trees. Throughout the growing season of 2024, I tracked phenological events weekly from the start of the spring treatments through the end of the fall treatments. During the growing season of 2025, the same measurements will be performed. In fall 2025, after the trees have grown in ambient temperatures for the season, I will assess growth on the individual (total biomass) and the cellular level (number of cells and their characteristics), using dendrochronological methods.\\
Chapter 2. For the first project, leaf phenology was monitored for 5 years and tree cross-sections were collected in early 2023. Then, I will analyze how the tree ring width of each of these years relates to leaf phenology and how these traits vary across years. For my second project, I will use the phenological data collected for the past 10 years by a citizen science program called the Treespotters. To relate these observations to growth, I will collect tree cores from 53 mature deciduous trees monitored by the Treespotters. With these two projects, I aim to draw a more complete picture of the underlying mechanisms driving tree growth and its relation to growing season length and temperature. By identifying future climate-resilient species, my work can directly inform municipal and provincial reforestation strategies. For instance, many trees planted in Metro Vancouver are native to North America, including (e.g. Red maple (\textit{Acer rubrum}), River birch (\textit{Betula nigra}), Yellow birch (\textit{Betula alleghaniensis}), Northern Red Oak (\textit{Quercus rubra}). These four species are among the eleven I will be studying in this project and are widely planted across the Greater Vancouver area. By using experimental data and observational data to demonstrate the impact of growing season length on tree growth—considering both urban-planted juvenile and mature trees—my research has the potential to enhance city green architecture and inform reforestation efforts following forest disturbances.\\
    % Describe methods of dissemination of research results;
\textbf{Methods of dissemination:} My first chapter should be written and ready to be sent for revision in mid-2026. I aim to present my experiments at the annual international phenology conference in 2026. The second chapter, which consists of two projects, will be condensed into one scientific article in which I investigate how the timing of phenological events affects growth across years for juvenile and mature deciduous trees. Once this publication is sent for revision, I will present my results at the 2025 edition of EcoEvo, in Squamish (BC) and at the 2026 conference of the Ecological Society of America to share the importance of preserving and improving tree planting policies in cities.
\par
    % Outline current progress of thesis/dissertation, including preliminary results;
\textbf{Progress of thesis/dissertation and preliminary results:}\\ 
Chapter 1. All treatments have been conducted in 2024 and preliminary results suggest that the replicates subjected to spring and autumn warming experienced longer seasons and growth increments. Trees will be measured until the completion of their growing seasons, after which their biomass will be assessed. Data collection, results and discussion will be completed in early 2026. The introduction, methods and abstract will be written in April 2026, and the paper will be sent for peer-review revision in June 2026.
\\
Chapter 2. Tree cross-sections have been processed for dendrochronological measurements. Tree ring widths have been measured, and model development is ongoing. I already cleaned and processed Treespotters data. Tree cores will be collected at the end of April  2025. Preliminary analyses of the phenological measurements indicate longer seasons when spring is early and fall is warmer. Data collection and analyses will be completed in August 2025. The paper will be written from September to November and be sent for peer-review revision in December 2025.

    % Outline plan for completion of thesis/dissertation;
\textbf{Outline for completion of thesis and feasibility:} Being 7 months into my Master’s, I have already completed the majority of the work for Chapter 1. The second chapter is also well underway. I am currently building robust statistical models for the tree cross-sections which will be further applied to the Treespotters tree-core data. 
    % Speak to the feasibility of completion of thesis/dissertation.
I am confident that I will complete my thesis by August 2026. Since a significant portion of my data collection is already finished and my statistical analyses are progressing well, I expect to complete the chapter writing between September 2025 and April 2026. These two chapters will then be submitted for revision by my supervisor and committee in April 2026, with the defense scheduled for around that time. I anticipate submitting the final version of my thesis in July 2026.

% Conclusion
\textbf{Conclusion:} BC’s forests and cities are already experiencing a small fraction of the future impacts of climate change. Therefore, I believe that my proposed work could offer valuable insights into tree species composition to select in BC's reforestation efforts following disturbances. By primarily focusing on deciduous trees, my findings have the potential to inform tree species selection in urban areas, ensuring that both affluent and underserved communities can benefit from the numerous advantages trees provide to people.
% References
\bibliographystyle{nature} % set citation style 
\bibliography{2025-Walls.bib}


\end{document}
