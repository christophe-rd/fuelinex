%------------------------------------------------------------------------------
\documentclass[11pt,letter]{article}
\usepackage[top=0.4in, bottom=0.7in, left=1in, right=1in]{geometry}
\renewcommand{\baselinestretch}{1}
\usepackage{graphicx}
\usepackage{natbib}
\usepackage{amsmath}
\usepackage{hyperref}
\parindent=0pt
\parskip=1 pt

%------------------------------------------------------------------------------
\title{Proposal}
\author{Christophe Rouleau-Desrochers}

\begin{document}
\pagenumbering{gobble}

\maketitle

%%%%%%%%%%%%%%%%%%%%%%%%%
% Lay Abstract %
%%%%%%%%%%%%%%%%%%%%%%%%%

%%%%%%%%%%%%%%%%%%%%%%%%%
% Substantive Eligibility %
%%%%%%%%%%%%%%%%%%%%%%%%%
% Please describe how your research matches the areas of interest specified for the award (max 1800 characters). This section must be filled out.
% 	sustainable approaches to and development of the general urban environment, including water, energy and transportation infrastructure in British Columbia;
% 	environmental protection of oceans, beaches and waterfronts that impact British Columbia; and
% 	sustainable approaches to resource-intensive industry in British Columbia.
\section *{Substantive Eligibility} 
In a context of rapid climate warming caused by increased anthropogenic greenhouse gas emissions in the atmosphere, the necessity to mitigate the future consequences induced by climate change is critical. The consequences of increased summer temperature and climatic variability in urban environments and forest ecosystems are widely reported. Indeed, cities are at the forefront of the contribution and the victim of climate change, where they are significant srouces of global greenhouse gas but they will also face heightened risks from extreme weather events, heat waves and rising sea levels (DAS2024). While decreasing greenhouse gas emissions is a worldwide objective, a local one that we can pursue in cities such as Vancouver is mitigating the potential impacts of climate change. In cities, trees provide cooling effects especially important during summer heat waves. They also provide carbon sequestration, storage management, improve air quality, increase human health and well being. The later is especially important because recent work has shown considerable mental health benefits linked to exposure humans that are exposed to trees on a daily basis. \\
In regards to forest ecosystems, increasing evidence shows that the response of trees to climate change will vary a lot by species and location, enlighting the importance of understanding how increased temperature and length of the growing season will affect tree species. Additionally, a common attribution of climate change on forest ecosystem is increased wild fire risk and this is a critical consequence for which we are aware of the mitigation strategy to use. For example, indigenous nations have understood for centuries that prescribed burning can significantly reduce wild fire intensity and severity. This highlights the critical importance of considering indigenous knowledge in our climate change responses policies. In addition to this, recent work has also shown the critical importance of the tree species on fire intensity where forests with a higher proportion of broaleaf tree species usually have lower chances of burning. %18march: 2000 characters


%%%%%%%%%%%%%%%%%%%%%%%%%
%  EDI & Indigeneity %
%%%%%%%%%%%%%%%%%%%%%%%%%.
% Are there equity or Indigeneity considerations for your research (yes/no)? Please explain (max 1800 characters). This section must be filled out.
\section *{EDI and Indigeneity}
Bridging the gap between the western view of science and indigenous knowledge is not simple. BC indigenous communities have extremely valuable and deep understanding of the forest ecosystem functionning. They have lived in west-coast boreal forests for centuries and have mitigated the risks of wildfires in a way we have yet to implement in the ways we do forestry. For example, prescribed burning at specific periods and place can significantly reduce the severity of summer wild fires. However, wildfires are only one of the future consequences of climate change on forest ecosystem and the importance of conserving there ancestrial lands is of critical importance. If pests infestation, summer drought keep on increasing, we could potentially face a significant decline in BC forests. Planning how climate change will reshape forest ecosystem is hard, but possible and we can act on it. Assisted migration, though controversial, is a popular and widely used technique where trees of certain genotypes are moved to other locations because we expect them to perform better at these places in a close future. Here, I argue that in order to preserve BC old growth forests and ancestral lands, we should provide data and information to indigenous nations so we can face future environmental change better.\\
In addition to this, the increased green spaces in cities also have setbacks. Historically inpoverished neighboorhood in which trees are planted are showing incresed gentrification leading. Consequently, urban areas that are usually poor in tree density are the poorest and with increased summer heat waves and the letal toll that comes with it, inpoverished city areas are likely to be significantly more affected than richer neighboorhoods. 

%%%%%%%%%%%%%%%%%%%%%%%%%
%  Research Statement % 2 pages
%%%%%%%%%%%%%%%%%%%%%%%%%.
    % Define and clearly describe the research objectives;
\section {Research Statement}
In temperate and boreal forests, temperature plays a crucial role in setting the boundaries for seasonal physiological activity. Thus, with rising temperatures from anthropogenic climate change, the climatically possible growing season has lengthened in many ecosystems worldwide by up to 11 days. Plants have tracked this through shifts in phenology—the study of recurring life history events—which are expected to continue with increasing temperatures. In particular, trees have shifted earlier in the spring and later in the fall. They may use these extra days to fix more carbon and increase growth during the current growing season.\\
First, I aim to assess tree species’ potential to prolong or stretch their activity schedule and if increased carbon pools translate into greater growth increments in the following growing season. I will also investigate potential variations in these responses across
deciduous and evergreen species, to test whether different patterns emerge within these distinct groups.
In my second chapter, I will investigate how tree growth relates to leaf phenology and across year climate. It will consist of two projects, both consisting of observational data. The first one is of a common garden study of trees grown from saplings (similar life stage to urban tree planting) of four different provenances. My second project consists of using citizen science phenological data and investigate how phenological stages relate to growth for deciduous mature tree species. 
    % Provide a justification and description of research methods undertaken, and its relevance to the subject matter as described above;
To investigate the impact of manipulated spring and fall temperatures on phenological responses, I successfully conducted experiments in 2024 across seven different tree species under controlled conditions, including species that span both fast and short-life strategies (e.g., Populus balsamifera) and slow growth and longer lifespan species (e.g., Quercus macrocarpa) and including both deciduous and evergreen species.(?) I used a full factorial design of spring and fall warming with two levels each (control/warmed) resulting in four treatments plus an additional two treatments to test fall nutrient effects, using 15 replicates each for a total of 630 individual trees. Throughout the growing season of 2024, I tracked phenological events weekly from the start of the spring treatments through the end of the fall treatments. During the growing season of 2025, the same measurements will be performed. In fall 2025, after the trees have grown in ambient temperatures for the season, I will assess growth on the individual (total biomass) and the cellular level (number of cells and their characteristics), using dendrochronological methods.
For the first project of my second chapter, leaf phenology was monitored for 5 years and tree cookies were collected in early 2023. Then I will analyse how tree ring width of each of these years relate to leaf phenology and how both of these traits relate to the climate and if trees from different latitude show differences in their capacity to grow. For my second project, I will use the phenological data collected for the past 10 years of a citizen science program called the Treespotters. Then next April, I will collect tree cores from 53 of the mature deciduous trees that were monitored by the Treespotters. Both of these projects will allow me to draw a more complete picture of the underlying mechanisms driving tree growth and its relation to growing season length and temperature. This is critical information as the proportion of trees that are planted in Metro Vancouver are from all of North America (– Red maple (Acer rubrum), River birch (Betula nigra), Yellow birch (Betula alleghaniensis), Northern Red Oak), tree species that will be used in my chapters. Having a better understanding of their future growth response in their native habitat could provide better guidance for future tree planting policies in the cities of the greater Vancouver area. 
    % Describe methods of dissemination of research results;
The first chapter investigating the impact of longer growing seasons on the following year’s growth under experimental condition should be written and ready for publishing in mid-2026. I will also present Fuelinex experiments at the annual international phenology conference in 2026.
The second chapter, which consists of two projects, will be condensed in one scientic article consisting of studies aiming to answer the broad question of the relationship between the growing season length and growth for both juvenile and mature trees. Once this publication is set for revision, I will present my results to the 2025 edition of EcoEvo and of the conference of Ecological Society of America to share the importance of preserving and planning tree species in cities.
    % Outline current progress of thesis/dissertation, including preliminary results;
All treatments for my first chapter have been conducted in 2024. During the growing season of 2025, the same measurements will be performed. In fall 2025, I will assess growth on the individual (total biomass) and the cellular level (number of cells and their characteristics), using dendrochronological methods. Phenological measurements have been cleaned and preliminary analyses have been conducted.
For my second chapter, tree cookies have been prepared for dendrochronological measurements. Tree ring widths have been measured and model development is ongoing. Treespotters data have already been collected and tree cores will be collected in the end of April of 2025. Preliminary analyses of the phenological measurements indicate longer seasons when spring is early and fall is warmer.  
• Chapter 1 data collection and analyses will be completed in early 2026. The results and discussion will be written in February and March 2026. The introduction, methods and abstract will be written in April 2026 and the paper will be sent for peer-review revision in June 2026.
• Chapter 2 data collectiona and analyses will be completed in August 2025. The paper will be written in September to November and will be sent for peer-review revision in December of 2025.
    % Outline plan for completion of thesis/dissertation;
Being 7 months into my Master’s, I have already successfully completed all the treatments and half the data collection from my main chapter. The second chapter is also well underway. I currently building robust stastistical models for the tree cookies which will be further applied to the data extracted from the tree cores that will be collected in April.
    % Speak to the feasibility of completion of thesis/dissertation.
I am confident that my thesis will be completed by August 2026. As a large proportion of my data collection has already been done and that my statistical analyses are underway, I believe that the chapter writing will feasibly be completeted between September 2025 and March 2026. These chapters will then be sent to revision by my supervisor and my committee meeting in April 2026 and my defense will take place during that period. I am confident that the final submission should be done in July 2026.
% References
\bibliographystyle{nature} % set citation style 
\bibliography{2024-CGSM.bib}


\end{document}
