%------------------------------------------------------------------------------
\documentclass[11pt,letter]{article}
\usepackage[top=0.4in, bottom=0.7in, left=1in, right=1in]{geometry}
\renewcommand{\baselinestretch}{1}
\usepackage{graphicx}
\usepackage{natbib}
\usepackage{amsmath}
\usepackage{hyperref}
\parindent=0pt
\parskip=1 pt

%------------------------------------------------------------------------------
\title{Proposal}
\author{Christophe Rouleau-Desrochers}

\begin{document}
\pagenumbering{gobble}

\maketitle

%%%%%%%%%%%%%%%%%%%%%%%%%
% Lay Abstract %
%%%%%%%%%%%%%%%%%%%%%%%%%
Climate change is intensifying extreme weather events, posing significant challenges to urban and forest ecosystems. Rising temperatures, heat waves, and increasing wildfire frequency threaten biodiversity and human well-being. In urban environments, tree cover provides critical services that can help mitigate these threats. Trees cool local areas to reduce heat waves, sequester carbon, and improve air quality. Additionally, exposure to trees has been linked to mental and physical health benefits. With my Master's research, I aim to identify tree species best suited for warmer and drier conditions in British Columbia, optimizing urban tree planting strategies and reforestation efforts for climate resilience.
% My one big comment is that the fire thing does not work, it is jarring to switch to it as a reader. I suggest you remove it and mention it as a possible EXTENSION of your work (see start of edits below for you to finish/fix). Use the space you save here to lay out your experiment quickly and then how you will scale it up to juvenile and adult tress by leveraging an urban common garden and community science data in an urban arboretum. 
% ' the across-year carry-over effect of the extended growing season' is so much! Why not just: the effects of the extended growing seasons'
In 2024, I conducted a large-scale experiment on saplings of seven tree species using a full factorial design with spring and fall warming to understand the across-year carry-over effect of the extended growing season. I also am scaling up this work to juvenile and mature tree species by leveraging data from a common garden and community science program in an urban arboretum. 

My MSc research has broader implications for both British Colombia forests and for environmental justice. For BC forests, my work could provide insights into selecting tree species that will have higher productivity under future climate change. For environmental justice, my work can improve green infrastructure policy to reduce disparities, which are currently high as green infrastructure is extremely poor in low-income and marginalized communities. By recommending climate-resilient species for urban planting to help refine plans to reduce these disparaties, my research supports equitable access to green spaces and sustainable forest management practices that integrate both scientific and Indigenous knowledge systems. % ending sentences could probably use a little more work

%%%%%%%%%%%%%%%%%%%%%%%%%
% Substantive Eligibility %
%%%%%%%%%%%%%%%%%%%%%%%%%
% Please describe how your research matches the areas of interest specified for the award (max 1800 characters). This section must be filled out.
% 	sustainable approaches to and development of the general urban environment, including water, energy and transportation infrastructure in British Columbia;
% 	environmental protection of oceans, beaches and waterfronts that impact British Columbia; and
% 	sustainable approaches to resource-intensive industry in British Columbia.
\section *{Substantive Eligibility} 
Given the accelerating pace of human-induced climate change, urgent and radical actions are essential to minimize its future impacts. Summers are hotter and extreme events like heat domes and severe drought events are more frequent \citep{zhang_increased_2023}. These phenomena and their consequences on urban environments and forest ecosystems are widely reported \citep{allen_global_2010, mccarthy_climate_2010}. Indeed, cities are the forefront victims of climate change \citep{das_unraveling_2024, corburn_cities_2009}. Urban residents are already facing a rise in extreme weather events, such as heat waves\citep{das_unraveling_2024}. Yet, cities like Vancouver can mitigate the future impacts of climate change through greener urban planning. For instance, urban trees provide cooling effects which can save lives during summer heat waves \citep{ettinger_street_2024, zandler_cooling_2024}. They also provide carbon sequestration, improve air quality, and increase human health and well-being \citep{wolf_urban_2020}. This is particularly important, as recent studies show that exposure to trees significantly benefits both mental and physical health  \citep{wolf_urban_2020, turnerskoff_benefits_2019}. My proposed work will provide valuable insights into selecting tree species best suited to warmer and drier urban environments, and optimizing tree-planting strategies for climate resilience. % I think you have duplicate phrases! Check your work on deciduous trees. 

% Again, re-work this to focus on urban systems and treat other parts as an EXTENSION. That means this will be shorter ... 
Beyond cities, climate change is also intensifying wildfire risk. \citep{wasserman_climate_2023}. 
% Point below is good! I would keep that ... but I would rephrase the second sentence to focus on it being somewhat selective so say something more like: increasing the role of  deciduous trees in the forestry industry---either through better understory planning or in selective locations ideal for deciduous species in the future---could help reduce fire severity while offering better options for enhancing forest productivity
Recent work highlights the crucial role of forest species composition in fire behaviour, with ecosystems containing more deciduous trees experience lower fire severity. \citep{park_impact_2024}. 
Increasing the role of deciduous trees in the forestry industry---either through better understory planning or in selective locations ideal for deciduous species--could help reduce fire severity in the future while offering better options for enhancing forest productivity.\citep{mack_carbon_2021} Increasing the use of deciduous trees—either through better understory planning or in selective locations ideal for deciduous species—could help mitigate fire intensity while increasing forest productivity. Thus, my research will contribute to strategies that enhance both urban forestry and broader forest management efforts in a changing climate.
%%%%%%%%%%%%%%%%%%%%%%%%%
%  EDI & Indigeneity %
%%%%%%%%%%%%%%%%%%%%%%%%%.
% Are there equity or Indigeneity considerations for your research (yes/no)? Please explain (max 1800 characters). This section must be filled out.
\section *{EDI and Indigeneity}
Bridging the gap between Western science and Indigenous knowledge is complex but essential. Indigenous communities in British Colombia hold an extremely valuable understanding of forest ecosystem functioning. A prime example is their successful wildfire mitigation in ways we have yet to implement in BC forest management. Along with increased wildfires, climate change-driven pest infestations and summer droughts are projected to severely impact BC forests. \citep{williams_climate_2002} Thus, modelling and experimental studies may have the power to better guide decisions on tree species for reforestation, potentially incorporating assisted migration to enhance forest resilience. \citep{aitken_time_2016}

While I believe my research could help refine the way we do forestry in BC, it could also improve tree species composition in impoverished urban areas. Green spaces in cities provide significant benefits. 
% I am confused by the next sentence, what setbacks? Do you mean the benefits? Check!
However, the setbacks are more subtle but can have tremendous impacts on people's lives. Low-income communities and people of color often experience climate injustice. One example is through green 'climate gentrification,' where new green infrastructure leads to residential displacement. \citep{anguelovski_why_2019} Consequently, the people living in urban areas poor in tree density are the most affected by increased summer heat waves and future sea level rise and their corresponding lethal toll. \citep{anguelovski_why_2019,triffo_green_2022} 

I argue that preserving BC's old-growth forests and ancestral lands, alongside advancing knowledge of species and genotypes to improve reforestation efforts, will help forests better withstand future environmental changes. Additionally, my research can inform the selection of drought and heat-resistant trees to be planted in BC cities, promoting more equitable and inclusive access to trees in urban environments.

%%%%%%%%%%%%%%%%%%%%%%%%%
%  Research Statement % 2 pages
%%%%%%%%%%%%%%%%%%%%%%%%%.
    % Define and clearly describe the research objectives;
\section* {Research Statement}
\textbf{Introduction:}
In temperate and boreal forests, temperature plays a crucial role in setting the boundaries for seasonal physiological activity. Thus, with rising temperatures from anthropogenic climate change, the climatically possible growing season has lengthened in many ecosystems worldwide by up to 11 days. \citep{korner_phenology_2010, menzel_growing_1999} Plants have tracked this through shifts in phenology—the study of recurring life history events—which are expected to continue with increasing temperatures \citep{wolkovich_warming_2012}. In particular, trees have shifted earlier in the spring and later in the fall. They may use these extra days to fix more carbon and increase growth during the current growing season. \citep{keenan_net_2014, wang_interactive_2020} This is a long-lasting assumption that was supported both by ecosystem-scale studies and warming experiments. This phenomenon could potentially have global consequences by partially offsetting anthropogenic warming, but could also affect trees on the individual level because longer seasons would push them to increase growth. \citep{grossiord_warming_2022} However, recent research challenges this assumption, suggesting that factors such as drought stress and internal growth limitations may counteract the expected benefits. \citep{dow_warm_2022, green_limits_2022}  % Lizzie's paper doesn't work here...
\par
\textbf{Research objectives}: \\
In my first chapter, I aim to assess tree species’ potential to prolong or stretch their activity schedule and if increased carbon pools translate into greater growth increments in the following growing season. I will also investigate potential variations in these responses across deciduous and evergreen species, to test whether different patterns emerge within these distinct groups.\\
Recent research has shown that phenological traits are critical in regulating tree growth in urban ecosystems. \citep{simovic_functional_2024} Thus, my first chapter will be complemented by two smaller projects, leveraging the data from a common garden and community science program to investigate how the timing of phenological events affects growth across years for juvenile and mature deciduous trees within an urban arboretum. 
\par
    % Provide a justification and description of the research methods undertaken, and its relevance to the subject matter as described above;
\textbf{Research methods}: \\
To investigate the impact of manipulated spring and fall temperatures on phenological responses, I successfully conducted experiments in 2024 across seven different tree species under controlled conditions, including species that span both fast and short-life strategies (e.g., \textit{Populus balsamifera}) and slow growth and longer lifespan species (e.g., \textit{Quercus macrocarpa}) and including both deciduous and evergreen species. \citep{jonsson_annual_2010}  I used a full factorial design of spring and fall warming with two levels each (control/warmed), resulting in four treatments plus an additional two treatments to test fall nutrient effects, using 15 replicates each for a total of 630 individual trees. Throughout the growing season of 2024, I tracked phenological events weekly from the start of the spring treatments through the end of the fall treatments. During the growing season of 2025, the same measurements will be performed. In fall 2025, after the trees have grown in ambient temperatures for the season, I will assess growth on the individual (total biomass) and the cellular level (number of cells and their characteristics), using dendrochronological methods.\\
My second chapter consists of two complementary projects, offering a broader perspective on the mechanisms driving tree growth in relation to growing season length and temperature. For my first project, I will use five years of leaf phenology data and tree cross-sections that were collected at the common garden study. Then, I will analyze how the tree ring width relates to leaf phenology and how these traits vary across years. Then, for my second project, I will use 10 years of phenological data from the Treespotters program which tracked 53 mature deciduous trees. To relate these observations to growth, I will collect tree cores from these individuals.  

\par 
By identifying future climate-resilient species, my work can directly inform municipal and provincial reforestation strategies. For instance, many trees planted in Metro Vancouver are native to North America, including (e.g. Red maple (\textit{Acer rubrum}), River birch (\textit{Betula nigra}), Yellow birch (\textit{Betula alleghaniensis}), Northern Red Oak (\textit{Quercus rubra}). These four species are among the eleven I will be studying in this project and are widely planted across the Greater Vancouver area. By integrating experimental and observational data to demonstrate the impact of growing season length on tree growth—considering both urban-planted juvenile and mature trees—my research has the potential to improve urban green architecture and guide reforestation efforts following forest disturbances.\\
    % Describe methods of dissemination of research results;
    % They probably want to hear that you will reach the community ... so I would get through what you have below in 1/2 to 1/3 the space, basically say as a topic sentence you plan to communicate through usual academic means, but also really want to reach beyond this community. Then quickly review what you will do with ESA etc. and then spend a little more time giving some examples of stuff you will also do. You could look around online, you could also go to this https://bcinvasives.ca/ (they have lots of meetings) and see if this thesis mentions how planning works (https://open.library.ubc.ca/soa/cIRcle/collections/ubctheses/24/items/1.0421336) you could also connect with Ailene Ettinger of TNC who works on a green infrastructure program for the emerald edge (https://www.nature.org/en-us/about-us/where-we-work/priority-landscapes/emerald-edge/) 
\textbf{Methods of dissemination:} I plan to communicate my research through usual academic means, including peer-reviewed publications and dissemination in scientific events. More specifically, I will present my results and findings to local and international scientific conferences, such as Ecological Society of America (ESA), Eco-Evo, the International Phenology Conference to share the importance of preserving and improving tree planting policies in cities. However, I am convinced that reaching beyond the scientific community is paramount.Public engagement is crucial, especially for such vital issues as urban green infrastructure, which has the potential to bring tremendous positive effects by informing policymakers about new scientific insights. In turn, these insights could lead to better and faster changes in the living environment of the most marginalized communities in Vancouver.

To break out of scientific echo chambers, I will present my research through webinars and talks at events organized by the Invasive Species Council of BC, the Vancouver City Parks Restoration Group, and Green Teams Canada. By disseminating my research in these spaces, I can more effectively address climate injustice in Vancouver as suggested by Triffo’s (2022) work.  \citep{triffo_green_2022} Beyond public presentations, I  intend to create accessible, engaging versions of my research. These will integrate perspectives of those most impacted by climate injustice, alongside insights from interdisciplinary experts, for publication in science magazines and Vancouver neighbourhood newspapers.
\par
    % Outline current progress of thesis/dissertation, including preliminary results;
    % Somewhere I would address concerns that you are doing too much ... maybe at the end here you could say: while the project I am undertaking is large, I am leveraging existing data and have already successfully completed most of the large experiment thus I am confident I will complete this with time for full dissemination as a Walls scholar (or whatever they call you). Also, you can save words by not always referring to your chapters. It may even more compelling to talk of your major experiment which is complemented by two related additional studies that allow you to scale up to juvenile and adult trees (BTW, check that calling trees 'juvenile' is okay ... I don't know the forestry terminology!)
    %CRD28March: I cut a big chunk of the objectives and replaced it by a sentence that basically says this! 
\textbf{Progress of thesis/dissertation and preliminary results:}\\ 
For my main experiment, I conducted all treatments in 2024, and preliminary results suggest that the replicates subjected to spring and autumn warming experienced extended seasons and increased growth. I will measure trees until the completion of their growing seasons, after which I will assess their biomass. I will complete data collection, results and discussion in early 2026. I will write the introduction, methods and abstract in April  2026, and the paper will be sent for peer-review revision in June 2026.
\\
Regarding my second chapter, I processed tree cross-sections for dendrochronological measurements and measured tree ring widths. Additionally, I cleaned and processed Treespotters data and will be collecting tree cores at the end of April 2025. Preliminary analyses of the phenological measurements indicate longer seasons when spring is early and fall is warmer. I will complete data collection and analyses by August and write the paper from September to November 2025. Once the paper has been finalized, it will be sent for peer-review revision in December 2025.
While the projects I am undertaking are large, I am leveraging existing data for my second chapter and have successfully completed most of my large experiments. I am therefore confident I will complete my thesis with time for full dissemination as a Wall's Award holder. 
\par
    % Outline plan for completion of thesis/dissertation;
\textbf{Outline for completion of thesis and feasibility:} Being seven months into my Master’s, I have already completed the majority of the work for my first chapter. Chapter 2 is also well underway and I am currently building robust statistical models with colleagues for the tree cross-sections and Treespotters tree-core data. 
    % Speak to the feasibility of completion of thesis/dissertation.
I am very positive that I will complete my thesis by August 2026. With a significant portion of my data collection already finished and my statistical analyses progressing well, I anticipate completing manuscript writing between November 2025 and April 2026. These two chapters will then be submitted for revision by my supervisor and committee in April 2026, with the defence scheduled for around that time. I anticipate submitting the final version of my thesis in July 2026.

% Conclusion
\textbf{Conclusion:} British Colombia’s forests and cities are already experiencing a small fraction of the future impacts of climate change. Therefore, I believe that my proposed work could offer valuable insights into tree species composition to select in BC's reforestation efforts following disturbances. By primarily focusing on deciduous trees, my findings have the potential to inform tree species selection in reforestation efforts and urban areas, ensuring that communities—both affluent and underserved—benefit from the many ecological and social advantages trees provide.
% References
\bibliographystyle{nature} % set citation style 
\bibliography{2025-Walls.bib}


\end{document}
