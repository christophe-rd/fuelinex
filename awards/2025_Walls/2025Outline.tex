%------------------------------------------------------------------------------
\documentclass[11pt,letter]{article}
\usepackage[top=0.4in, bottom=0.7in, left=1in, right=1in]{geometry}
\renewcommand{\baselinestretch}{1}
\usepackage{graphicx}
\usepackage{natbib}
\usepackage{amsmath}
\usepackage{hyperref}
\parindent=0pt
\parskip=1 pt

%------------------------------------------------------------------------------
\title{Personal Statement}
\author{Christophe Rouleau-Desrochers}

\begin{document}
\pagenumbering{gobble}

\maketitle
% On an attached sheet, please describe your academic interests, extracurricular activities, and the program of studies you intend to pursue, and outline your plans for a career. Your statement must be readily understandable by someone outside your discipline, and must be complete in 600 words


%%%%%%%%%%%%%%%%%%%%%%%%%
% Lay Abstract %
%%%%%%%%%%%%%%%%%%%%%%%%%

%%%%%%%%%%%%%%%%%%%%%%%%%
% Substantive Eligibility %
%%%%%%%%%%%%%%%%%%%%%%%%%
% Please describe how your research matches the areas of interest specified for the award (max 1800 characters). This section must be filled out.
% 	sustainable approaches to and development of the general urban environment, including water, energy and transportation infrastructure in British Columbia;
% 	environmental protection of oceans, beaches and waterfronts that impact British Columbia; and
% 	sustainable approaches to resource-intensive industry in British Columbia.
\section *{Substantive Eligibility}
\begin{itemize}
	\item In the context of a warming climate, the necessity of trees in cities is increasing.
	\item Benefits of city trees:
	\begin{itemize}
		\item Water retention
		\item Urban heat islands
		\item Decreased pollution
		\item Increased wellness
	\end{itemize}
	\item With longer seasons, increased heat waves, drought, and pest invasion, city trees could decline.
	\item In the Greater Vancouver area, trees from all around the continent are planted—from red oaks to birches, sequoias to pines—so understanding how they will respond to increasing temperatures is essential to prevent their decline and eventually replant more resilient species.
	\item If longer seasons lead to increased growth, this would raise timber value. However, if the relationship is negative, setbacks could be enormous.
	\item One of the mitigation strategies for decreasing fire risks (in addition to prescribed burns by Indigenous people) is to plant deciduous trees such as balsam poplars and paper birch.
\end{itemize}

%%%%%%%%%%%%%%%%%%%%%%%%%
%  EDI & Indigeneity %
%%%%%%%%%%%%%%%%%%%%%%%%%.
% Are there equity or Indigeneity considerations for your research (yes/no)? Please explain (max 1800 characters). This section must be filled out.
\section *{EDI and Indigeneity}
\begin{itemize}
	\item The fact that most trees in cities are planted in wealthy areas and that impoverished, often have less trees
	\item More access to parks. Paper showing that parks and neighbourhood richness are negatively correlated.
	\item Provide support to Indigenous communities in BC on which species, from which provenances should be planted to mitigate fire risks and boost biodiversity
\end{itemize}
%%%%%%%%%%%%%%%%%%%%%%%%%
%  Research Statement % 2 pages
%%%%%%%%%%%%%%%%%%%%%%%%%.
    % Define and clearly describe the research objectives;
\section*{Research Statement}
In temperate and boreal forests, temperature plays a crucial role in setting the boundaries for seasonal physiological activity. Thus, with rising temperatures from anthropogenic climate change, the climatically possible growing season has lengthened in many ecosystems worldwide by up to 11 days.\citep{korner_phenology_2010, menzel_growing_1999} Plants have tracked this through shifts in phenology—the study of recurring life history events—which are expected to continue with increasing temperatures.\citep{wolkovich_warming_2012} In particular, trees have shifted earlier in the spring and may use these extra days to fix more carbon and increase growth during the current growing season.\citep{keenan_net_2014, wang_interactive_2020} At the same time, fall events in trees (e.g., leaf senescence) have been delayed, but the impacts on their fitness are not well understood. Together earlier spring and delayed fall events are often hypothesized to affect growth in the next growing season. This is rarely tested, however, and tests to date have used adult trees where many co-varying factors make teasing out the effect of longer seasons difficult. Here, I propose an extended season experiment using saplings to mechanistically test this critical hypothesis. My proposed work will provide valuable insight into the regeneration capacity of forests under a warming climate, considering the importance of young trees on forest recruitment. \citep{zohner_how_2021} \\
\textbf {Research Question:} How do extended growing seasons affect tree growth across different species both immediately (in the same year as the extended season) and in subsequent years? \\
\textbf {Hypothesis:} I hypothesize that an extension of the growing season could modify a tree’s capacity to fill carbon and nitrogen storage pools.\citep{chapin_ecology_1990, lawrence_variable_2018} Trees that use this opportunity by fixing more carbon may experience increased growth in the subsequent growing season.\citep{landhausser_partitioning_2012, martens_first-year_2007} Thus, species capable of accumulating nutrients after growth cessation while going through leaf senescence might exhibit growth increment in the following growing season.\citep{schott_premature_2013} \\
\subsection*{Chapter 1}
\textbf {Objectives:} First, I aim to assess tree species' potential to prolong or stretch their activity schedule. Second, I will determine whether trees can absorb nutrients beyond their theoretical growing season. I will also examine if increased carbon pools translate into greater growth increments in the following growing season. Finally, I will investigate potential variations in these responses across deciduous and evergreen species, to test whether different patterns emerge within these distinct groups.\\
\subsection*{Chapter 2}
\textbf {Objectives} 
\begin{itemize}
	\item how trees grown from saplings (like they are often planted in cities) in an urban Arboretum have their growth affected by their phenology and year climate. 
	\item understand how the phenological events of mature deciduous trees relate to their growth.
\end{itemize}
    % Provide a justification and description of research methods undertaken, and its relevance to the subject matter as described above;
\begin {itemize}
	\item To investigate the impact of manipulated spring and fall temperatures on phenological responses, I successfully conducted experiments in 2024 across seven different tree species under controlled conditions, including species that span both fast and short-life strategies (e.g., \emph{Populus balsamifera}) and slow growth and longer lifespan species (e.g., \emph{Quercus macrocarpa}) and including both deciduous and evergreen species.\citep{jonsson_annual_2010} I used a full factorial design of spring and fall warming with two levels each (control/warmed) resulting in four treatments plus an additional two treatments to test fall nutrient effects, using 15 replicates each for a total of 630 individual trees. 
Throughout the growing season of 2024, I tracked phenological events weekly from the start of the spring treatments through the end of the fall treatments. During the growing season of 2025, the same measurements will be performed. In fall 2025, after the trees have grown in ambient temperatures for the season, I will assess growth on the individual (total biomass) and the cellular level (number of cells and their characteristics), using dendrochronological methods.
	\item Phenological observations were conducted for 5 years and tree cookies were collected in beginning of 2023. I measured tree ring width of each cookies.
	\item A citizen science program was conducted at the Arnold Arboretum of Harvard University since 2015. Thousands of phenological observations have been recorded on 53 trees. For the first time, the phenology of mature trees and its relation to year growth on the cellular level will be assessed. As most of the trees used in this study are planted accross North America, for example, 
	\begin{itemize}
		\item Red maple (Acer rubrum): 
		\item River birch (Betula nigra)
		\item Yellow birch (Betula alleghaniensis)
		\item Northern Red Oak 
	\end {itemize}
\end {itemize}
    % Describe methods of dissemination of research results;
\begin {itemize}
	\item The first chapter investigating the impact of longer growing season on the following year's growth under experimental condition should be written and ready for publishing in mid-2026. I will also present Fuelinex experiments at the annual international phenology conference in 2026. 
	\item The second chapter, which consists of two projects, will be condensed in a longer and denser scientic article which will aim to answer the question of the relationship between the GSL and growth for both juvenile and mature trees. I also present my results to the 2025 edition of EcoEvo at ESA to share the importance of preserving and planning tree species in cities. 
\end {itemize}
    % Outline current progress of thesis/dissertation, including preliminary results;
\begin {itemize}
	\item All treatments for my first chapter have been conducted in 2024. During the growing season of 2025, the same measurements will be performed. In fall 2025, I will assess growth on the individual (total biomass) and the cellular level (number of cells and their characteristics), using dendrochronological methods. Phenological measurements have been cleaned and preliminary analyses have been conducted.
	\item Tree cookies have been prepared for dendrochronological measurements. Tree ring width have been measured and model developement is currently ongoing. 
	\item Treespotters data have already been collected and tree cores will be collected in the end of April of 2025. Preliminary analyses of the phenological measurements indicate longer seasons when spring is early and fall is warmer. 
\end {itemize}
    % Outline plan for completion of thesis/dissertation;
\begin {itemize}
	\item Chapter 1 data collection and analyses will be completed in early 2026. The results and discussion will be written in February and March 2026. The introduction, methods and abstract will be written in April 2026 and the paper will be sent for peer-review revision in June 2026.
	\item Chapter 2 data collectiona and analyses will be completed in August 2025. The paper will be written in September to November and will be sent for peer-review revision in December of 2025.
\end {itemize}
    % Speak to the feasibility of completion of thesis/dissertation.
\begin {itemize}
	\item Being 7 months into my Master's, I have already successfully completed all the treatements and half the data collection from my main chapter. The second chapter is also well underway. I will be able to build robust models for the tree cookies which will be further applied to the tree cores that will be collected in April. 
\end {itemize}

% References
\bibliographystyle{nature} % set citation style 
\bibliography{2024-CGSM.bib}


\end{document}
