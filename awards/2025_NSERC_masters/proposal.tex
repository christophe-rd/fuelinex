%------------------------------------------------------------------------------

\documentclass{article}
\renewcommand{\baselinestretch}{1.6}
% \renewcommand*{\thefootnote}{\fnsymbol{footnote}}
\usepackage{sectsty,setspace} 
\usepackage[top=1.87cm, bottom=1.87cm, left=1.87cm, right=1.87cm]{geometry} 
\usepackage{epstopdf} % for pdf creation
\usepackage{amsmath,latexsym,amssymb,wasysym} % improving structure
\usepackage{cite}
%\usepackage{citesupernumber}
%\usepackage{citecollapse}
%\usepackage{hyperref}
\usepackage{float} % for figures and tables
\usepackage[sort&compress, super, numbers]{natbib}
\floatstyle{plaintop}
\restylefloat{table}
%\usepackage[table]{xcolor} % http://ctan.org/pkg/xcolor
\usepackage{times} % setting font to times

\setlength\parindent{0pt} % no indents throughout
%------------------------------------------------------------------------------
\begin{document}
\section*{Fueling Next Year's Growth of Trees with Carbon and Nitrogen}
\textbf {Context:}
In temperate and boreal forests, temperature plays a crucial role in setting the boundaries for seasonal physiological activity. Thus, with rising temperatures from anthropogenic climate change, the climatically possible growing season has lengthened in many ecosystems worldwide by up to 11 days \citep{korner_phenology_2010, menzel_growing_1999}. Plants have tracked this through shifts in phenology—the study of recurring life history events—which are expected to continue with increasing temperatures \citep{wolkovich_warming_2012}. In particular, trees have shifted earlier in the spring and may use these extra days to fix more carbon and increase growth during the current growing season \citep{keenan_net_2014, wang_interactive_2020}. Also, autumn events in trees (e.g., leaf senescence) have been delayed but the impacts on their fitness are not well understood. Both earloer spring and delayed fall events are likely to affect the next growing season, though this is rarely tested. \\
\textbf {Research Question:} How do extended growing seasons affect tree growth across different species both immediately (in the same year as the extended season) and in subsequent years? \\
\textbf {Hypothesis:} I hypothesize that an extension of the growing season could modify a tree’s capacity to fill carbon and nitrogen storage pools \citep{chapin_ecology_1990, lawrence_variable_2018}. Trees that use this opportunity by fixing more carbon may experience increased growth in the subsequent growing season \citep{landhausser_partitioning_2012, martens_first-year_2007}. Thus, species capable of accumulating nutrients after growth cessation while going through leaf senescence might exhibit growth increment in the following growing season \citep{schott_premature_2013}. \\
\textbf {Objectives:} First, I aim to assess the tree species' potential to prolong or stretch their activity schedule. Second, I will determine whether trees can absorb nutrients beyond their theoretical growing season. I will also examine if increased carbon storage pools translate into growth increment in the following growing season. Finally, I will investigate potential variations in these responses across deciduous and evergreen tree species, aiming to discern whether different patterns emerge within these distinct groups.\\
\textbf {Methodology:} To investigate the impact of manipulated spring and autumn temperatures on phenological responses, I will conduct experiments across seven different tree species under controlled conditions. For deciduous trees, I have selected five species spanning both fast and short-life strategies (e.g., Populus balsamifera) and slow growth and longer lifespan species (e.g., Quercus macrocarpa). Since phenological monitoring is more difficult and trends are less likely to be observed for evergreen trees, only two of the seven species will be conifers \citep{jonsson_annual_2010}. I plan a full factorial experiment of spring and fall warming with two levels each (control/warmed) resulting in four treatments: spring or autumn warming, or both, and a control. To test that responses are not limited by nutrient depletion later in the season, I plan two additional nutrient enrichment treatments (6 total treatments across the whole experiment). For this, I will add liquid nutrients to the treatment trees in regular and warmer autumn temperature treatments. I plan of 15 replicates per species, adhering to the standards in tree phenological monitoring, which generally require 10-15 replicates \citep{siegel_collaborative_2009}.
Throughout the summer of 2025, I will continuously monitor radial growth using magnetic micro-dendrometers and track phenology every 2-3 days. In fall 2026, after the trees have grown in ambient temperatures for the season, I will assess growth on the individual (total biomass) and the cellular level (number of cells and their characteristics). \\
\textbf{Research outreach:} Given the widespread impacts of climate change on ecosystems, understanding how forest communities respond to prolonged growing seasons is crucial. Observing the reactions of deciduous and conifer species to extended season and nutrient supplementation may reveal potential benefits for some species and harm for others. These shifts are likely to influence forest stand dynamics across North America.

% References
\bibliographystyle{amnat.bst} % set citation style 
\bibliography{2024-CGSM.bib}


\end{document}
