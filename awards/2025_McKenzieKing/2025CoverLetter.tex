%------------------------------------------------------------------------------
\documentclass[11pt,letter]{article}
\usepackage[top=0.4in, bottom=0.7in, left=1in, right=1in]{geometry}
\renewcommand{\baselinestretch}{1}
\usepackage{graphicx}
\usepackage{natbib}
\usepackage{amsmath}
\usepackage{hyperref}


\parindent=0pt
\parskip=1 pt

%------------------------------------------------------------------------------
\title{Personal Statement}
\author{Christophe Rouleau-Desrochers}

\begin{document}
\pagenumbering{gobble}

\maketitle
% On an attached sheet, please describe your academic interests, extracurricular activities, and the program of studies you intend to pursue, and outline your plans for a career. Your statement must be readily understandable by someone outside your discipline, and must be complete in 600 words

%%%%%%%%%%%%%%%%%%%%%%%%%
% Academic interests %
%%%%%%%%%%%%%%%%%%%%%%%%%.
In the fall of 2020, I returned to my studies after several years in the restaurant industry, where I trained and worked as a sommelier. Recognizing that selling wine in restaurants was too far removed from wine production, I travelled to New Zealand in 2019 to work in a small-scale organic vineyard. Over six months, I realized that my deep-rooted connection to nature—shaped by my father’s work as a maple syrup producer—was stronger than I had imagined. This experience led me to enroll in the Certificate in Ecology program at the Université du Québec à Montréal, which I completed in June 2021. \\

My academic and professional journey has shaped my character by developing qualities such as boundless curiosity and rigour, which drive me to continuously improve, expand my knowledge, and refine my study techniques. After working as a research assistant on Pileated Woodpecker ecology under the supervision of Pierre Drapeau, I joined the Temporal Ecology Lab in 2023, where I have since honed my research skills under the guidance of Elizabeth Wolkovich. \\

Fueled by an insatiable thirst for knowledge and a commitment to addressing the climate crisis, I have sought to broaden my research experience. In the summer of 2023, I contributed to Frederik Baumgarten’s postdoctoral project studying the effects of climate change on tree growth and phenology (the study of recurring life history events). After witnessing my newly acquired skill set and leadership qualities, Drs. Wolkovich and Baumgarten offered me the opportunity to take over the research project in 2024—an offer I gratefully accepted. This project is one of three that will form the foundation of my Master’s thesis. It presents a significant challenge, requiring me to coordinate complex experiments and manage a team of two to four undergraduate research assistants. \\

%%%%%%%%%%%%%%%%%%%%%%%%%
%  Extracurricular activities %
%%%%%%%%%%%%%%%%%%%%%%%%%.
During my limited free time, I spend as much time outdoors as possible. Whether hiking or on a cycling expedition, ornithology is always at the back of my mind. Since joining Dr. Drapeau's lab, I have developed a deep passion for birds and love to travel to observe, identify, and photograph them. For the past two years, I have combined my long-standing passion for photography with my growing interest in wildlife. As a result, I never go anywhere without my camera and binoculars! \\

%%%%%%%%%%%%%%%%%%%%%%%%%
%  Program of studies you intend to pursue %
%%%%%%%%%%%%%%%%%%%%%%%%%.
I am currently pursuing a Master of Science in Forestry, which I plan to complete by 2026. However, because I truly love my work and feel fulfilled in my lab, I intend to transfer my Master’s into a PhD. I believe my ongoing projects have the potential to be expanded and diversified to address a fundamental question in climate change research: how will tree growth and phenology be impacted by rising temperatures and abiotic stressors? \\

%%%%%%%%%%%%%%%%%%%%%%%%%
% Outline your plans for a career  %
%%%%%%%%%%%%%%%%%%%%%%%%%.
As mentioned earlier, my immediate goal is to enroll in a PhD program at UBC. With the aspiration of becoming a university professor, I plan to pursue one or two postdoctoral fellowships after completing my PhD. Before that, I aim to broaden my research experience by collaborating with international labs that share my research interests. Janneke Hille Ris Lambers from ETH Zurich and Neil Pederson from Harvard University are two researchers I would love to work with. The McKenzie scholarship would greatly enhance my research outreach by facilitating international collaborations and conference attendance.

% References
\bibliographystyle{nature} % set citation style 
\bibliography{2024-CGSM.bib}


\end{document}
