

\documentclass{article}
\usepackage[utf8]{inputenc}
\usepackage{authblk}
\usepackage{setspace}
\usepackage{natbib}
\usepackage{hyperref}
%\usepackage{cite}
\usepackage[margin=1in]{geometry}
\usepackage{array}
\usepackage{graphicx}
\usepackage{caption}
\graphicspath{ {./figures/} }
\usepackage{subcaption}
\usepackage{amsmath}
\usepackage{lineno}
\usepackage{soul}
\usepackage{xcolor}
\sethlcolor{yellow}
\linenumbers

%%%%%%%%%%%%%%%%%
\title{Abstract CSEE}
\date{\today}
\author{Christophe Rouleau-Desrochers}

\begin{document}
%%%%%%%%%%%%%%%%%%%%%%%%%%%%%%

\maketitle

	
%<><><><><><><><><><><><><><><><><><><><><><><><><><><><><><><><><><><><><><><><>
% ABSTRACT %
% 175 words max
%<><><><><><><><><><><><><><><><><><><><><><><><><><><><><><><><><><><><><><><><>
Anthropogenic climate change, and particularly increased temperature, affects many natural systems at the global scale. The most frequently observed biological impact of climate change over the past decades is major changes in spring and autumn phenology---the timing of recurring life history events---which are likely to have direct and indirect effects. For trees, these shifts extend the growing season, suggesting an increase in growth. However, recent work did not find that longer seasons enhanced growth, potentially affecting forest carbon sequestration dynamics. Here, we address this decoupling by leveraging two unique datasets of vegetative phenology and growth (tree rings) data, one from a common garden and the other from a citizen science program. Importantly, these studies took place in an urban arboretum, which makes them unlikely to be constrained by water. Across our 14 tree species over 10 years of growing season length data spanning 111 to 157 days, we show that trees under well watered conditions did not to grow more during longer seasons. Our results support the recently observed decoupling between growing season length and growth.

\end{document}