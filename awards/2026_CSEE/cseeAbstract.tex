

\documentclass{article}
\usepackage[utf8]{inputenc}
\usepackage{authblk}
\usepackage{setspace}
\usepackage{natbib}
\usepackage{hyperref}
%\usepackage{cite}
\usepackage[margin=1in]{geometry}
\usepackage{array}
\usepackage{graphicx}
\usepackage{caption}
\graphicspath{ {./figures/} }
\usepackage{subcaption}
\usepackage{amsmath}
\usepackage{lineno}
\usepackage{soul}
\usepackage{xcolor}
\sethlcolor{yellow}
\linenumbers

%%%%%%%%%%%%%%%%%
\title{Evidence that growing season length and tree growth are decoupled in an urban arboretum}
\date{\today}
\author{Christophe Rouleau-Desrochers,  Neil Pederson, Victor van der Meersch, E.M. Wolkovich}
\begin{document}
%%%%%%%%%%%%%%%%%%%%%%%%%%%%%%

\maketitle

	
%<><><><><><><><><><><><><><><><><><><><><><><><><><><><><><><><><><><><><><><><>
% ABSTRACT %
% 175 words max
%<><><><><><><><><><><><><><><><><><><><><><><><><><><><><><><><><><><><><><><><>
Anthropogenic climate change affects many natural systems at the global scale. The most frequently observed biological impact of climate change---shifts in the timing of recurring life history events (phenology)---is likely to have cascading/additional/knock-on effects. For trees,  shifted phenology has extended the vegetative growing season, which is widely expected to increase tree growth, with important effects on forest carbon sequestration dynamics. However, multiple recent studies have failed to find this relationship and suggested shifts in drought or competition may prevent increased growth. Here, we leverage two unique datasets of vegetative phenology and growth (tree rings) data, one from a common garden and the other from a citizen science program, collected in an urban arboretum where drought and competition are limited. Across our 14 tree species over 10 years of growing season length data spanning 111 to 157 days, we found no evidence that trees grow more during longer seasons. Our results support the recently observed decoupling between growing season length and growth, but suggest it may be driven by other constraints than currently proposed.

\end{document}