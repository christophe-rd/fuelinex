

\documentclass{article}
\usepackage[utf8]{inputenc}
\usepackage{authblk}
\usepackage{setspace}
\usepackage{natbib}
\usepackage{hyperref}
%\usepackage{cite}
\usepackage[margin=1in]{geometry}
\usepackage{array}
\usepackage{graphicx}
\usepackage{caption}
\graphicspath{ {./figures/} }
\usepackage{subcaption}
\usepackage{amsmath}
\usepackage{lineno}
\usepackage{soul}
\usepackage{xcolor}
\sethlcolor{yellow}
\linenumbers

%%%%%%%%%%%%%%%%%
\title{TITLE}
\date{\today}
\author{Christophe Rouleau-Desrochers}

\begin{document}
%%%%%%%%%%%%%%%%%%%%%%%%%%%%%%

\maketitle


%<><><><><><><><><><><><><><><><><><><><><><><><><><><><><><><><><><><><><><><><>
% ABSTRACT %
% 175 words max
%<><><><><><><><><><><><><><><><><><><><><><><><><><><><><><><><><><><><><><><><>
\section{ABSTRACT}
Climate change is intensifying extreme weather events, posing significant challenges to urban and forest ecosystems. Rising temperatures, heat waves, and increasing wildfire frequency threaten biodiversity and human well-being. In urban environments, tree cover provides critical services that can help mitigate these threats. Trees help cool local areas, reducing the impact of heat waves, while also sequestering carbon and improving air quality. Additionally, exposure to trees has been linked to mental and physical health benefits. With my research, I aim to identify tree species best suited for warmer and drier conditions in cities. By leveraging data from a common garden and a community science program in an urban arboretum. I showed that while some species seem to benefit from longer seasons, others seem to have their growth potential offset, with large implications for tree growth in cities with future anthropogenic climate change. 

My MSc research has broader implications for both British Columbia forests and for environmental justice. For BC forests, my work could provide insights into selecting tree species that will have higher productivity under future climate change. For environmental justice, my work can improve green infrastructure policy to reduce disparities, which are currently high as green infrastructure is extremely poor in low-income and marginalized communities. By recommending climate-resilient species for urban planting to help refine plans to reduce these disparities, my research supports equitable access to green spaces and sustainable forest management practices that integrate both scientific and Indigenous knowledge systems.