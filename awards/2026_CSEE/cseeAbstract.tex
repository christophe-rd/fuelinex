

\documentclass{article}
\usepackage[utf8]{inputenc}
\usepackage{authblk}
\usepackage{setspace}
\usepackage{natbib}
\usepackage{hyperref}
%\usepackage{cite}
\usepackage[margin=1in]{geometry}
\usepackage{array}
\usepackage{graphicx}
\usepackage{caption}
\graphicspath{ {./figures/} }
\usepackage{subcaption}
\usepackage{amsmath}
\usepackage{lineno}
\usepackage{soul}
\usepackage{xcolor}
\sethlcolor{yellow}
\linenumbers

%%%%%%%%%%%%%%%%%
\title{TITLE}
\date{\today}
\author{Christophe Rouleau-Desrochers}

\begin{document}
%%%%%%%%%%%%%%%%%%%%%%%%%%%%%%

\maketitle

	
%<><><><><><><><><><><><><><><><><><><><><><><><><><><><><><><><><><><><><><><><>
% ABSTRACT %
% 175 words max
%<><><><><><><><><><><><><><><><><><><><><><><><><><><><><><><><><><><><><><><><>
\section{ABSTRACT}
Anthropogenic climate change, and particularly increased temperature, affects many natural systems at the global scale. The most frequently observed biological impact of climate change over the past decades is major changes in spring and autumn tree phenology---the timing of recurring life history events. These shifts extend the growing season and a long-standing assumption suggests that this leads to increased growth. However, recent work shows an absence of increased growth despite longer seasons, potentially affecting forest carbon sequestration dynamics. Therefore, we address this paradox by leveraging two unique datasets of phenological data from a common garden and a citizen science program in an urban arboretum. These are unique because as they extend for several consecutive years for the same individuals and because they make drought limitations unlikely as the trees were watered during low precipitation periods. By deriving the growing season length from these phenological observations and relating them to yearly growth with tree rings, we show no increased growth despite apparent better seasonal conditions. Our results support the paradox of a non-positive relationship between growth and season length, unlikely driven by drought.

\end{document}